\chapter{The Ethernet Model}
\label{cha:ethernet}

% FIXME: EtherBus default propagationSpeed
% TODO: comment numWirelessPorts in MacRelayUnitPP
% TODO: comment origByteLength in EtherFrame
% FIXME: wrong header length in EtherFrame.msg
% TODO: we cant send EtherFrameWithSnap

\section{Overview}

Variations: 10Mb/s ethernet, fast ethernet, Gigabit Ethernet, Fast Gigabit Ethernet, full duplex

The Ethernet model contains a MAC model (\nedtype{EtherMAC}), LLC model (\nedtype{EtherLLC}) as well
as a bus (\nedtype{EtherBus}, for modelling coaxial cable) and a hub (\nedtype{EtherHub}) model.
A switch model (\nedtype{EtherSwitch}) is also provided.

\begin{itemize}
  \item \nedtype{EtherHost} is a sample node with an Ethernet NIC;
  \item \nedtype{EtherSwitch}, \nedtype{EtherBus}, \nedtype{EtherHub} model switching hub, repeating hub and
        the old coxial cable;
  \item basic compnents of the model: \nedtype{EtherMAC}, \nedtype{EtherLLC}/\nedtype{EtherEncap} module types,
        \nedtype{MACRelayUnit} (\nedtype{MACRelayUnitNP} and \nedtype{MACRelayUnitPP}), \nedtype{EtherFrame} message type,
        \cppclass{MACAddress} class
\end{itemize}

Sample simulations:

\begin{itemize}
  \item the \nedtype{MixedLAN} model contains hosts, switch, hub and bus
  \item the \nedtype{LargeNet} model contains hundreds of computers, switches and hubs
        (numbers depend on model configuration in largenet.ini) and mixes all
        kinds of Ethernet technologies
\end{itemize}

\subsection{Implemented Standards}

The Ethernet model operates according to the following standards:

\begin{itemize}
  \item Gigabit Ethernet: IEEE 802.3z-1998
  \item Full-Duplex Ethernet with Flow Control: IEEE 802.3x-1997
  \item Fast Ethernet: IEEE 802.3u-1995
  \item Ethernet: IEEE 802.3-1998
\end{itemize}

Note: switches don't implement the Spanning Tree Protocol. You need to
avoid cycles in the LAN topology.

\section{Physical layer}

\subsection{EtherBus}

A generic bus model.

The ethg[i] gates represent taps. Messages arriving on a tap
travel on the bus on both directions, and copies of it are sent out
on every other tap after delays proportional to their distances.

Tap connections should have zero delays and zero data rates!

Messages are not interpreted by the bus model in any way, thus the bus
model is not specific to Ethernet in any way. Messages may
represent anything, from the beginning of a frame transmission to
end (or abortion) of transmission.

% FIXME: NED comment is wrong: data rate must not be zero!

\subsection{EtherHub}

A generic wiring hub model.

ethg[i] gates represent ports. Messages arriving on a port are broadcast
to every other port.

The connections should have data rate set to zero. Cable lengths
should be reflected in the delays of the connections.

Messages are not interpreted by the hub model in any way, thus the hub
model is not specific to Ethernet in any way. Messages may
represent anything, from the beginning of a frame transmission to
end (or abortion) of transmission.

each connection datarate must have the same datarate (should not be zero)

% FIXME: NED comment is wrong: data rate must not be zero!
% FIXME: default propagation speed is wrong (should be 2e8mps)
%        btw there is a hard coded propagation speed in EtherMACBase.cc

\section{MAC layer}

\subsection{IEtherMac}

defined in inet.linklayer.base

\subsection{EtherMAC}

Ethernet MAC layer. MAC performs transmission and reception of frames.
See the ~IEtherMAC for the Ethernet MAC layer general informations.
Doesn't do encapsulation/decapsulation; see ~EtherLLC and ~EtherEncap for
that.

Supported variations:
- 10Mb Ethernet bus or twisted pair
- 100Mb Ethernet duplex or half-duplex
- 1Gb Ethernet

Supports all three Ethernet frame types. (It handles ~EtherFrame message class;
specific frame classes (Ethernet-II, IEEE 802.3) are subclassed from that one.)
RAW mode (only used by the IPX protocol) is not supported.

Expected environment:
- phys$i and phys$o should be connected to the "network"
- upperLayerIn and upperLayerOut are usually connected to ~EtherLLC (in hosts)
  or ~MACRelayUnitPP (in a switch)

<b>Operation</b>

Processing of frames received from higher layers:
- if src address in the frame is empty, fill it out
- frames get queued up until transmission
- transmit according to the CSMA/CD protocol
- can send PAUSE message if requested by higher layers (PAUSE protocol,
  used in switches).

Processing of frames incoming from the network:
- receive according to the CSMA/CD protocol
- CRC checking (frames with the error bit set are discarded).
- respond to PAUSE frames
- in promiscuous mode, pass up all received frames;
  otherwise, only frames with matching MAC addresses and
  broadcast frames are passed up.

The module does not perform encapsulation or decapsulation of frames --
this is done by higher layers (~EtherLLC or ~EtherEncap).

When a frame is received from the higher layers, it must be an ~EtherFrame,
and with all protocol fields filled out
(including the destination MAC address). The source address, if left empty,
will be filled in. Then frame is queued and transmitted according
to the CSMA/CD protocol.

Data frames received from the network are EtherFrames. They are passed to
the higher layers without modification.
Also, the module properly responds to PAUSE frames, but never sends them
by itself -- however, it transmits PAUSE frames received from upper layers.
See <a href="ether-pause.html">PAUSE handling</a> for more info.

For more info see <a href="ether-overview.html">Ethernet Model Overview</a>.

<b>Disabling and disconnecting</b>

If the MAC is not connected to the network ("cable unplugged"), it will
start up in "disabled" mode. A disabled MAC simply discards any messages
it receives. It is currently not supported to dynamically connect/disconnect
a MAC.


<b>Queueing</b>

In routers, MAC relies on an external queue module (see ~OutputQueue)
to model finite buffer, implement QoS and/or RED, and requests packets
from this external queue one-by-one.

In hosts, no such queue is used, so MAC contains an internal
queue named txQueue to queue up packets waiting for transmission.
Conceptually, txQueue is of infinite size, but for better diagnostics
one can specify a hard limit in the txQueueLimit parameter -- if this is
exceeded, the simulation stops with an error.


<b>Physical layer messaging</b>

Please see <a href="physical.html">Messaging on the physical layer</a>.

<b>Statistics</b>

Output vectors:
- txPkBytes: bytes sent, including Ethernet frame fields (but excluding preamble and SFD)
- rxPkBytesOK: total bytes received  without collision or CRC error,
  including Ethernet frame fields (but excluding preamble and SFD)
- passedUpPkBytes: number of bytes of frames actually passed up to higher layer
- txPausePkUnits: number of PAUSE frames sent out
- rxPausePkUnits: number of PAUSE frames received from network
- rxPkBytesFromHL: number of bytes of frames received from higher layer
- collision: number of collisions (NOT number of collided frames!) sensed
- backoff: number of retransmissions
- droppedPkBytesIfaceDown: number of bytes of frames from higher layer dropped
- droppedPkBytesBitError: number of bytes of frames dropped because of bit errors
- droppedPkBytesNotForUs: number of bytes of frames dropped because destination address didn't match

Output scalars (written in the finish() function) include the final values of
the above variables and throughput.

@see ~EtherMACFullDuplex, ~EthernetInterface, ~IOutputQueue, ~EtherEncap, ~EtherLLC
@see ~EtherFrame, ~EthernetIIFrame, ~EtherFrameWithLLC, ~Ieee802Ctrl

\subsection{EtherMACFullDuplex}


Ethernet MAC which supports full duplex operation ONLY.
See the ~IEtherMAC for general informations.

Most of today's Ethernet networks are switched, and operate
in full duplex mode. Full-duplex transmission can be used for
point-to-point connections only. Since full-duplex connections
cannot be shared, collisions are eliminated. This setup eliminates
most of the need for the CSMA/CD access control mechanism because
there is no need to determine whether the connection is already
being used. This allows for a much simpler simulation model
for MAC. (In "traditional" Ethernet simulations, most of the code
deals with the shared medium and the CSMA/CD mechanism.)
~EtherMACFullDuplex implements Ethernet without shared medium and CSMA/CD.
(If you need half-duplex operation, see ~EtherMAC which is for a full-blown
and therefore more complicated Ethernet MAC model.)

~EtherMACFullDuplex performs transmission and reception of frames.
It does not do encapsulation/decapsulation; see ~EtherLLC and ~EtherEncap
for that.

Supported variations:
- 10Mb Ethernet (full duplex mode)
- 100Mb Ethernet (full duplex mode)
- 1Gb Ethernet (full duplex mode)

Supports all three Ethernet frame types. (It handles ~EtherFrame message class;
specific frame classes (Ethernet-II, IEEE 802.3) are subclassed from that one.)
RAW mode (only used by the IPX protocol) is not supported.

<b>Operation</b>

Processing of frames received from higher layers:
- if src address in the frame is empty, fill it out
- frames get queued up until transmission
- transmit according to the CSMA/CD protocol
- can send PAUSE message if requested by higher layers (PAUSE protocol,
  used in switches).

Processing of frames incoming from the network:
- receive according to the CSMA/CD protocol
- CRC checking (frames with the error bit set are discarded).
- respond to PAUSE frames
- in promiscuous mode, pass up all received frames;
  otherwise, only frames with matching MAC addresses and
  broadcast frames are passed up.

The module does not perform encapsulation or decapsulation of frames --
this is done by higher layers (~EtherLLC or ~EtherEncap).

When a frame is received from the higher layers, it must be an ~EtherFrame,
and with all protocol fields filled out
(including the destination MAC address). The source address, if left empty,
will be filled in. Then frame is queued and transmitted according
to the CSMA/CD protocol.

Data frames received from the network are EtherFrames. They are passed to
the higher layers without modification.
Also, the module properly responds to PAUSE frames, but never sends them
by itself -- however, it transmits PAUSE frames received from upper layers.
See <a href="ether-pause.html">PAUSE handling</a> for more info.

<b>Disabling and disconnecting</b>

If the MAC is not connected to the network ("cable unplugged"), it will
start up in "disabled" mode. A disabled MAC simply discards any messages
it receives. It is currently not supported to dynamically connect/disconnect
a MAC.


<b>Queueing</b>

In routers, MAC relies on an external queue module (see ~IOutputQueue)
to model finite buffer, implement QoS and/or RED, and requests packets
from this external queue one-by-one.

In hosts, no such queue is used, so MAC contains an internal
queue named txQueue to queue up packets waiting for transmission.
Conceptually, txQueue is of infinite size, but for better diagnostics
one can specify a hard limit in the txQueueLimit parameter -- if this is
exceeded, the simulation stops with an error.


<b>Physical layer messaging</b>

Please see <a href="physical.html">Messaging on the physical layer</a>.

<b>Statistics</b>

Output vectors:
- txPkBytes: bytes sent, including Ethernet frame fields (but excluding preamble and SFD)
- rxPkBytesOK: total bytes received  without collision or CRC error,
  including Ethernet frame fields (but excluding preamble and SFD)
- passedUpPkBytes: number of bytes of frames actually passed up to higher layer
- txPausePkUnits: number of PAUSE frames sent out
- rxPausePkUnits: number of PAUSE frames received from network
- rxPkBytesFromHL: number of bytes of frames received from higher layer
- droppedPkBytesIfaceDown: number of bytes of frames from higher layer dropped
- droppedPkBytesBitError: number of bytes of frames dropped because of bit errors
- droppedPkBytesNotForUs: number of bytes of frames dropped because destination address didn't match

Output scalars (written in the finish() function) include the final values of
the above variables and throughput.

% TODO notifications: NF_PP_TX_BEGIN, NF_PP_TX_END, NF_PP_RX_END

\subsection{Messaging on the Physical Layer}

Messages sent by \nedtype{EtherMAC} mark the beginning of a transmission.
The end of a transmission is not explicitly represented by a message,
but instead, the \nedtype{EtherMAC} calculates it from the frame length and
the transmission rate. Frames are represented by \msgtype{EtherFrame}.

When frames collide, the transmission is aborted -- in this case
\nedtype{EtherMAC} makes use of the modelled jam signals to figure out
when colliding transmissions end.

When a transmitting station senses a collision, it transmits a jam signal.
Jam signals are represented by a \msgtype{EtherJam} message.
When \nedtype{EtherMAC} received a jam signal, it knows that one transmission
has ended in jamming -- thus when it receives as many jam messages
as colliding frames, it can be sure all transmissions have been aborted.

Receiving a jam message marks the beginning (and not the end)
of a jam signal, so actually \nedtype{EtherMAC} has to wait for the duration
of the jamming before assuming the channel is free again.

\subsection{Autoconfiguration}

In order to facilitate building large models, \nedtype{EtherMAC} and other Ethernet model
components provide some degree of auto-configuration. Specifically, transmission
rate and half duplex/full duplex mode can be chosen automatically so that
connecting Ethernet MACs have matching settings. The purpose is similar to
Ethernet Auto-Negotiation; however the mechanism is NOT a model of
Auto-Negotiation (e.g. \nedtype{EtherBus} and \nedtype{EtherHub} also actively participate,
which obviously does not happen in a real Ethernet.)

What it does:

\begin{itemize}
  \item the txrate parameters of \nedtype{EtherMAC} represent the highest speed supported
        by that station, or 0 for full autoconfiguration. Autoconfig will choose
        the largest common denominator (the speed of the slowest station in the
        collision domain) for all stations. If all stations are set to auto
        txrate, 100Mb will be chosen, or 10Mb if there's a bus (\nedtype{EtherBus}) in
        the collision domain.
  \item the duplexEnabled parameter of \nedtype{EtherMAC} means whether the station supports
        duplex operation. However, duplex operation will actually be chosen only
        if it's a DTE-to-DTE direct connection (there's no shared media like
        \nedtype{EtherHub} or \nedtype{EtherBus}) and both sides have duplexEnabled=true set.
\end{itemize}

How it works:

Auto-configuration occurs at the beginning of the simulation, by
Ethernet model components (\nedtype{EtherMAC}, \nedtype{EtherHub} and \nedtype{EtherBus}) exchanging
\msgtype{EtherAutoconfig} messages with each other. See description of \msgtype{EtherAutoconfig}
for more info.

\subsection{MAC and higher layers}

MAC and LLC are implemented as separate modules (\nedtype{EtherMAC} and
\nedtype{EtherLLC}/\nedtype{EtherEncap}) because encapsulation/decapsulation functionality
is not always needed. (Switches don't do encapsulation/decapsulation.)
In switches, \nedtype{EtherMAC} is used with \nedtype{MACRelayUnit}.

\subsection{PAUSE handling}

The 802.3x standard supports PAUSE frames as a means of flow
control. The frame contains a timer value, expressed as a multiple
of 512 bit-times, that specifies how long the transmitter should
remain quiet. If the receiver becomes uncongested before the
transmitter's pause timer expires, the receiver may elect to send
another Pause frame to the transmitter with a timer value of zero,
allowing the transmitter to resume immediately.

\nedtype{EtherMAC} will properly respond to PAUSE frames it receives
(\msgtype{EtherPauseFrame} class),
however it will never send a PAUSE frame by itself. (For one thing,
it doesn't have an input buffer that can overflow.)

\nedtype{EtherMAC}, however, transmits PAUSE frames received by higher layers,
and \nedtype{EtherLLC} can be instructed by a command to send a PAUSE frame to MAC.

\nedtype{MACRelayUnit} types (and thus \nedtype{EtherSwitch}) currently implement a very simple
scheme for sending PAUSE frames -- this can be refined if the need arises.


\section{Switches}

\subsection{MACRelayUnitNP}

A IMACRelayUnit implementation which models one or more CPUs
with shared memory, working from a single shared queue.

It also models fixed delay for precessing each frame.
Finite memory is taken into account by dropping frames if
total number of bits enqueued exceed a given limit.

A simple scheme for sending PAUSE frames is built in (although
users will probably change it). When the buffer level goes
above a high watermark, PAUSE frames are sent on all ports.
The watermark and the pause time is configurable; use zero
values to disable the PAUSE feature.

\subsection{MACRelayUnitPP}

A ~IMACRelayUnit implementation which models one CPU assigned to each
incoming port, working with shared memory but separate queues.

It also models fixed delay for precessing each frame.
Finite memory is taken into account by dropping frames if
total number of bits enqueued exceed a given limit.

A simple scheme for sending PAUSE frames is built in (although
users will probably change it). When the buffer level goes
above a high watermark, PAUSE frames are sent on all ports.
The watermark and the pause time is configurable; use zero
values to disable the PAUSE feature.

\subsection{EtherSwitch}

Model of an Ethernet switch.

The duplexChannel attributes of the MACs must be set according to the
medium connected to the port; if collisions are possible (it's a bus or hub)
it must be set to false, otherwise it can be set to true.
By default used half duples CSMA/CD mac

This model does not contain the spanning tree algorithm.

\section{Link Layer Control}

% FIXME there is no module for sending EtherFrameWithSNAP frames
% FIXME ETHER_SNAP_HEADER_LENGTH is wrong in Ethernet.h (+3 bytes, ssap+dsap+control)

\subsection{Frame types}

The raw 802.3 frame format header contains the MAC addresses of the destination and source
of the packet and the length of the data field. The frame footer contains the FCS
(Frame Check Sequence) field which is a 32-bit CRC.

\begin{center}
\begin{bytefield}[bitwidth=1.2em,bitheight=2\baselineskip]{20}
\bitbox{4}{\small MAC \\ destination} &
\bitbox{4}{\small MAC \\ source} &
\bitbox{4}{\small Length} &
\bitbox{4}{\small Payload} &
\bitbox{4}{\small FCS} \\
\bitbox{4}{\small 6 octets} &
\bitbox{4}{\small 6 octets} &
\bitbox{4}{\small 2 octets} &
\bitbox{4}{\small 46-1500 octets} &
\bitbox{4}{\small 4 octets}
\end{bytefield}
\end{center}

Each such frame is preceded by a 7 octet Preamble (with 10101010 octets) and
a 1 octet SFD (Start of Frame Delimiter) field (10101011) and followed by an
12 octet interframe gap. These fields are added and removed in the MAC layer,
so they are omitted here.

When multiple upper layer protocols use the same Ethernet line,
the kernel has to know which which component handles the incoming frames.
For this purpose a protocol identifier was added to the standard Ethernet
frames.

The first solution preceded the 802.3 standard and used a 2 byte protocol
identifier in place of the Length field. This is called Ethernet II
or DIX frame.
Each protocol id is above 1536, so the Ethernet II frames and the 802.3
frames can be distinguished.

\begin{center}
\begin{bytefield}[bitwidth=1.2em,bitheight=2\baselineskip]{20}
\bitbox{4}{\small MAC \\ destination} &
\bitbox{4}{\small MAC \\ source} &
\bitbox{4}{\small EtherType} &
\bitbox{4}{\small Payload} &
\bitbox{4}{\small FCS} \\
\bitbox{4}{\small 6 octets} &
\bitbox{4}{\small 6 octets} &
\bitbox{4}{\small 2 octets} &
\bitbox{4}{\small 46-1500 octets} &
\bitbox{4}{\small 4 octets}
\end{bytefield}
\end{center}

The LLC frame format uses a 1 byte source, a 1 byte destination, and a 1 byte
control information to identify the encapsulated protocol adopted from the
802.2 standard. These fields follow the standard 802.3 header, so the maximum
length of the payload is 1497 bytes:

\begin{center}
\begin{bytefield}[bitwidth=1.2em,bitheight=2\baselineskip]{20}
\bitbox{4}{\small MAC \\ destination} &
\bitbox{4}{\small MAC \\ source} &
\bitbox{4}{\small Length} &
\bitbox{4}{\small DSAP} &
\bitbox{4}{\small SSAP} &
\bitbox{4}{\small Control} &
\bitbox{4}{\small Payload} &
\bitbox{4}{\small FCS} \\
\bitbox{4}{\small 6 octets} &
\bitbox{4}{\small 6 octets} &
\bitbox{4}{\small 2 octets} &
\bitbox{4}{\small 1 octets} &
\bitbox{4}{\small 1 octets} &
\bitbox{4}{\small 1 octets} &
\bitbox{4}{\small 43-1497 octets} &
\bitbox{4}{\small 4 octets}
\end{bytefield}
\end{center}

The SNAP header uses the EtherType protocol identifiers inside an LLC header.
The SSAP and DSAP fields are filled with 0xAA (SAP\_SNAP), and the control
field is 0x03. They are followed by a 3 byte orgnaization and a 2 byte local
code the identify the protocol. If the organization code is 0, the local field
contains an EtherType protocol identifier.

\begin{center}
\begin{bytefield}[bitwidth=1.2em,bitheight=2\baselineskip]{20}
\bitbox{4}{\small MAC \\ destination} &
\bitbox{4}{\small MAC \\ source} &
\bitbox{4}{\small Length} &
\bitbox{4}{\small DSAP \\ 0xAA} &
\bitbox{4}{\small SSAP \\ 0xAA} &
\bitbox{4}{\small Control \\ 0x03} &
\bitbox{4}{\small OrgCode} &
\bitbox{4}{\small Local \\ Code} &
\bitbox{4}{\small Payload} &
\bitbox{4}{\small FCS} \\
\bitbox{4}{\small 6 octets} &
\bitbox{4}{\small 6 octets} &
\bitbox{4}{\small 2 octets} &
\bitbox{4}{\small 1 octets} &
\bitbox{4}{\small 1 octets} &
\bitbox{4}{\small 1 octets} &
\bitbox{4}{\small 3 octets} &
\bitbox{4}{\small 2 octets} &
\bitbox{4}{\small 38-1492 octets} &
\bitbox{4}{\small 4 octets}
\end{bytefield}
\end{center}

The INET defines these frames in the \ffilename{EtherFrame.msg} file.
The models supports Ethernet II, 803.2 with LLC header, and 803.3 with LLC and SNAP headers.
The corresponding classes are:
\msgtype{EthernetIIFrame}, \msgtype{EtherFrameWithLLC} and \msgtype{EtherFrameWithSNAP}. They all class
from \msgtype{EtherFrame} which only represents the basic MAC frame with source and
destination addresses. \nedtype{EtherMAC} only deals with \msgtype{EtherFrame}s, and does not
care about the specific subclass.

Ethernet frames carry data packets as encapsulated cMessage objects.
Data packets can be of any message type (cMessage or cMessage subclass).

The model encapsulates data packets in Ethernet frames using the \ttt{encapsulate()}
method of cMessage. Encapsulate() updates the length of the Ethernet frame too,
so the model doesn't have to take care of that.

The fields of the Ethernet header are passed in a \cppclass{Ieee802Ctrl}
control structure to the LLC by the network layer.

EtherJam, EtherPadding (interframe gap), EtherPauseFrame?


\subsection{EtherEncap}

EtherFrameII

\subsection{EtherLLC}

EtherFrameWithLLC

% TODO delete EtherLLC, because LLC without SNAP is not used with IP (no ARP,IPv6 SAP)
% TODO modify EtherEncap to handle EtherFrameWithSNAP frames too (we can not send EtherFrameWithSNAP now)

\subsubsection{\nedtype{EtherLLC} and higher layers}

The \nedtype{EtherLLC} module can serve several applications (higher layer protocols),
and dispatch data to them. Higher layers are identified by DSAP.
See section "Application registration" for more info.

\nedtype{EtherEncap} doesn't have the functionality to dispatch to different
higher layers because in practice it'll always be used with IP.

\subsubsection{Communication between LLC and Higher Layers}

Higher layers (applications or protocols) talk to the \nedtype{EtherLLC} module.

When a higher layer wants to send a packet via Ethernet, it just
passes the data packet (a cMessage or any subclass) to \nedtype{EtherLLC}.
The message kind has to be set to IEEE802CTRL\_DATA.

In general, if \nedtype{EtherLLC} receives a packet from the higher layers,
it interprets the message kind as a command. The commands include
IEEE802CTRL\_DATA (send a frame), IEEE802CTRL\_REGISTER\_DSAP (register highher layer)
IEEE802CTRL\_DEREGISTER\_DSAP (deregister higher layer) and IEEE802CTRL\_SENDPAUSE
(send PAUSE frame) -- see EtherLLC for a more complete list.

The arguments to the command are NOT inside the data packet but
in a "control info" data structure of class \cppclass{Ieee802Ctrl}, attached to
the packet. See controlInfo() method of cMessage (OMNeT++ 3.0).

For example, to send a packet to a given MAC address and protocol
identifier, the application sets the data packet's message kind
to ETH\_DATA ("please send this data packet" command),
fills in the \nedtype{Ieee802Ctrl} structure with the destination MAC address and
the protocol identifier, adds the control info to the message, then sends
the packet to \nedtype{EtherLLC}.

When the command doesn't involve a data packet (e.g.
IEEE802CTRL\_(DE)REGISTER\_DSAP, IEEE802CTRL\_SENDPAUSE), a dummy packet
(empty cMessage) is used.

\subsubsection{Rationale}

The alternative of the above communications would be:

\begin{itemize}
  \item adding the parameters such as destination address into the data
    packet. This would be a poor solution since it would make the
    higher layers specific to the Ethernet model.
  \item encapsulating a data packet into an \textit{interface packet} which
    contains the destination address and other parameters. The
    disadvantages of this approach is the overhead associated with
    creating and destroying the interface packets.
\end{itemize}

Using a control structure is more efficient than the interface packet
approach, because the control structure can be created once inside
the higher layer and be reused for every packet.

It may also appear to be more intuitive in Tkenv because one can observe
data packets travelling between the higher layer and Ethernet
modules -- as opposed to "interface" packets.


\subsubsection{EtherLLC: SAP Registration}

The Ethernet model supports multiple applications or higher layer
protocols.

So that data arriving from the network can be dispatched to the
correct applications (higher layer protocols), applications
have to register themselves in \nedtype{EtherLLC}. The registration
is done with the IEEE802CTRL\_REGISTER\_DSAP command
(see section "Communication between LLC and higher layers")
which associates a SAP with the LLC port. Different applications
have to connect to different ports of \nedtype{EtherLLC}.

The ETHERCTRL\_REGISTER\_DSAP/IEEE802CTRL\_DEREGISTER\_DSAP commands use only the
dsap field in the \cppclass{Ieee802Ctrl} structure.

\subsection{EthernetInterface module}

The \nedtype{EthernetInterface} compound module implements the \nedtype{IWiredNic}
interface. Complements \nedtype{EtherMAC} and \nedtype{EtherEncap} with an output queue
for QoS and RED support. It also has configurable input/output filters as \nedtype{IHook}
components similarly to the \nedtype{PPPInterface} module.

% TODO there is no IWiredNic with EtherLLC

\section{Ethernet applications}

The \nedtype{inet.applications.ethernet} package contains modules
for a simple client-server application. The \nedtype{EtherAppCli} is a simple
traffic generator that peridically sends \msgtype{EtherAppReq} messages
whose length can be configured. destAddress, startTime,waitType, reqLength, respLength

The server component of the model (\nedtype{EtherAppSrv}) responds with a
\msgtype{EtherAppResp} message of the requested length. If the response does
not fit into one ethernet frame, the client receives the data in multiple
chunks.

% FIXME reqLength>1500 causes an error in the LLC module
% FIXME numFrames field of EtherAppRes is not used
% FIXME server always sends 1497 byte chunks, it should depend on the framing (1497 is for LLC)
% FIXME if registerSAP is false (default), the and EtherLLC used, then the client won't receive messages (auto config?)
% FIXME Ieee802Nic -> EthernetInterface in the NED comment

Both applications have a \fpar{registerSAP} boolean parameter.
This parameter should be set to \ttt{true} if the application is connected
to the \nedtype{EtherLLC} module which requires registration of the SAP
before sending frames.

Both applications collects the following statistics: sentPkBytes, rcvdPkBytes,
endToEndDelay.

The client and server application works with any model that accepts
Ieee802Ctrl control info on the packets (e.g. the 802.11 model).
The applications should be connected directly to the \nedtype{EtherLLC}
or an EthernetInterface NIC module.

The model also contains a host component that groups the applications
and the LLC and MAC components together (\nedtype{EtherHost}). This node does
not contain higher layer protocols, it generates Ethernet traffic directly.
By default it is configured to use half duplex MAC (CSMA/CD).

\section{Ethernet networks}

\subsection{\nedtype{LargeNet} model}

The \nedtype{LargeNet} model demonstrates how one can put together models of large
LANs with little effort, making use of MAC auto-configuration.

\nedtype{LargeNet} models a large Ethernet campus backbone. As configured in the
default omnetpp.ini, it contains altogether about 8000 computers
and 900 switches and hubs. This results in about 165MB process size
on my (32-bit) linux box when I run the simulation.
The model mixes all kinds of Ethernet technology: Gigabit Ethernet,
100Mb full duplex, 100Mb half duplex, 10Mb UTP, 10Mb bus ("thin Ethernet"),
switched hubs, repeating hubs.

The topology is in \nedtype{LargeNet}.ned, and it looks like this: there's chain
of n=15 large "backbone" switches (switchBB[]) as well as four more
large switches (switchA, switchB, switchC, switchD) connected to
somewhere the middle of the backbone (switchBB[4]). These 15+4 switches
make up the backbone; the n=15 number is configurable in omnetpp.ini.

Then there're several smaller LANs hanging off each backbone switch.
There're three types of LANs: small, medium and large (represented by
compound module types \nedtype{SmallLAN}, \nedtype{MediumLAN}, \nedtype{LargeLAN}). A small LAN
consists of a few computers on a hub (100Mb half duplex); a medium
LAN consists of a smaller switch with a hub on one of its port
(and computers on both); the large one also has a switch and a hub,
plus an Ethernet bus hanging of one port of the hub (there's still hubs
around with one BNC connector besides the UTP ones).
By default there're 5..15 LANs of each type hanging off each backbone
switch. (These numbers are also omnetpp.ini parameters like the length
of the backbone.)

The application model which generates load on the simulated LAN is
simple yet powerful. It can be used as a rough model for any
request-response based protocol such as SMB/CIFS (the Windows file
sharing protocol), HTTP, or a database client-server protocol.

Every computer runs a client application (\nedtype{EtherAppCli}) which connects
to one of the servers. There's one server attached to switches A, B,
C and D each: serverA, serverB, serverC and serverD -- server selection
is configured in omnetpp.ini). The servers run \nedtype{EtherAppSrv}.
Clients periodically send a request to the server, and the request
packet contains how many bytes the client wants the server to send back
(this can mean one or more Ethernet frames, depending on the byte count).
 Currently the request and reply lengths are configured in omnetpp.ini
as intuniform(50,1400) and truncnormal(5000,5000).

The volume of the traffic can most easily be controlled with the
time period between sending requests; this is currently
set in omnetpp.ini to exponential(0.50) (that is, average 2
requests per second). This already causes frames to be dropped
in some of the backbone switches, so the network is a bit
overloaded with the current settings.

The model generates extensive statistics. All MACs (and most other
modules too) write statistics into omnetpp.sca at the end
of the simulation: number of frames sent, received, dropped, etc.
These are only basic statistics, however it still makes the
scalar file to be several ten megabytes in size. You can use
the analysis tools provided with OMNeT++ to visualized the data
in this file. (If the file size is too big, writing statistics
can be disabled, by putting **.record-scalar=false in the ini file.)
The model can also record output vectors, but this is currently
disabled in omnetpp.ini because the generated file can easily reach
gigabyte sizes.

%%% Local Variables:
%%% mode: latex
%%% TeX-master: "usman"
%%% End:

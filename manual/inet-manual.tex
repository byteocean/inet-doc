\documentclass{book}
\usepackage{a4wide}

%% possible fonts -- in order of preference
%%\usepackage{palatino}
\usepackage{bookman}
%%\usepackage{charter}
%%\usepackage{newcent}
%%\usepackage{times}
%%\usepackage{avant}
%%\usepackage{helvet}
%%\usepackage{sans}
%%\usepackage{chancery}

\usepackage[T1]{fontenc}
\usepackage{setspace}
\usepackage{ifpdf}
\usepackage{makeidx}
\usepackage{longtable}  %% page wrapping table environment
\usepackage{colortbl}   %% colors for tables
\usepackage{fancyvrb}   %% the "Verbatim" environment
\usepackage{fancyhdr}   %% custom headers and footers
\usepackage{multicol}
\usepackage{listings}   %% source code listings with syntax highlight (lstxxx commands)
\usepackage[tight]{shorttoc}   %% for generating a second table of contents, only containing chapter titles
\usepackage{bytefield}  %% for drawing protocol frames
\usepackage{paralist}   %% for compact lists
\usepackage{tocbibind}  %% makes Bibliography and Index show up in TOC
\settocbibname{References}

\setlength{\textwidth}{160mm}
%\setlength{\oddsidemargin}{12.5mm}
%\setlength{\evensidemargin}{12.5mm}
%\setlength{\topmargin}{0mm}
\setlength{\textheight}{220mm}
%\setlength{\parskip}{1ex}
%\setlength{\parindent}{5ex}

\renewcommand{\bottomfraction}{0.9}
\renewcommand{\topfraction}{0.9}
\renewcommand{\floatpagefraction}{0.9}

%% try to cure overfull hboxes
%% \tolerance=500

%% for navigation in dvi files, only needed by old teTeX versions
%%\usepackage{srcltx}


%% try this for spell checking: cat ess2002.tex | ispell -l -t -a | sort | uniq | more

%%
%% OMNeT++ logo, use as {\opp}
%%
\makeatletter
%%\DeclareRobustCommand{\omnetpp}{OM\-NeT\kern-.18em++\@}
\DeclareRobustCommand{\omnetpp}{OMNeT++\@}
\makeatother

\newcommand{\opp}{\omnetpp}

%%
%% PDF Header
%%
% note: \ifpdf now comes from the ifpdf package
%\newif\ifpdf
%\ifx\pdfoutput\undefined
%  \pdffalse
%\else
%  \pdfoutput=1
%  \pdftrue
%\fi
%% PDF-Info
\ifpdf
  \usepackage[pdftex]{graphicx}
  \usepackage[plainpages=false,linktocpage,bookmarksnumbered=true,pdftex]{hyperref}   %% automatic hyperlinking
  \pdfcompresslevel=9
  \pdfinfo{/Author (Andras Varga and others)
    /Title (INET Framework Manual)
    /Subject ()
    /Keywords (INET, INETMANET, OMNeT++, manual)}
\else
  \usepackage{graphicx}
  \usepackage[plainpages=false]{hyperref}   %% automatic hyperlinking
\fi

%%
%% Generate Index
%%
\makeindex


%%
%% Link colors (hyperref package)
%%
\definecolor{MyDarkBlue}{rgb}{0.16,0.16,0.5}
%% XXX the next line apparently screws up all links except in TOC! they'll be colored nicely, but won't work.
%\hypersetup{
%    colorlinks=true,
%    linkcolor=MyDarkBlue,
%    anchorcolor=MyDarkBlue,
%    citecolor=MyDarkBlue,
%    filecolor=MyDarkBlue,
%    menucolor=MyDarkBlue,
%    runcolor=MyDarkBlue,
%    urlcolor=blue,
%}

%%
%% Heading and Footer
%%
\pagestyle{fancy}
\fancyhf{}
\renewcommand{\footrulewidth}{0.5pt}
\renewcommand{\chaptermark}[1]{\markboth{#1}{}}
\lhead{{\opp} Manual -- \leftmark}
\rfoot{\thepage}

%% this is used for chapter start pages
\fancypagestyle{plain}{
    \rfoot{\thepage}
}

%%
%% Use \begin{graybox}...\end{graybox} for notes
%%
\definecolor{MyGray}{rgb}{0.85,0.85,1.0}
\makeatletter\newenvironment{graybox}%
   {\begin{flushright}\begin{lrbox}{\@tempboxa}\begin{minipage}[r]{0.95\textwidth}}%
   {\end{minipage}\end{lrbox}\colorbox{MyGray}{\usebox{\@tempboxa}}\end{flushright}}%
\makeatother


\newenvironment{note}{\begin{graybox}\textbf{NOTE: }}{\end{graybox}}
\newenvironment{warning}{\begin{graybox}\textbf{WARNING: }}{\end{graybox}}
\newenvironment{caution}{\begin{graybox}\textbf{CAUTION: }}{\end{graybox}}
\newenvironment{rationale}{\begin{graybox}\textbf{Rationale: }}{\end{graybox}}
\newenvironment{important}{\begin{graybox}\textbf{IMPORTANT: }}{\end{graybox}}

%%
%% Set up listings package
%%
\lstloadlanguages{C++,make,perl,tcl,XML,R,Matlab}

%% See listings.pdf,pp20
\lstdefinelanguage{NED} {
    morekeywords={allowunconnected,bool,channel,channelinterface,connections,const,
                  default,double,extends,false,for,gates,if,import,index,inout,input,
                  int,like,module,moduleinterface,network,output,package,parameters,
                  property,simple,sizeof,string,submodules,this,true,types,volatile,
                  xml,xmldoc},
    sensitive=true,
    morecomment=[l]{//},
    morestring=[b]",
}
\lstdefinelanguage{MSG} {
    morekeywords={abstract,bool,char,class,cplusplus,double,enum,extends,false,
                  fields,int,long,message,namespace,noncobject,packet,properties,
                  readonly,short,string,struct,true,unsigned},
    sensitive=true,
    morecomment=[l]{//},
    morestring=[b]",
}
\lstdefinelanguage{inifile} {
    morekeywords={},
    sensitive=true,
    morecomment=[l]{\#},
    morestring=[b]",
}
\lstdefinelanguage{pseudocode} {
    morekeywords={if,then,else,otherwise,whenever,while},
    sensitive=true,
    morecomment=[l]{//},
    morestring=[b]",
    mathescape=true,
}

%% thick ruler on the left; also, designate backtick as LaTeX escape character
%% (e.g. \opp needs to be written as `\opp` inside listing blocks)
\lstset{
    escapechar=`,
    basicstyle=\ttfamily,
    showstringspaces=false,
    frame=leftline,
    framesep=10pt,
    framerule=3pt,
    xleftmargin=15pt
}

\definecolor{NEDRulerColor}{rgb}{0.8,1.0,0.8}  % pale green
\definecolor{MSGRulerColor}{rgb}{0.8,1.0,0.8}  % pale green
\definecolor{CPPRulerColor}{rgb}{0.8,0.8,1.0}  % pale blue
\definecolor{IniRulerColor}{rgb}{0.9,0.9,0.2}  % pale yellow
\definecolor{FileListingRulerColor}{rgb}{0.85,0.85,0.85}  % grey
%\definecolor{CommandLineRulerColor}{rgb}{0.9,0.9,0.2}
\definecolor{PseudoCodeRulerColor}{rgb}{0.0,1.0,1.0}  % cyan

%% See listings.pdf,pp39
\lstnewenvironment{ned}
    {\lstset{language=NED,rulecolor=\color{NEDRulerColor}}}
    {}
\lstnewenvironment{msg}
    {\lstset{language=MSG,rulecolor=\color{MSGRulerColor}}}
    {}
\lstnewenvironment{cpp}
    {\lstset{language=C++,rulecolor=\color{CPPRulerColor}}}
    {}
\lstnewenvironment{inifile}
    {\lstset{language=inifile,rulecolor=\color{IniRulerColor}}}
    {}
\lstnewenvironment{filelisting}
    {\lstset{language={},rulecolor=\color{FileListingRulerColor}}}
    {}
\lstnewenvironment{commandline}
    {\lstset{language={},framesep=11pt,framerule=1pt,xleftmargin=16pt}}
    {}
\lstnewenvironment{pseudocode}
    {\lstset{language=pseudocode,rulecolor=\color{PseudoCodeRulerColor}}}
    {}

%%
%% some customization
%%
\setlength{\parindent}{0pt}
\setlength{\parskip}{1ex}

%%
%% Shortcuts
%%
\newcommand{\appendixchapter}{\chapter} %% html converter needs to know which chapters are appendices

\newcommand{\tbf}{\textbf} %% bold faced text
\newcommand{\ttt}{\texttt} %% type writer font text

\newcommand{\tab}{\hspace*{5mm}} %% tabulator settings

\newcommand{\new}{$^{New!}$}
\newcommand{\changed}{$^{Changed!}$}

%% Colordefinition for table header rows (requires package colortbl)
\newcommand{\tabheadcol}{\rowcolor[gray]{0.8}}

%%
%% Module parameters list
%%
\newenvironment{params}{\begin{itemize}}{\end{itemize}}
\newcommand{\param}[2]{\item \fpar{#1}: #2}

%%
%% Function/Class/Macro/Variable/Program/Parameter/Define names
%%
%% Write the names in type writer font and do an index entry
%% Allows word wrap by automatic hyphenation
%%
%% Usage: \ffunc{take()}
%%    or: \ffunc[take()]{take(obj)}
%% the second form uses the bracketed word for the index entry
%%

%% NED type names
\newcommand{\nedtype}[2][\DefaultOpt]{\def\DefaultOpt{#2}%
  \index{#1}%
  \texttt{\hyphenchar\font=`\-\relax#2}}

%% MSG type names
\newcommand{\msgtype}[2][\DefaultOpt]{\def\DefaultOpt{#2}%
  \index{#1}%
  \texttt{\hyphenchar\font=`\-\relax#2}}

%% Function names
\newcommand{\ffunc}[2][\DefaultOpt]{\def\DefaultOpt{#2}%
  \index{#1}%
  \texttt{\hyphenchar\font=`\-\relax#2}}

%% Class names
\newcommand{\cppclass}[2][\DefaultOpt]{\def\DefaultOpt{#2}%
  \index{#1}%
  \texttt{\hyphenchar\font=`\-\relax#2}}

%% Macro names
\newcommand{\fmac}[2][\DefaultOpt]{\def\DefaultOpt{#2}%
  \index{#1}%
  \texttt{\hyphenchar\font=`\-\relax#2}}

%% Variable names
\newcommand{\fvar}[2][\DefaultOpt]{\def\DefaultOpt{#2}%
  \index{#1}%
  \texttt{\hyphenchar\font=`\-\relax#2}}

%% Program names
\newcommand{\fprog}[2][\DefaultOpt]{\def\DefaultOpt{#2}%
  \index{#1}%
  \texttt{\hyphenchar\font=`\-\relax#2}}

%% Parameter names
\newcommand{\fpar}[2][\DefaultOpt]{\def\DefaultOpt{#2}%
  \index{#1}%
  \texttt{\hyphenchar\font=`\-\relax#2}}

%% Defines
\newcommand{\fdef}[2][\DefaultOpt]{\def\DefaultOpt{#2}%
  \index{#1}%
  \texttt{\hyphenchar\font=`\-\relax#2}}

%% NED/MSG properties
\newcommand{\fprop}[2][\DefaultOpt]{\def\DefaultOpt{#2}%
  \index{#1}%
  \texttt{\hyphenchar\font=`\-\relax#2}}

%% Keywords (NED, MSG)
\newcommand{\fkeyword}[2][\DefaultOpt]{\def\DefaultOpt{#2}%
  \index{#1}%
  \textbf{\texttt{\hyphenchar\font=`\-\relax#2}}}

%% Configuration options
\newcommand{\fconfig}[2][\DefaultOpt]{\def\DefaultOpt{#2}%
  \index{#1}%
  \textbf{\texttt{\hyphenchar\font=`\-\relax#2}}}

%% File names
\newcommand{\ffilename}[2][\DefaultOpt]{\def\DefaultOpt{#2}%
  \index{#1}%
  \texttt{\hyphenchar\font=`\-\relax#2}}

%% Signals
\newcommand{\fsignal}[2][\DefaultOpt]{\def\DefaultOpt{#2}%
  \index{#1}%
  \texttt{\hyphenchar\font=`\-\relax#2}}

%% do not number subsubsections
%\setcounter{secnumdepth}{4}

% limit the depth of TOC
\setcounter{tocdepth}{2}

%%
%% Start of document
%%
\begin{document}

%% set the image type preference
\DeclareGraphicsExtensions{.pdf,.png}

\pagestyle{empty}
\pagenumbering{roman}

%% %%\begin{figure}[htbp]
%%\begin{center}
%%\includegraphics[width=3.648in, height=0.990in]{figures/usmanFig1}
%%\end{center}
%% %%\end{figure}

%% the following {center} is a trick -- vspace does nothing if there's
%% nothing above it in the page
\begin{center}\end{center}
\vspace{16em}
\hrule
\vspace{2em}
\begin{center}
\tbf{{\huge {\opp}}}\\
\vspace{1.5em}
{\LARGE Discrete Event Simulation System}\\
\vspace{1em}
{\large Version {\oppversion}}\\
\vspace{1em}
\tbf{\textit{{\LARGE User Manual}}}
\end{center}
\vspace{2em}
\hrule

%\vspace{4em}
%by {\large Andr\'{a}s Varga}
%\vspace{3em}

\vspace{8em}

%%\begin{center}
%%\textit{WORK IN PROGRESS}
%%\textit{Last updated: Oct 14, 2008}
%%\end{center}



%%% Local Variables:
%%% mode: latex
%%% TeX-master: "usman"
%%% End:

\cleardoublepage

%%\setcounter{page}{1}
%\newpage
%%\pagenumbering{roman}

%% \shorttableofcontents{Chapters}{0}
%% \cleardoublepage

\tableofcontents
\cleardoublepage

\pagestyle{fancy}
\pagenumbering{arabic}

\chapter{Introduction}
\label{cha:introduction}


\section{What Is {\opp}?}

{\opp} is an object-oriented modular discrete event network simulation
framework. It has a generic architecture, so it can be (and has been)
used in various problem domains:

\begin{itemize}
  \item{modeling of wired and wireless communication networks}
  \item{protocol modeling}
  \item{modeling of queueing networks}
  \item{modeling of multiprocessors and other distributed hardware systems}
  \item{validating of hardware architectures}
  \item{evaluating performance aspects of complex software systems}
  \item{in general, modeling and simulation of
        any system where the discrete event approach is suitable, and
        can be conveniently mapped into entities communicating by exchanging
        messages.}
\end{itemize}

{\opp} itself is not a simulator of anything concrete, but rather
provides infrastructure and tools for \textit{writing} simulations. One of
the fundamental ingredients of this infrastructure is a component
architecture for simulation models. Models are assembled from reusable
components termed \textit{modules}. Well-written modules are truly reusable,
and can be combined in various ways like LEGO blocks.

Modules can be connected with each other via gates (other systems would
call them ports), and combined to form compound modules. The depth of
module nesting is not limited. Modules communicate through message passing,
where messages may carry arbitrary data structures. Modules can pass
messages along predefined paths via gates and connections, or directly to
their destination; the latter is useful for wireless simulations, for
example. Modules may have parameters that can be used to customize module
behavior and/or to parameterize the model's topology.
Modules at the lowest level of the module hierarchy are called
simple modules, and they encapsulate model behavior. Simple modules
are programmed in C++, and make use of the simulation library.

{\opp} simulations can be run under various user interfaces.
Graphical, animating user interfaces are highly useful for
demonstration and debugging purposes, and command-line user
interfaces are best for batch execution.

The simulator as well as user interfaces and tools are highly portable.
They are tested on the most common operating systems (Linux, Mac OS/X,
Windows), and they can be compiled out of the box or after trivial
modifications on most Unix-like operating systems.

{\opp} also supports parallel distributed simulation. {\opp} can
use several mechanisms for communication between partitions of
a parallel distributed simulation, for example MPI or named pipes.
The parallel simulation algorithm can easily be extended, or new
ones can be plugged in. Models do not need any special instrumentation
to be run in parallel -- it is just a matter of configuration.
{\opp} can even be used for classroom presentation of parallel
simulation algorithms, because simulations can be run in parallel
even under the GUI that provides detailed feedback on what is going on.

{\omnest} is the commercially supported version of {\omnetpp}.
{\omnetpp} is free only for academic and non-profit use;
for commercial purposes, one needs to obtain {\omnest} licenses
from Simulcraft Inc.


% \section{Where Does {\opp} Stand in the World of Simulation Tools?}
%
% There are numerous network simulation tools on the market today,
% both commercial and non-commercial. In this section I will try
% to give an overview by picking some of the most important or
% most representative ones in both categories and comparing them
% to {\opp}: PARSEC, SMURPH, NS, Ptolemy, NetSim++, C++SIM, CLASS
% as non-commercial, and OPNET, COMNET III as commercial tools.
% (The {\opp} Home Page contains a list of Web sites with collections
% of references to network simulation tools where the reader can
% get a more complete list.) In the commercial category, OPNET
% is widely held to be the state of the art in network simulation.
% {\opp} is targeted at roughly the same segment of network simulation
% as OPNET.
%
% Seven issues are examined to get an overview about the network
% simulation tools:
%
%
% \textbf{Detail Level}. \textit{Does the simulation tool have the necessary
% power to express details in the model?} In other words, can the
% user implement arbitrary new building blocks like in {\opp}
% or he is confined to the predefined blocks implemented by the
% supplier? Some tools like COMNET III are not programmable by
% the user to this extent therefore they cannot be compared to
% {\opp}. Specialized network simulation tools like NS (for IP)
% and CLASS (for ATM) also rather fall into this category.
%
%
% \textbf{Available Models.} \textit{What protocol models are readily available
% for the simulation tool?} As of end 2004, there are three large
% protocol modelling frameworks available for {\opp}:
% the Mobility Framework for modelling mobile, wireless and ad-hoc networks;
% the INET Framework with TCP, IP, MPLS and other Internet-related protocols;
% and IPv6Suite which provides detailed models for IPv6, Mobile IPv6, 802.11
% and other protocols. Several other simulation models (such as AntNet routing)
% have also been published -- the list is ever growing, and model frameworks
% are constantly maturing and converging.
%
%
% \textbf{Defining Network Topology}. \textit{How does the simulation
% tool support defining the network topology?} Is it possible to
% create some form of hierarchy (nesting) or only ``flat'' topologies
% are supported? Network simulation tools naturally share the property
% that a model (network) consists of ``nodes'' (blocks, entities,
% modules, etc.) connected by ``links'' (channels, connections, etc.).
% Many commercial simulators have graphical editors to define the
% network; however, this is only a good solution if there is an
% alternative form of topology description (e.g. text file) which allows
% one to generate the topology by program. OPNET follows a unique way:
% the network topology is stored in a proprietary binary file format
% which can be generated (and read) by the graphical editor and C
% programs linked against a special library. On the other hand, most
% non-commercial simulation tools do not provide explicit support for
% topology description: one must program a ``driver entity'' which will
% boot the model by creating the necessary nodes and interconnecting
% them (PARSEC, SMURPH, NS). Finally, a large part of the tools that do
% support explicit topology description supports only flat topologies
% (CLASS). {\opp} probably uses the most flexible method: it has a
% human-readable textual topology description format (the NED language)
% which is easy to create with any text-processing tool (\fprog{perl},
% \fprog{awk}, etc.), and the same format is used by the graphical
% editor. It is also possible to create a ``driver entity'' to build a
% network at run-time by program. {\opp} also supports submodule
% nesting.
%
%
% \textbf{Programming Model.} \textit{What is the programming model supported
% by the simulation environment?} Network simulators typically use
% either thread/coroutine-based programming (such as \ffunc{activity()}
% in {\opp}), or FSMs built upon a \ffunc{handleMessage()}-like function.
% For example, OPNET, SMURPH and NetSim++ use FSMs (with underlying
% handleMessage()), PARSEC and C++SIM use threads. {\opp} supports
% both programming models; the author does not know of another
% simulation tool that does so.
%
%
% \textbf{Debugging and Tracing Support.} \textit{What debugging or tracing
% facilities does the simulation tool offer?} Simulation programs
% are infamous for long debugging periods. C++-based simulation
% tools rarely offer much more than \ffunc{printf()}-style debugging; often
% the simulation kernel is also capable of dumping selected debug
% information on the standard output. Animation is also often supported,
% either off-line (record\&playback) or in some client-server architecture,
% where the simulation program is the ``server'' and
% it can be viewed using the ``client''. Off-line animation
% naturally lacks interactivity and is therefore little use in
% debugging. The client-server solution typically has limited power
% because the simulation and the viewer run as independent operating
% system processes, and the viewer has limited access to the simulation
% program's internals and/or it does not have enough control over
% the course of simulation execution. OPNET has a very good support
% for command-line debugging and provides both off-line and client-server
% style animation. NetSim++ and Ptolemy use the client-server method
% of animation. {\opp} goes a different way by linking the GUI
% library with the debugging/tracing capability into the simulation
% executable. This architecture enables the GUI to be very powerful:
% every user-created object is visible (and modifiable) in the
% GUI via inspector windows and the user has tight control over
% the execution. To the author's best knowledge, the tracing feature
% {\opp} provides is unique among the C++-based simulation tools.
%
%
% \textbf{Performance.} \textit{What performance can be expected from the
% simulation?} Simulation programs typically run for several hours.
% Probably the most important factor is the programming language;
% almost all network simulation tools are C/C++-based. Performance
% is a particularly interesting issue with {\opp} since the GUI
% debugging/tracing support involves some extra overhead in the
% simulation library. However, in a reported case, an {\opp} simulation
% was only 1.3 slower than its counterpart implemented in plain
% C (i.e. one containing very little administration overhead),
% which is a very good showing. A similar result was reported in
% a performance comparison with a PARSEC simulation.
%
%
% \textbf{Source Availability.} \textit{Is the simulation library available
% in source?} This is a trivial question but it immediately becomes
% important if one wants to examine or teach the internal workings
% of a simulation kernel, or one runs into trouble because some
% function in the simulation library has a bug and/or it is not
% documented well enough. In general it can be said that non-commercial
% tools (like {\opp}) are open-source and commercial ones are
% not. This is also true for OPNET: the source for simulation kernel
% is not available (although the ready-made protocol models come
% with sources).
%
%
% In conclusion, it can be said that {\opp} has enough features
% to make it a good alternative to most network simulation tools,
% and it has a strong potential to become one of the most widely
% used network simulation packages in academic and research environments.
%

\section{Organization of This Manual}

The manual is organized as follows:

\begin{itemize}
  \item{The Chapters \ref{cha:introduction} and \ref{cha:overview}
    contain introductory material}
  \item{The second group of chapters,
    \ref{cha:the-ned-language},
    \ref{cha:simple-modules} and
    \ref{cha:the-simulation-library}
    are the programming guide. They present the NED language\index{ned!language},
    describe the simulation concepts and their implementation in {\opp}, explain
    how to write simple\index{module!simple} modules, and describe the class library.}
  \item{The chapters \ref{cha:graphics} and \ref{cha:neddoc}
    explain how to customize
    the network graphics and how to write NED source code comments
    from which documentation can be generated.}
  \item{Chapters \ref{cha:building-simulation-programs},
    \ref{cha:config-sim}, \ref{cha:run-sim} and
    \ref{cha:analyzing-simulation-results} deal with practical issues
    like building and running simulations and analyzing results, and
    describe the tools {\opp} provides to support these tasks.}
  \item{Chapter \ref{cha:parallel-execution} is devoted to the support
    of distributed execution.}
  \item{Chapters \ref{cha:plugin-exts} and \ref{cha:embedding}
    explain the architecture and internals of {\opp}, as well as
    ways to extend it and embed it into larger applications.}
  \item{The appendices provide a reference on the NED language,
    configuration options, file formats, and other details.}
\end{itemize}


%
% NOTE: the following stuff could maybe go into an appendix?
%


% \section{History}
%
% \tbf{The early days: 1992-1997}
%
% {\omnetpp} has its distant roots in OMNeT, a simulator written
% in Object Pascal by dr. Gy\"{o}rgy Pongor.
% The development of {\omnetpp} started as a semester's programming
% assignment at the Technical University of Budapest (BME) in 1992.
% The assignment (``creation of an object-oriented discrete event
% simulation system in C++'') was handed out by Prof. Dr Gy\"{o}rgy
% Pongor, and two students signed up: \'{A}kos Kun and Andr\'{a}s Varga.
% The basis for the design was Mr. Pongor's existing simulation
% software written in Pascal, named OMNeT.
%
% We started developing the code in Borland C++ 3.1. The idea
% of multiple runtime environments, a significant addition to the
% original OMNeT design, was there from the very beginning.
% We used Turbo Vision (Borland's then successful character-based
% GUI) for the first `graphical' user interface.
%
% In 1992, we submitted a paper about {\omnetpp} to the
% student's annual university conference
% (named ``TDK'') and won first prize in the ``Software'' section.
% Later we also won 1st prize in the national ``TDK''. Then the
% idea came to port {\omnetpp} to Unix (first for AIX on an RS/6000
% with only 16MB RAM, later Linux), until all development was done
% in Linux and BC3.1 could no longer be supported.
%
% Well, here is a brief list of events since then -- maybe one time
% I'll make up my mind to enhance them to a whole story\dots
%
% 1994: XEnv (a GUI in pure MOTIF, superceded by Tkenv by now)
% was written as diploma work
%
% 1994: used OPNET for several simulation projects. OPNET features
% (and flaws) gave lots of ideas how to continue with {\omnetpp}.
%
% 1995: initial version of nedc was written by a group of exchange
% students from Delft
%
% 1996: initial version of PVM support was programmed by Zoltan
% Vass as diploma work
%
% 1997: started working on Tkenv
%
% 1997 Dec: added GNED
%
% \tbf{Regular open-source releases: 1997-2003}
%
% Until 1997, some people occasionally contributed to {\omnetpp}.
% Since 1997, all development is done entirely by Andras;
% independent of the University since 1998. (He leaves
% the University in 1998, and is no longer affiliated with it
% since then.)
%
% 1997 Sept: web site set up (www.hit.bme.hu/phd/vargaa/omnetpp), first public release
%
% 1997 Feb-1998 Sept: simulation projects for a small company in
% Hungary. We used a version of {\omnetpp}.
%
% 1998 March: added Plove
%
% 1998 June: animation implemented in Tkenv
%
% 1998 Sept-1999 May: work at MeTechnology (later Brokat) in Leipzig
%
% 2000 Jan: MSVC porting
%
% 2000 Sept: contributed model repository set up
%
% 2000: IPSuite created in Karlsruhe
%
% 2001 May: {\omnetpp} 2.1 release
%
% 2001 June: the CVS gets hosted in Karlsruhe
%
% 2002 May: {\omnetpp} 2.2 release
%
% 2003 Jan: Omnest Global Inc. was founded
%
% 2003 Feb-Oct: Andras's stay at CTIE, Monash University, Melbourne, Australia
% with Ahmet Sekercioglu's group; development of {\omnetpp}'s parallel simulation
% framework, doing parallel simulation experiments
%
% 2003 June: first public release of IPv6Suite (CTIE, Monash University)
%
% 2003 July: launch of www.omnetpp.org
%
% 2003 July: release of RSVP/TE models at UTS Sydney
%
% 2003 Aug: Andras takes over IPSuite maintenance
%
% 2003 Sept: Ethernet model made available
%
% 2003 Nov: {\omnetpp} 2.3p1 release
%
% 2004 July: Mobility Framework first official release (TKN, TU Berlin)
%
% 2004 Oct: IPSuite renamed to INET Framework
%
% \dots
%

%
% \section{Credits}
%
% {\omnetpp} was developed by Andr\'{a}s Varga (andras@omnetpp.org,
% andras.varga@omnest.com).
%
% In the early stage of the project, several people contributed
% to {\omnetpp}. Most contributed code is no longer part of
% {\omnetpp}; nevertheless, I'd like to acknowledge the work of the
% following people. First of all, I want thank Dr Gy\"{o}rgy Pongor
% (pongor@hit.bme.hu), my advisor at the Technical University of Budapest
% who initiated {\omnetpp} as a student project.
%
% My fellow student \'{A}kos Kun started to program the first NED parser
% in 1992-93, but it was abandoned after a few months.
% The first version of nedc was finally developed in summer 1995,
% by three exchange students from TU Delft: Jan Heijmans, Alex Paalvast
% and Robert van der Leij. nedc was first called JAR after their initials
% until being renamed nedc. nedc was further developed and refactored
% several times until finally being retired and replaced by nedtool in {\omnetpp} 3.0.
% The second group of Delft exchange students (Maurits Andr\'{e},
% George van Montfort, Gerard van de Weerd) arrived in fall 1995.
% They performed some testing of the simulation library, and
% wrote some example simulations, for example the original version of Token Ring,
% and simulation of the NIM game which survived until {\omnetpp} 3.0.
% These student exchanges were organized by Dr. Leon Rothkranz
% at TU Delft, and Gy\"{o}rgy Pongor at TU Budapest.
%
% The diploma thesis of Zolt\'{a}n Vass (spring 1996) was to prepare
% {\omnetpp} for parallel execution over PVM. This code has been
% replaced with the new Parallel Simulation Architecture in {\omnetpp} 3.0.
% G\'{a}bor Lencse (lencse@hit.bme.hu) was also interested in parallel
% simulation, namely a method called Statistical Synchronization (SSM).
% He implemented the FDDI model (which retired in the 4.0 version), and added
% some extensions into NED for SSM. These extensions have since been removed
% ({\omnetpp} 3.0 does parallel execution on different principles).
%
% The $P^{2}$ algorithm and the original implementation of the k-split algorithm
% was programmed in fall 1996 by Babak Fakhamzadeh from TU Delft.
% k-split was later reimplemented by Andr\'{a}s.
%
% Several bugfixes and valuable suggestions for improvements came
% from the user community of {\omnetpp}. It would be impossible to
% mention everyone here, and the list is constantly growing --
% instead, the README and ChangeLog files contain acknowledgements.
%
% Between summer 2001 and fall 2004, {\omnetpp} CVS was hosted
% at the University of Karlsruhe. Credit for setting
% up and maintaining the CVS server goes to Ulrich Kaage.
% Ulrich can also be credited with converting the user manual from
% Microsoft Word format to LaTeX.
%

%%% Local Variables:
%%% mode: latex
%%% TeX-master: "usman"
%%% End:


\cleardoublepage

\chapter{Using the INET Framework}
\label{cha:usage}

\section{Installation}

To install the INET Framework, download the most recent INET Framework source
release from the download link on the http://inet.omnetpp.org web site,
unpack it, and follow the instructions in the INSTALL file in the root
directory of the archive.

If you plan to simulate ad-hoc networks or you need more wireless link layer
protocols than provided in INET, download and install the INETMANET source
archive instead. (INETMANET aready contains a copy of the INET Framework.)

If you plan to make use of other INET extensions (e.g. HttpTools, VoipTools or TraCI),
follow the installation instructions provided with them. If there are no
install instructions, check if the archive contains a \ttt{.project} file.
If it does, then the project can be imported into the IDE (use File > Import >
General > Existing Project into workspace); make sure that the project is recognized
as an OMNeT++ project (the Project Properties dialog contains an OMNeT++ page)
and it lists the INET or INETMANET project as dependency (check the Project References
page in the Project Properties dialog).

If the extension project contains no \ttt{.project} file, create an empty OMNeT++
project using the New OMNeT++ Project wizard in File > New, add the INET or INETMANET
project as dependency using the Project References page in the Project Properties dialog,
and copy the source files into the project.

After installation, run the example simulations to make sure everything works
correctly. The INSTALL file also describes how to do that.


\section{INET as an OMNeT++-based simulation framework}

TODO what is documented where in the OMNeT++ manual (chapter, section title)

TODO the pattern syntax: *, **; there is also syntax to match index ranges and
ranges of numbers embedded in names


The INET Framework builds upon OMNeT++, and uses the same concept:
\tbf{modules} that communicate by message passing. Hosts, routers, switches
and other network devices are represented by OMNeT++ compound modules.
These compound modules are assembled from simple modules that represent
protocols, applications, and other functional units. A network is again an
OMNeT++ compound module that contains host, router and other modules. The
external interfaces of modules are described in NED files. NED files
describe the parameters and gates (i.e. ports or connectors) of modules,
and also the submodules and connections (i.e. netlist) of compound modules.

The internal structure of compound modules can be customized in several
ways. The first way is the use of \tbf{gate vectors and submodule vectors}.
The sizes of vectors may come from parameters or derived by the number of
external connections to the module. For example, one can have an Ethernet
switch model that has as many ports as needed, i.e. equal to the number of
Ethernet devices connected to it. The second way of customization is
\tbf{parametric types}, that is, the type of a submodule may be specified
as a string parameter. For example, the relay unit inside an Ethernet
switch has several alternative implementations, each one being a distinct
module type. The switch model contains a parameter which allows the user to
select the appropriate relay unit implementation. A third way of
customizing modules is \tbf{inheritance}: a derived module may add new
parameters, gates, submodules or connections, and may set inherited
unassigned parameters to specific values.

Modules are organized into hierarchical \tbf{packages} that directly map to
a folder tree, very much like packages in the Java language. Packages in
INET are organized roughly according to OSI layers; the top packages
include \ttt{inet.applications}, \ttt{inet.transport},
\ttt{inet.networklayer}, and \ttt{inet.linklayer}. Other packages are
\ttt{inet.base}, \ttt{inet.util}, \ttt{inet.world}, \ttt{inet.mobility} and
\ttt{inet.nodes}. These packages correspond to the \ttt{src/applications/},
\ttt{src/transport/}, etc. directories in the INET source tree. (The
\ttt{src/} directory corresponds to the \ttt{inet} package, as defined by
the \ttt{src/package.ned} file.) Subdirectories within the top packages
usually correspond to concrete protocols or protocol families. The
implementations of simple modules are C++ classes with the same name, with
the source files placed in the same directory as the NED file.

The \ttt{inet.nodes} package contains various pre-assembled host, router,
switch, access point, and other modules, for example
\nedtype{StandardHost}, \nedtype{Router} and \nedtype{EtherSwitch} and
\nedtype{WirelessAP}. These compound modules contain some customization
options via parametric submodule types, but they are not meant to be
universal: it is expected that you will create your own node models for
your particular simulation scenarios.

Network interfaces (Ethernet, IEEE 802.11, etc) are usually compound modules
themselves, and are being composed of a queue, a MAC, and possibly other
simple modules. See \nedtype{EthernetInterface} as an example.

Not all modules implement protocols. There are modules which hold data (for
example \nedtype{RoutingTable}), facilitate communication of modules
(\nedtype{NotificationBoard}), perform autoconfiguration of a network
(\nedtype{FlatNetworkConfigurator}), move a mobile node around (for example
\nedtype{ConstSpeedMobility}), and perform housekeeping associated with
radio channels in wireless simulations (\nedtype{ChannelControl}).

Protocol headers and packet formats are described in message definition
files (msg files), which are translated into C++ classes by OMNeT++'s
\textit{opp\_msgc} tool. The generated message classes subclass from OMNeT++'s
\ttt{cPacket} or \ttt{cMessage} classes.


\section{Creating and Running Simulations}

To create a simulation, you would write a NED file that contains the network,
i.e. routers, hosts and other network devices connected together. You can
use a text editor or the IDE's graphical editor to create the network.

Modules in the network contain a lot of unassigned parameters, which need
to be assigned before the simulation can be run.\footnote{The simulator can
interactively ask for parameter values, but this is not very convenient
for repeated runs.} The name of the network to be simulated, parameter values
and other configuration option need to be specified in the \ttt{omnetpp.ini}
file.\footnote{This is the default file name; using other is also possible.}

\ttt{omnetpp.ini} contains parameter assignments as \textit{key=value}
lines, where each key is a wildcard pattern. The simulator matches these
wildcard patterns against full path of the parameter in the module tree
(something like \ttt{ThruputTest.host[2].tcp.nagleEnabled}), and value from
the first match will be assigned for the parameter. If no matching line is
found, the default value in the NED file will be used. (If there is no
default value either, the value will be interactively prompted for, or, in
case of a batch run, an error will be raised.)

OMNeT++ allows several configurations to be put into the \ttt{omnetpp.ini}
file under \ttt{[Config <name>]} section headers, and the right
configuration can be selected via command-line options when the simulation
is run. Configurations can also build on each other: \ttt{extends=<name>}
lines can be used to set up an inheritance tree among them. This feature
allows minimizing clutter in ini files by letting you factor out common
parts. (Another ways of factoring out common parts are ini file inclusion
and specifying multiple ini files to a simulation.) Settings in the
\ttt{[General]} section apply to all configurations, i.e. \ttt{[General]}
is the root of the section inheritance tree.

Parameter studies can be defined by specifying multiple values for a
parameter, with the \ttt{\$\{10,50,100..500 step 100, 1000\}} syntax;
a repeat count can also be specified.

how to run;

C++ -> dll (opp\_run) or exe

\section{Result Collection and Analysis}

how to anaylize results

how to configure result collection


\begin{note}
These are standard OMNeT++ features; please see the manual for more info.
Or go through the TicToc tutorial at least.
\end{note}

\section{Setting up wired network simulations}

todo which modules are needed into it, what they do, etc.

how to add apps, etc


\subsection{Specifying IP (IPv6) addresses in module parameters}

In INET, TCP, UDP and all application layer modules work with
both IPv4 and IPv6. Internally they use the \cppclass{IPvXAddress} C++ class, which
can represent both IPv4 and IPv6 addresses.

Most modules use the \cppclass{IPAddressResolver} C++ class to resolve addresses
specified in module parameters in omnetpp.ini.
\cppclass{IPAddressResolver} accepts the following syntax:

\begin{itemize}
  \item literal IPv4 address: \ttt{"186.54.66.2"}
  \item literal IPv6 address: \ttt{"3011:7cd6:750b:5fd6:aba3:c231:e9f9:6a43"}
  \item module name: \ttt{"server"}, \ttt{"subnet.server[3]"}
  \item interface of a host or router: \ttt{"server/eth0"}, \ttt{"subnet.server[3]/eth0"}
  \item IPv4 or IPv6 address of a host or router: \ttt{"server(ipv4)"},
      \ttt{"subnet.server[3](ipv6)"}
  \item IPv4 or IPv6 address of an interface of a host or router:
      \ttt{"server/eth0(ipv4)"}, \ttt{"subnet.server[3]/eth0(ipv6)"}
\end{itemize}


\section{Setting up wireless network simulations}

todo which modules are needed into it, what they do, etc.

\section{Setting up ad-hoc network simulations}

todo which modules are needed into it, what they do, etc.

\section{Emulation}


\section{Packet traces}

Recording packet traces

Traffic generation using packet traces

\section{Developing New Protocols}

where to put the source files: you can copy and modify the INET framework (fork it)
in the hope that you'll contribute back the changes; or you can develop in
a separate project (create new project in the IDE; mark INET as referenced project)

NED and source files in the same folder; examples under examples/; etc.

for details, read the OMNeT++ manual and the following chapters of this manual

todo...



%%% Local Variables:
%%% mode: latex
%%% TeX-master: "usman"
%%% End:


\cleardoublepage

\chapter{Node Architecture}
\label{cha:node-architecture}


\section{Overview}

This chapter describes the architecture of INET host and router models.

Hosts and routers in the INET Framework are OMNeT++ compound modules that
are composed of the following ingredients:

\begin{itemize}

\item \tbf{Interface Table (\nedtype{InterfaceTable})}. This module
contains the table of network interfaces (eth0, wlan0, etc) in the host.
Interfaces are registered dynamically during the initialization phase by
modules that represent network interface cards (NICs). Other modules access
interface information via a C++ class interface.

\item \tbf{Routing Table (\nedtype{RoutingTable})}. This module contains
the IPv4 routing table. It is also accessed from other via a C++ interface.
The interface contains member functions for adding, removing, enumerating
and looking up routes, and finding the best matching route for a given
destination IP address. The IP module calls uses the latter function for
routing packets, and routing protocols such as OSPF or BGP call the route
manipulation methods to add and manage discovered routes. For IPv6
simulations, \nedtype{RoutingTable} is replaced (or co-exists) with
a \nedtype{RoutingTable6} module; and for Mobile IPv6 simulations
(xMIPv6 project [TODO]), possibly with a \nedtype{BindingCache} module
as well.

\item \tbf{Notification Board (\nedtype{NotificationBoard})}. This module
makes it possible for several modules to communicate in a publish-subscribe
manner. Notifications include change notifications (``routing table
changed'') and state changes (``radio became idle'').

\item \tbf{Mobility module}. In simulations involving node mobility, this
module is responsible for moving around the node in the simulated
``playground.'' A mobility module is also needed for wireless simulations
even if the node is stationery, because the mobility module stores the
node's location, needed to compute wireless transmissions. Different
mobility models (Random Walk, etc.) are supported via different module
types, and many host models define their mobility submodules with
parametric type so that the mobility model can be changed in the
configuration (\ttt{"mobility: <mobilityType> like IMobility"}).

\item \tbf{NICs}. Network interfaces are usually compound modules
themselves, composed of a queue and a MAC module (and in the case of
wireless NICs, a radio module or modules). Examples are
\nedtype{PPPInterface}, \nedtype{EthernetInterface}, and WLAN interfaces
such as \nedtype{Ieee80211NicSTA}. The queue submodule stores packets
waiting for transmission, and it is usually defined as having parametric
type as it has multiple implementations to accomodate different needs
(\nedtype{DropTailQueue}, \nedtype{REDQueue}, \nedtype{DropTailQoSQueue},
etc.) Most MACs also have an internal queue to allow operation without an
external queue module, the advantage being smaller overhead. The NIC's
entry in the host's \nedtype{InterfaceTable} is usually registered by the
MAC module at the beginning of the simulation.

\item \tbf{Network layer}. Modules that represent protocols of the network
layer are usually grouped into a compound module: \nedtype{NetworkLayer}
for IP, and \nedtype{NetworkLayer6} for IPv6. \nedtype{NetworkLayer}
contains the modules \nedtype{IP}, \nedtype{ARP}, \nedtype{ICMP} and
\nedtype{ErrorHandling}. The \nedtype{IP} module performs IP
encapsulation/decapsulation and routing of datagrams; for the latter it
accesses the C++ function call interface of the \nedtype{RoutingTable}.
Packet forwarding can be turned on/off via a module parameter. The
\nedtype{ARP} module is put into the path of packets leaving the network
layer towards the NICs, and performs address resolution for interfaces that
need it (e.g. Ethernet). \nedtype{ICMP} deals with sending and receiving
ICMP packets. The \nedtype{ErrorHandling} module receives and logs ICMP
error replies. The IPv6 network layer, \nedtype{NetworkLayer6} contains the
modules \nedtype{IPv6}, \nedtype{ICMPv6}, \nedtype{IPv6NeighbourDiscovery}
and \nedtype{IPv6ErrorHandling}. For Mobile IPv6 simulations (xMIPv6
project [TODO]), \nedtype{NetworkLayer6} is extended with further modules.

\item \tbf{Transport layer protocols}. Transport protocols are represented
by modules connected to the network layer; currently TCP, UDP and SCTP are
supported. TCP has several implementations: \nedtype{TCP} is the OMNeT++
native implementation; the \nedtype{TCP\_lwip} module wraps the lwIP TCP
stack [TODO]; and the \nedtype{TCP\_NSC} module wraps the Network
Simulation Cradle library [TODO]. For this reason, the \ttt{tcp} submodule
is usually defined with a parametric submodule type (\ttt{"tcp: <tcpType>
like ITCP"}).  UDP and SCTP are implemented by the \nedtype{UDP} and
\nedtype{SCTP} modules, respectively.

\item \sloppypar \tbf{Applications}. Application modules typically connect to TCP
and/or UDP, and model the user behavior as well as the application program
(e.g. browser) and application layer protocol (e.g. HTTP). For convenience,
\nedtype{StandardHost} supports any number of UDP, TCP and SCTP
applications, their types being parametric (\ttt{"tcpApp[numTcpApps]:
<tcpAppType> like TCPApp; udpApp[numUdpApps]: <udpAppType> like
UDPApp; ..."}). This way the user can configure applications entirely from
\ttt{omnetpp.ini}, and does not need to write a new NED file every time
different applications are needed in a host model. Application modules
are typically not present in router models.

\item \tbf{Routing protocols}. Router models typically contain modules that
implement routing protocols such as OSPF or BGP. These modules are
connected to the TCP and/or the UDP module, and manipulate routes in the
\nedtype{RoutingTable} module via C++ function calls.

\item \tbf{MPLS modules}. Additional modules are needed for MPLS
simulations. The \nedtype{MPLS} module is placed between the network layer
and NICs, and implements label switching. \nedtype{MPLS} requires a
\nedtype{LIB} module (Label Information Base) to be present in the router
which it accesses via C++ function calls. MPLS control protocol
implementations (e.g. the \nedtype{RSVP} module) manage the information in
\nedtype{LIB} via C++ calls.

\item \tbf{Relay unit}. Ethernet (and possibly other) switch models may
contain a relay unit, which switches frames among Ethernet (and other
IEEE 802) NICs. Concrete relay unit types include \nedtype{MACRelayUnitPP}
and \nedtype{MACRelayUnitNP}, which differ in their performance models.

\item \tbf{Battery module}. INET extensions uses for wireless sensor
networks (WSNs) may add a battery module to the node model. The battery
module would keep track of energy consumption. A battery module is provided
e.g. by the INETMANET project.

\end{itemize}

The \nedtype{StandardHost} and \nedtype{Router} predefined NED types are
only one possible example of host/router models, and they do not contain
all the above components; for specific simulations it is a perfectly
valid approach to create custom node models.

Most modules are optional, i.e. can be left out of hosts or other node
models when not needed in the given scenario. For example, switch models do
not need a network layer, a routing table or interface table; it may need a
notification board though. Some NICs (e.g. \nedtype{EtherMAC}) can be used
without and interface table and queue models as well.

The notification board (\nedtype{NotificationBoard}) and the interface
table (\nedtype{InterfaceTable}) will be described later in this chapter.
Other modules are covered in later chapters, i.e. \nedtype{RoutingTable}
in the IPv4 chapter.


\section{Addresses}

The INET Framework uses a number of C++ classes to represent various
addresses in the network. These classes support initialization and
assignment from binary and string representation of the address, and
accessing the address in both forms. (Storage is in binary form). They also
support the "unspecified" special value (and the \ffunc{isUnspecified()}
method) that corresponds to the all-zeros address.

\begin{itemize}
  \item \cppclass{MACAddress} represents a 48-bit IEEE 802 MAC address. The
    textual notation it understands and produces is hex string.

  \item \cppclass{IPAddress} represents a 32-bit IPv4 address. It can parse
    and produce textual representations in the "dotted decimal" syntax.

  \item \cppclass{IPv6Address} represents a 128-bit IPv6 address. It can parse
    and produce address strings in the canonical (RFC 3513) syntax.

  \item \cppclass{IPvXAddress} is conceptually a union of a \cppclass{IPAddress}
    and \cppclass{IPv6Address}: an instance stores either an IPv4 address or an
    IPv6 address. \cppclass{IPvXAddress} is mainly used in the transport layer and above
    to abstract away network addresses. It can be assigned from both \cppclass{IPAddress}
    and \cppclass{IPv6Address}, and can also parse string representations of both.
    The \ffunc{isIPv6()}, \ffunc{get4()} and \ffunc{get6()} methods can be used
    to access the value.
\end{itemize}

TODO: Resolving addresses with IPAddressResolver


\section{The Notification Board}

The \nedtype{NotificationBoard} module allows publish-subscribe
communication among modules within a host. Using
\nedtype{NotificationBoard}, modules can notify each other about
events such as routing table changes, interface status changes
(up/down), interface configuration changes, wireless handovers, changes in
the state of the wireless channel, mobile node position changes, etc.
\nedtype{NotificationBoard} acts as a intermediary between the module where
the events occur and modules which are interested in learning about
those events.

\nedtype{NotificationBoard} has exactly one instance within a host or
router model, and it is accessed via direct C++ method calls (not message
exchange). Modules can \textit{subscribe} to categories of changes
(e.g. ``routing table changed'' or ``radio channel became empty''). When
such a change occurs, the corresponding module (e.g. the
\nedtype{RoutingTable} or the physical layer module) will let
\nedtype{NotificationBoard} know, and it will disseminate this information
to all interested modules.

\sloppypar Notification events are grouped into \textit{categories}.
Examples of categories are: \ttt{NF\_HOSTPOSITION\_UPDATED},
\ttt{NF\_RADIOSTATE\_CHANGED}, \ttt{NF\_PP\_TX\_BEGIN},
\ttt{NF\_PP\_TX\_END}, \ttt{NF\_IPv4\_ROUTE\_ADDED},
\ttt{NF\_BEACON\_LOST}, \ttt{NF\_NODE\_FAILURE}, \ttt{NF\_NODE\_RECOVERY},
etc. Each category is identified by an integer (right now it's assigned in
the source code via an enum, in the future we'll convert to dynamic
category registration).

To trigger a notification, the client must obtain a pointer to the
\nedtype{NotificationBoard} of the given host or router (explained later),
and call its \ffunc{fireChangeNotification()} method. The notification will
be delivered to all subscribed clients immediately, inside the
\ffunc{fireChangeNotification()} call.

Clients that wish to receive notifications should implement (subclass from)
the \cppclass{INotifiable} interface, obtain a pointer to the
\nedtype{NotificationBoard}, and subscribe to the categories they are
interested in by calling the \ffunc{subscribe()} method of the
\nedtype{NotificationBoard}. Notifications will be delivered to the
\ffunc{receiveChangeNotification()} method of the client (redefined from
\cppclass{INotifiable}).

In cases when the category itself (an \ttt{int}) does not carry enough
information about the notification event, one can pass additional
information in a data class. There is no restriction on what the data class
may contain, except that it has to be subclassed from \cppclass{cObject},
and of course producers and consumers of notifications should agree on its
contents. If no extra info is needed, one can pass a \ttt{NULL} pointer in
the \ffunc{fireChangeNotification()} method.

A module which implements \cppclass{INotifiable} looks like this:

\begin{cpp}
class Foo : public cSimpleModule, public INotifiable {
    ...
    virtual void receiveChangeNotification(int category, const cObject *details) {..}
    ...
};
\end{cpp}

Note: \cppclass{cObject} was called \cppclass{cPolymorphic} in earlier versions
of OMNeT++. You may occasionally still see the latter name in source code; it
is an alias (typedef) to \cppclass{cObject}.

Obtaining a pointer to the \nedtype{NotificationBoard} module of that host/router:

\begin{cpp}
NotificationBoard *nb; // this is best made a module class member
nb = NotificationBoardAccess().get();  // best done in initialize()
\end{cpp}

TODO how to fire a notification



\section{The Interface Table}

The \nedtype{InterfaceTable} module holds one of the key data structures in
the INET Framework: information about the network interfaces in the host.
The interface table module does not send or receive messages; other modules
access it using standard C++ member function calls.

\subsection{Accessing the Interface Table}

If a module wants to work with the interface table, first it needs to obtain a
pointer to it. This can be done with the help of the
\cppclass{InterfaceTableAccess} utility class:

\begin{cpp}
IInterfaceTable *ift = InterfaceTableAccess().get();
\end{cpp}

\cppclass{InterfaceTableAccess} requires the interface table module to be a
direct child of the host and be called \ttt{"interfaceTable"} in order to
be able to find it. The \ffunc{get()} method never returns \ttt{NULL}: if
it cannot find the interface table module or cannot cast it to the
appropriate C++ type (\cppclass{IInterfaceTable}), it throws an exception
and stop the simulation with an error message.

For completeness, \cppclass{InterfaceTableAccess} also has a
\ffunc{getIfExists()} method which can be used if the code does not require
the presence of the interface table. This method returns \ttt{NULL} if the
interface table cannot be found.

Note that the returned C++ type is \cppclass{IInterfaceTable}; the initial
"\ttt{I}" stands for "interface". \cppclass{IInterfaceTable} is an abstract
class interface that \cppclass{InterfaceTable} implements. Using the abstract
class interface allows one to transparently replace the interface table with
another implementation, without the need for any change or even
recompilation of the INET Framework.

\subsection{Interface Entries}

Interfaces in the interface table are represented with the
\cppclass{InterfaceEntry} class. \cppclass{IInterfaceTable} provides member
functions for adding, removing, enumerating and looking up interfaces.

Interfaces have unique names and interface IDs; either can be used to look up
an interface (IDs are naturally more efficient). Interface IDs are invariant to
the addition and removal of other interfaces.

Data stored by an interface entry include:

\begin{itemize}
  \item \textit{name} and \textit{interface ID} (as described above)
  \item \textit{MTU}: Maximum Transmission Unit, e.g. 1500 on Ethernet
  \item several flags:
    \begin{itemize}
      \item \textit{down}: current state (up or down)
      \item \textit{broadcast}: whether the interface supports broadcast
      \item \textit{multicast} whether the interface supports multicast
      \item \textit{pointToPoint}: whether the interface is point-to-point link
      \item \textit{loopback}: whether the interface is a loopback interface
    \end{itemize}
  \item \textit{datarate} in bit/s
  \item \textit{link-layer address} (for now, only IEEE 802 MAC addresses are supported)
  \item \textit{network-layer gate index}: which gate of the network layer within the host the NIC is connected to
  \item \textit{host gate IDs}: the IDs of the input and output gate of the host the NIC is connected to
\end{itemize}

\tbf{Extensibility}. You have probably noticed that the above list does not
contain data such as the IP or IPv6 address of the interface. Such
information is not part of \cppclass{InterfaceEntry} because we do not want
\nedtype{InterfaceTable} to depend on either the IPv4 or the IPv6 protocol
implementation; we want both to be optional, and we want
\nedtype{InterfaceTable} to be able to support possibly other network
protocols as well.

Thus, extra data items are added to \cppclass{InterfaceEntry} via
extension. Two kinds of extensions are envisioned: extension by the link
layer (i.e. the NIC), and extension by the network layer protocol:

\begin{itemize}

\item \tbf{NICs} can extend interface entries via C++ class inheritance, that is, by
simply subclassing \cppclass{InterfaceEntry} and adding extra data and
functions. This is possible because NICs create and register entries in
\nedtype{InterfaceTable}, so in their code one can just write
\ttt{new MyExtendedInterfaceEntry()} instead of \ttt{new InterfaceEntry()}.

\item \textbf{Network layer protocols} cannot add data via subclassing, so
composition has to be used. \cppclass{InterfaceEntry} contains pointers to
network-layer specific data structures. Currently there are four pointers:
one for IPv4 specific data, one for IPv6 specific data, and two additional
ones that are unassigned. The four data objects can be accessed with the
following \cppclass{InterfaceEntry} member functions: \ffunc{ipv4Data()},
\ffunc{ipv6Data()}, \ffunc{protocol3Data()}, and \ffunc{protocol4Data()}.
They return pointers of the types \cppclass{IPv4InterfaceData},
\cppclass{IPv6InterfaceData} and \cppclass{InterfaceProtocolData} (2x),
respectively. For illustration, \cppclass{IPv4InterfaceData} is installed
onto the interface entries by the \nedtype{RoutingTable} module, and it
contains data such as the IP address of the interface, the netmask, link
metric for routing, and IP multicast addresses associated with the
interface. A protocol data pointer will be \ttt{NULL} if the corresponding
network protocol is not used in the simulation; for example, in IPv4
simulations only \ffunc{ipv4Data()} will return a non-\ttt{NULL} value.


\end{itemize}


\subsection{Interface Registration}

Interfaces are registered dynamically in the initialization phase by modules
that represent network interface cards (NICs). The INET Framework makes use
of the multi-stage initialization feature of OMNeT++, and interface registration takes
place in the first stage (i.e. stage 0).

Example code that performs interface registration:

\begin{cpp}
void PPP::initialize(int stage)
{
    if (stage==0) {
        ...
        interfaceEntry = registerInterface(datarate);
    ...
}

InterfaceEntry *PPP::registerInterface(double datarate)
{
    InterfaceEntry *e = new InterfaceEntry();

    // interface name: NIC module's name without special characters ([])
    e->setName(OPP_Global::stripnonalnum(getParentModule()->getFullName()).c_str());

    // data rate
    e->setDatarate(datarate);

    // generate a link-layer address to be used as interface token for IPv6
    InterfaceToken token(0, simulation.getUniqueNumber(), 64);
    e->setInterfaceToken(token);

    // set MTU from module parameter of similar name
    e->setMtu(par("mtu"));

    // capabilities
    e->setMulticast(true);
    e->setPointToPoint(true);

    // add
    IInterfaceTable *ift = InterfaceTableAccess().get();
    ift->addInterface(e, this);

    return e;
}
\end{cpp}


\subsection{Interface Change Notifications}

\nedtype{InterfaceTable} has a change notification mechanism built in, with
the granularity of interface entries.

Clients that wish to be notified when something changes in
\nedtype{InterfaceTable} can subscribe to the following notification
categories in the host's \nedtype{NotificationBoard}:

\begin{itemize}
  \item \tbf{\ttt{NF\_INTERFACE\_CREATED}}: an interface entry has been
    created and added to the interface table
  \item \tbf{\ttt{NF\_INTERFACE\_DELETED}}: an interface entry is going
    to be removed from the interface table. This is a pre-delete
    notification so that clients have access to interface data that are
    possibly needed to react to the change
% TODO hmmm -- rename the constant? also fire a post-change notification??
  \item \tbf{\ttt{NF\_INTERFACE\_CONFIG\_CHANGED}}: a configuration setting
    in an interface entry has changed (e.g. MTU or IP address)
  \item \tbf{\ttt{NF\_INTERFACE\_STATE\_CHANGED}}: a state variable in an
    interface entry has changed (e.g. the up/down flag)
\end{itemize}

In all those notifications, the data field is a pointer to the
corresponding \cppclass{InterfaceEntry} object. This is even true for
\ttt{NF\_INTERFACE\_DELETED} (which is actually a pre-delete notification).


\section{Initialization Stages}

Node architecture makes it necessary to use multi-stage initialization.
What happens in each stage is this:

In stage 0, interfaces register themselves in \nedtype{InterfaceTable} modules

In stage 1, routing files are read.

In stage 2, network configurators (e.g. \nedtype{FlatNetworkConfiguration})
assign addresses and set up routing tables.

In stage 3, TODO...

In stage 4, TODO...

...

The multi-stage initialization process itself is described in the OMNeT++ Manual.


\section{Communication between protocol layers}

In the INET Framework, when an upper-layer protocol wants to send a data
packet over a lower-layer protocol, the upper-layer module just sends the
message object representing the packet to the lower-layer module, which
will in turn encapsulate it and send it. The reverse process takes place
when a lower layer protocol receives a packet and sends it up after
decapsulation.

It is often necessary to convey extra information with the packet. For
example, when an application-layer module wants to send data over TCP, some
connection identifier needs to be specified for TCP. When TCP sends a
segment over IP, IP will need a destination address and possibly other
parameters like TTL. When IP sends a datagram to an Ethernet interface for
transmission, a destination MAC address must be specified. This extra
information is attached to the message object to as \textit{control info}.

Control info are small value objects, which are attached to packets
(message objects) with its \ttt{setControlInfo()} member function.
Control info only holds auxiliary information for the next protocol layer,
and is not supposed to be sent over the network to other hosts and routers.


\section{Publish-Subscribe Communication within Nodes}

The \nedtype{NotificationBoard} module makes it possible for several modules to
communicate in a publish-subscribe manner. For example, the radio module
(\nedtype{Ieee80211Radio}) fires a \textit{"radio state changed"} notification when
the state of the radio channel changes (from TRANSMIT to IDLE, for example),
and it will be delivered to other modules that have previously subscribed
to that notification category. The notification mechanism uses C++ functions
calls, no message sending is involved.

The notification board submodule within the host (router) must be called
\ttt{notificationBoard} for other modules to find it.


\section{Network interfaces}

todo...

\section{The wireless infrastructure}

todo...



\section{NED Conventions}

\subsection{The @node Property}

By convention, compound modules that implement network devices (hosts,
routers, switches, access points, base stations, etc.) are marked with the
\ttt{@node} NED property. As node models may themselves be hierarchical, the
\ttt{@node} property is used by protocol implementations and other simple
modules to determine which ancestor compound module represents the physical
network node they live in.

\subsection{The @labels Module Property}

The \ttt{@labels} property can be added to modules and gates, and it
allows the OMNeT++ graphical editor to provide better editing experience.
First we look at \ttt{@labels} as a module property.

\ttt{@labels(node)} has been added to all NED module types that may occur on
network level. When editing a network, the graphical editor will NED types
with \ttt{@labels(node)} to the top of the component palette, allowing the
user to find them easier.

Other labels can also be specified in the \ttt{@labels(...)} property. This
has the effect that if one module with a particular label has already been
added to the compound module being edited, other module types with the same
label are also brought to the top of the palette. For example,
\nedtype{EtherSwitch} is annotated with \ttt{@labels(node,ethernet-node)}.
When you drop an \nedtype{EtherSwitch} into a compound module, that will
bring \nedtype{EtherHost} (which is also tagged with the
\ttt{ethernet-node} label) to the top of the palette, making it easier to
find.

\begin{ned}
module EtherSwitch
{
    parameters:
        @node();
        @labels(node,ethernet-node);
        @display("i=device/switch");
    ...
}
\end{ned}

Module types that are already present in the compound module also appear in
the top part of the palette. The reason is that if you already added a
\nedtype{StandardHost}, for example, then you are likely to add more of the
same kind. Gate labels (see next section) also affect palette order: modules
which can be connected to modules already added to the compound module
will also be listed at the top of the palette. The final ordering is the
result of a scoring algorithm.


\subsection{The @labels Gate Property}

Gates can also be labelled with \ttt{@labels()}; the purpose is to make it easier
to connect modules in the editor. If you connect two modules in the editor,
the gate selection menu will list gate pairs that have a label in common.

TODO screenshot

For example, when connecting hosts and routers, the editor will offer connecting
Ethernet gates with Ethernet gates, and PPP gates with PPP gates. This is the
result of gate labelling like this:

\begin{ned}
module StandardHost
{
    ...
    gates:
        inout pppg[] @labels(PPPFrame-conn);
        inout ethg[] @labels(EtherFrame-conn);
    ...
}
\end{ned}

Guidelines for choosing gate label names: For gates of modules that
implement protocols, use the C++ class name of the packet or acompanying
control info (see later) associated with the gate, whichever applies;
append \ttt{/up} or \ttt{/down} to the name of the control info class. For
gates of network nodes, use the class names of packets (frames) that travel
on the corresponding link, with the \ttt{-conn} suffix. The suffix prevents
protocol-level modules to be promoted in the graphical editor palette when
a network is edited.

Examples:

\begin{ned}
simple TCP like ITCP
{
    ...
    gates:
        input appIn[] @labels(TCPCommand/down);
        output appOut[] @labels(TCPCommand/up);
        input ipIn @labels(TCPSegment,IPControlInfo/up);
        output ipOut @labels(TCPSegment,IPControlInfo/down);
        input ipv6In @labels(TCPSegment,IPv6ControlInfo/up);
        output ipv6Out @labels(TCPSegment,IPv6ControlInfo/down);
}
\end{ned}


\begin{ned}
simple PPP
{
    ...
    gates:
        input netwIn;
        output netwOut;
        inout phys @labels(PPPFrame);
}
\end{ned}



%%% Local Variables:
%%% mode: latex
%%% TeX-master: "usman"
%%% End:


\cleardoublepage

\chapter{Point-to-Point Links}
\label{cha:ppp}


\section{Overview}

The INET Framework contains an implementation of the Point-to-Point Protocol
as described in RFC1661 with the following limitations:

\begin{itemize}
\item There are no LCP messages for link configuration,
link termination and link maintenance.
The link can be configured by setting module parameters.
\item PFC and ACFC are not supported, the PPP frame
always contains the 1-byte Address and Control fields
and a 2-byte Protocol field.
\item PPP authentication is not supported
\item Link quality monitoring protocols are not supported.
\item There are no NCP messages, the network layer protocols are
configured by other means.
\end{itemize}

The modules of the PPP model can be found in the \nedtype{inet.linklayer.ppp}
package:

\begin{description}
\item[\nedtype{PPP}] This simple module performs encapsulation
of network datagrams into PPP frames and decapsulation of
the incoming PPP frames. It can be connected to the network
layer directly or can be configured to get the outgoing messages
from an output queue. The module collects statistics about
the transmitted and dropped packages.

\item[\nedtype{PPPInterface}] is a compound module complementing
the \nedtype{PPP} module with an output queue. It implements
the \nedtype{IWiredNic} interface. Input and output hooks can be configured
for further processing of the network messages.
  
\end{description}

\section{PPP frames}

According to RFC1662 the PPP frames contain the following fields:

\begin{bytefield}[bitheight=2\baselineskip]{40}
\bitbox{8}{Flag \\ 01111110} &
\bitbox{8}{Address \\ 11111111} &
\bitbox{8}{Control \\ 00000011} \\
\bitbox{8}{Protocol \\ 8/16 bits} &
\bitbox{10}{Information \\ \textasteriskcentered } &
\bitbox{8}{Padding \\ \textasteriskcentered } \\
\bitbox{8}{FCS \\ 16/32 bits}
\bitbox{8}{Flag \\ 01111110 } &
\bitbox[ltb]{14}{Inter-frame Fill \\ or next Address} 
\end{bytefield}

The corresponding message type in the INET framework is \msgtype{PPPFrame}.
It contains the Information field as an encapsulated \cppclass{cMessage}
object. The Flag, Address and Control fields are omitted
from \msgtype{PPPFrame} because they are constants. The FCS field is
omitted because no CRC computed during the simulation, the bit error
attribute of the \cppclass{cMessage} used instead. The Protocol
field is omitted because the protocol is determined from the class
of the encapsulated message.

The length of the PPP frame is equal to the length of the encapsulated
datagram plus 7 bytes. This computation assumes that
\begin{itemize}
\item there is no inter-octet time fill, so only one Flag sequence
needed per frame
\item padding is not applied
\item PFC and ACFC compression is not applied
\item FCS is 16 bit
\item no escaping was applied
\end{itemize}

\section{PPP module}

The PPP module receives packets from the upper layer in the \fvar{netwIn}
gate, encapsulates them into \msgtype{PPPFrame}s, and send it to the
physical layer through the \fvar{phys} gate. The \msgtype{PPPFrame}s
received from the \fvar{phys} gate are decapsulated and sent to the upper
layer immediately through the \fvar{netwOut} gate.

Incoming datagrams are waiting in a queue if the line is currently busy.
In routers, PPP relies on an external queue module (implementing 
\nedtype{IOutputQueue}) to model finite buffer, implement QoS and/or RED,
and requests packets from this external queue one-by-one. The name
of this queue is given as the \fpar{queueModule} parameter.

In hosts, no such queue is used, so \nedtype{PPP} contains an internal
queue named txQueue to queue up packets wainting for transmission.
Conceptually txQueue is of inifinite size, but for better diagnostics
one can specify a hard limit in the \fpar{txQueueLimit} parameter -- if
this is exceeded, the simulation stops with an error.

The PPP module registers itself in the interface table of the node.
The \fvar{mtu} of the entry can be specified by the
\fpar{mtu} module parameter. The module checks the state of the physical link
and updates the entry in the interface table.

% FIXME:
% When the link is down, no further messages should be sent, and
% when it alive again, it should continue transmitting messages.
% Currently the implementation is buggy: when the link is down,
% datarateChannel set to NULL, but endTransmissionEvent might
% remain in the FES, causing scheduling the next message
% from txQueue (and an ASSERT or *NULL). On the other hand, when the
% messages comes from the external queueModule, the next message
% is dropped, but no further messages are requested even if the
% line gets alive again.


The node containing the PPP module must also contain a
\nedtype{NofiticationBoard} component. Notifications are sent when
transmission of a new PPP frame started (\verb!NF_PP_TX_BEGIN!), finished
(\verb!NF_PP_TX_END!) or when a PPP frame received (\verb!NF_PP_RX_END!).

The PPP component is the source of the following signals:
\begin{itemize}
\item \tbf{txState} state of the link (0=idle,1=busy)
\item \tbf{txPkBytes} number of bytes transmitted
\item \tbf{rxPkBytesOk} number of bytes received successfully
\item \tbf{droppedPkBytesBitError} number of bytes received in erronous frames
\item \tbf{droppedPkBytesIfaceDown} number of bytes dropped because the link is down
\item \tbf{rcvdPkBytesFromHL} number of bytes received from the the upper layer
\item \tbf{passedUpPkBytes} number of bytes sent to the the upper layer
\end{itemize}

% TODO describe animation effects: link broken, busy, idle

% FIXME if the link originally broken the icon of the
% component is grayed (from initialize()), but this effect
% is not cleared, when the line is back (only when a transmission starts).
% Furthermore oldConnColor is not initialized properly in this case.

\section{PPPInterface module}

The \nedtype{PPPInterface} is a compound module implementing the \nedtype{IWiredNic}
interface. It contains a \nedtype{PPP} module and a passive queue for the messages
received from the network layer.

The queue type is specified by the \fpar{queueType} parameter.
It can be set to \nedtype{NoQueue} or to a module type implementing
the \nedtype{IOutputQueue} interface. There are implementations
with QoS and RED support.

In typical use of the \nedtype{PPP} module it is augmented with other nodes
that monitor the traffic or simulate package loss and duplication.
The \nedtype{PPPInterface} module abstract that usage by adding
\nedtype{IHook} components to the network input and output of the
\nedtype{PPP} component. Any number of hook can be added by
specifying the \fpar{numOutputHooks} and \fpar{numInputHooks}
parameters and the types of the \fvar{outputHook} and \fvar{inputHook}
components. The hooks are chained in their numeric order.


%%% Local Variables:
%%% mode: latex
%%% TeX-master: "usman"
%%% End:



\cleardoublepage

\chapter{The Ethernet Model}
\label{cha:ethernet}

% TODO: comment numWirelessPorts in MacRelayUnitPP
% TODO: comment origByteLength in EtherFrame
% FIXME: wrong header length in EtherFrame.msg

\section{Overview}

Variations: 10Mb/s ethernet, fast ethernet, Gigabit Ethernet, Fast Gigabit Ethernet, full duplex

The Ethernet model contains a MAC model (\nedtype{EtherMAC}), LLC model (\nedtype{EtherLLC}) as well
as a bus (\nedtype{EtherBus}, for modelling coaxial cable) and a hub (\nedtype{EtherHub}) model.
A switch model (\nedtype{EtherSwitch}) is also provided.

\begin{itemize}
  \item \nedtype{EtherHost} is a sample node with an Ethernet NIC;
  \item \nedtype{EtherSwitch}, \nedtype{EtherBus}, \nedtype{EtherHub} model switching hub, repeating hub and
        the old coxial cable;
  \item basic compnents of the model: \nedtype{EtherMAC}, \nedtype{EtherLLC}/\nedtype{EtherEncap} module types,
        \nedtype{MACRelayUnit} (\nedtype{MACRelayUnitNP} and \nedtype{MACRelayUnitPP}), \nedtype{EtherFrame} message type,
        \cppclass{MACAddress} class
\end{itemize}

Sample simulations:

\begin{itemize}
  \item the \nedtype{MixedLAN} model contains hosts, switch, hub and bus
  \item the \nedtype{LargeNet} model contains hundreds of computers, switches and hubs
        (numbers depend on model configuration in largenet.ini) and mixes all
        kinds of Ethernet technologies
\end{itemize}

\subsection{Implemented Standards}

The Ethernet model operates according to the following standards:

\begin{itemize}
  \item Ethernet: IEEE 802.3-1998
  \item Fast Ethernet: IEEE 802.3u-1995
  \item Full-Duplex Ethernet with Flow Control: IEEE 802.3x-1997
  \item Gigabit Ethernet: IEEE 802.3z-1998
\end{itemize}

\section{Physical layer}

The nodes of the Ethernet networks are connected by coaxial,
twisted pair or fibre cables. There are several cable types specified
in the standard.

In the INET framework, the cables are represented by connections.
The connections used in Ethernet LANs must be derived from
\nedtype{DatarateConnection} and should have their \fpar{delay} and
\fpar{datarate} parameters set.
The delay parameter can be used to model the distance between the
nodes. The datarate parameter can have four values:

\begin{itemize}
  \item \ttt{10Mbps} classic Ethernet
  \item \ttt{100Mbps} Fast Ethernet
  \item \ttt{1Gbps} Gigabit Ethernet
  \item \ttt{10Gbps} Fast Gigabit Ethernet
\end{itemize}


\subsection{EtherBus}

The \nedtype{EtherBus} component can model a common coaxial cable
found in earlier Ethernet LANs. The nodes are attached at specific
positions of the cable. If a node sends a packet, it is transmitted
in both direction by a given propagation speed.

The gates of the \nedtype{EtherBus} represent taps. The positions
of the taps are given by the \fpar{positions} parameter as a
space separated list of distances in metres. If there are more
gates then positions given, the last distance is repeated.
The bus component send the incoming message in one direction and
a copy of the message to the other direction (except at the ends).
The propagation delays are computed from the distances of the taps
and the \fpar{propagationSpeed} parameter.

Messages are not interpreted by the bus model in any way, thus the bus
model is not specific to Ethernet in any way. Messages may
represent anything, from the beginning of a frame transmission to
end (or abortion) of transmission.

% FIXME #356 NED comment is wrong: data rate must not be zero!
% FIXME #354 default propagation speed is wrong (should be 2e8mps)
%            btw there is a hard coded propagation speed in EtherMACBase.cc

\subsection{EtherHub}

Ethernet hubs are a simple broadcast devices. Messages arriving on a port
are regenerated and broadcast to every other port.

The connections connected to the hub must have the same data rate.
Cable lengths should be reflected in the delays of the connections.

Messages are not interpreted by the \nedtype{EtherType} hub model in any way,
thus the hub model is not specific to Ethernet. Messages may
represent anything, from the beginning of a frame transmission to
end (or abortion) of transmission.

The hub module collects the following statistics:

\begin{itemize}
\item \ttt{pkBytes} handled packets length (vector)
\item \ttt{messages/sec} number of packets per seconds (scalar)
\end{itemize}

% FIXME #356 NED comment is wrong: data rate must not be zero!
% TODO: model delay in hubs: class I device 140 bit time, class II device 92 bit time (for fast ethernet)

\section{MAC layer}

The Ethernet MAC (Media Access Control) layer transmits the Ethernet frames on
the physical media. This is a sublayer within the data link layer. Because
encapsulation/decapsulation is not always needed (e.g. switches does not do
encapsulation/decapsulation), it is implemented in a separate modules
(\nedtype{EtherEncap} and \nedtype{EtherLLC}) that are part of the LLC layer.

\begin{center}
\setlength{\unitlength}{1mm}
\begin{picture}(50,50)
\put(2,47){Network layer}
\put(12,35){\vector(0,1){10}}
\put(12,45){\vector(0,-1){10}}
\put(0,15){\line(1,0){50}}
\put(50,15){\line(0,1){20}}
\put(50,35){\line(-1,0){50}}
\put(0,35){\line(0,-1){20}}
\put(25,25){\line(0,1){10}}
\put(0,25){\line(1,0){50}}
\put(10,28){LLC}
\put(27,28){MAC Relay}
\put(22,18){MAC}
\put(25,5){\vector(0,1){10}}
\put(25,15){\vector(0,-1){10}}
\put(15,2){Physical layer}
\end{picture}
\end{center}

Nowadays almost all Ethernet networks operate using full-duplex
point-to-point connections between hosts and switches. This means
that there are no collisions, and the behaviour of the MAC component
is much simpler than in classic Ethernet that used coaxial cables and
hubs. The INET framework contains two MAC modules for Ethernet:
the \nedtype{EtherMACFullDuplex} is simpler to understand and easier to extend,
because it supports only full-duplex connections. The \nedtype{EtherMAC}
module implements the full MAC functionality including CSMA/CD, it
can operate both half-duplex and full-duplex mode.

\subsection*{Packets and frames}

The environment of the MAC modules is described by the \nedtype{IEtherMAC}
module interface. Each MAC modules has gates to connect to the physical
layer (\ttt{phys\$i} and \ttt{phys\$o}) and to connect to the upper layer
(LLC module is hosts, relay units in switches): \ttt{upperLayerIn} and
\ttt{upperLayerOut}.

When a frame is received from the higher layers, it must be an
\msgtype{EtherFrame}, and with all protocol fields filled out
(including the destination MAC address). The source address, if left empty,
will be filled in with the configured \fpar{address} of the MAC.
% TODO document auto MAC address


Packets received from the network are \msgtype{EtherTraffic} objects.
They are messages representing inter-frame gaps (\msgtype{EtherPadding}),
jam signals (\msgtype{EtherJam}), control frames (\msgtype{EtherPauseFrame})
or data frames (all derived from \msgtype{EtherFrame}). Data frames
are passed up to the higher layers without modification.
In \fpar{promiscuous} mode, the MAC passes up all received frames;
otherwise, only the frames with matching MAC addresses and
the broadcast frames are passed up.

Also, the module properly responds to PAUSE frames, but never sends them
by itself -- however, it transmits PAUSE frames received from upper layers.
See section~\ref{subsec:pause_handling} for more info.

\subsection*{Queuing}

When the transmission line is busy, messages received from the upper layer
needs to be queued.

In routers, MAC relies on an external queue module (see \nedtype{OutputQueue}),
and requests packets from this external queue one-by-one. The name of the
external queue must be given as the \fpar{queueModule} parameer.
There are implementations of \nedtype{OutputQueue} to model finite buffer,
QoS and/or RED.

In hosts, no such queue is used, so MAC contains an internal
queue named \fvar{txQueue} to queue up packets waiting for transmission.
Conceptually, \fvar{txQueue} is of infinite size, but for better diagnostics
one can specify a hard limit in the \fpar{txQueueLimit} parameter -- if this is
exceeded, the simulation stops with an error.

\subsection*{Jamming}

Messages sent by \nedtype{EtherMAC} mark the beginning of a transmission.
The end of a transmission is not explicitly represented by a message,
but instead, the \nedtype{EtherMAC} calculates it from the frame length and
the transmission rate.

When frames collide, the transmission is aborted -- in this case
\nedtype{EtherMAC} makes use of the modelled jam signals to figure out
when colliding transmissions end.

When a transmitting station senses a collision, it transmits a jam signal.
Jam signals are represented by a \msgtype{EtherJam} message.
When \nedtype{EtherMAC} received a jam signal, it knows that one transmission
has ended in jamming -- thus when it receives as many jam messages
as colliding frames, it can be sure all transmissions have been aborted.

Receiving a jam message marks the beginning (and not the end)
of a jam signal, so actually \nedtype{EtherMAC} has to wait for the duration
of the jamming before assuming the channel is free again.

\subsection*{PAUSE handling}
\label{subsec:pause_handling}

The 802.3x standard supports PAUSE frames as a means of flow
control. The frame contains a timer value, expressed as a multiple
of 512 bit-times, that specifies how long the transmitter should
remain quiet. If the receiver becomes uncongested before the
transmitter's pause timer expires, the receiver may elect to send
another PAUSE frame to the transmitter with a timer value of zero,
allowing the transmitter to resume immediately.

\nedtype{EtherMAC} will properly respond to PAUSE frames it receives
(\msgtype{EtherPauseFrame} class),
however it will never send a PAUSE frame by itself. (For one thing,
it doesn't have an input buffer that can overflow.)

\nedtype{EtherMAC}, however, transmits PAUSE frames received by higher layers,
and \nedtype{EtherLLC} can be instructed by a command to send a PAUSE frame to MAC.

% FIXME PAUSE frames should only be sent on full-duplex ethernet.
%       If a switch uses half-duplex mode to connect to hosts, it can ask sending hosts
%       to slow down their sending rates:
%       - force collisions with incoming frames
%       - make it appear as if the channel is busy
% FIXME PAUSE frames should have 0x8808 in the etherType field

\subsection*{Error handling}

If the MAC is not connected to the network ("cable unplugged"), it will
start up in "disabled" mode. A disabled MAC simply discards any messages
it receives. It is currently not supported to dynamically connect/disconnect
a MAC.

CRC checks are modeled by the \fvar{bitError} flag of the packets. Erronous
packets are dropped by the MAC.

\subsection*{Signals and notifications}

Both MAC modules emits the following signals:
\begin{itemize}
  \item \fsignal{txPkBytesSignal} after successful data frame transmission, the number
         of bytes sent (without padding and preamble)
  \item \fsignal{rxPkBytesOkSignal} after successful data frame reception, the number
         of bytes received (without padding and preable)
  \item \fsignal{txPausePkUnitsSignal} after PAUSE frame sent, the pause time
  \item \fsignal{rxPausePkUnitsSignal} after PAUSE frame received, the pause time
  \item \fsignal{rxPkBytesFromHLSignal} when a data frame received from higher layer,
         the number of bytes
  \item \fsignal{dropPkNotForUsSignal} when a data frame received not addressed
         to the MAC, the data frame
  \item \fsignal{dropPkBitErrorSignal} when a frame received with bit error, the frame
  \item \fsignal{dropPkIfaceDownSignal} when a message received and the MAC is not connected,
        the dropped message
  \item \fsignal{packetSentToLowerSignal} before starting to send a packet on phys\$o gate,
        the packet
  \item \fsignal{packetReceivedFromLowerSignal} after a packet received on phys\$i gate,
        the packet (excluding PAUSE and JAM messages and dropped data frames)
  \item \fsignal{packetSentToUpperSignal} before sending a packet on upperLayerOut, the packet
  \item \fsignal{packetReceivedFromUpperSignal} after a packet received on upperLayerIn,
        the packet
\end{itemize}

If the node containing the MAC module has a NotificationBoard component,
notifications are sent about the following events:
\begin{itemize}
  \item \verb!NF_PP_TX_BEGIN! transmission of a frame started
  \item \verb!NF_PP_TX_END! transmission of a frame finished
  \item \verb!NF_PP_RX_END! a frame received
\end{itemize}

The payload of the notifications contain the frame itself.

% FIXME EtherMAC does not send notifications
% FIXME if there is no NotificationBoard in the node -> NULL pointer reference

Apart from statistics can be generated from the signals, the modules collects
the following scalars:

\begin{itemize}
  \item \ttt{simulated time} total simulation time
  \item \ttt{full duplex} boolean value, indicating whether the module operated
          in full-duplex mode
  \item \ttt{frames/sec sent} data frames sent (not including PAUSE frames) per second
  \item \ttt{frames/sec rcvd} data frames received (not including PAUSE frames) per second
  \item \ttt{bits/sec sent}
  \item \ttt{bits/sec rcvd}
\end{itemize}

Note that four of these scalars could be recorded as the count and value of the \fsignal{txPkBytesSignal}
and \fsignal{rxPkBytesSignal} signals resp.

\subsection*{Visual effects}

In the graphical environment, some animation effects help to follow the simulation.
The color of the transmission channel is changed to yellow during transmission, and
turns to red when collision detected. The icon of disconnected MAC modules are grayed out.

The icon of the Ethernet NICs are also colored according to the state of MAC module:
yellow if transmitting, blue if receiving, red if collision detected, white in backoff
and gray in paused state.
 
%\subsection*{Auto-Negotiation}
% Ethernet Auto-Negotiation not supported



\subsection{EtherMACFullDuplex}

From the two MAC implementation \nedtype{EtherMACFullDuplex} is the simpler one,
it operates only in full-duplex mode (its \fpar{duplexEnabled} parameter fixed to
\ttt{true} in its NED definition). This module does not need to implement
CSMA/CD, so there is no collision detection, retransmission with exponential backoff,
carrier extension and frame bursting. Flow control works as described in section
\ref{subsec:pause_handling}.

% FIXME remove frameBursting from NED def, or fix it to false

In the \nedtype{EtherMACFullDuplex} module,
packets arrived at the \ttt{phys\$i} gate are handled when their last bit received.

Outgoing packets are transmitted according to the following state diagram:
 
\begin{center}
\includegraphics{figures/EtherMACFullDuplex_txstates}
\end{center}

The \nedtype{EtherMACFullDuplex} module records two scalars in addition to the
ones mentioned earlier:
\begin{itemize}
\item \ttt{rx channel idle (\%)}: reception channel idle time
        as a percentage of the total simulation time
\item \ttt{rx channel utilization (\%)}: total reception
        time as a percentage of the total simulation time
\end{itemize}

\subsection{EtherMAC}

Ethernet MAC layer implementing CSMA/CD. It supports both half-duplex and full-duplex operations;
in full-duplex mode it behaves as \nedtype{EtherMACFullDuplex}. In half-duplex  mode
it detects collisions, sends jam messages and retransmit frames upon collisions using
the exponential backoff algorithm. In Gigabit Ethernet networks it supports carrier
extension and frame bursting. Carrier extension can be turned off by setting the
\fpar{carrierExtension} parameter to \ttt{false}.

Unlike \nedtype{EtherMACFullDuplex}, this MAC module processes the incoming packets when their
first bit is received. The end of the reception is detected by scheduling a self message.

The operation of the MAC module can be schematized by the following state chart:

\begin{center}
\includegraphics{figures/EtherMAC_txstates}
\end{center}

The module generates these extra signals:
\begin{itemize}
\item \fsignal{collisionSignal} when collision starts (received a frame,
         while transmitting or receiving another one; or start to transmit while receiving a frame),
         the constant value 1
\item \fsignal{backoffSignal} when jamming period ended and before waiting according to the
         exponential backoff algorith, the constant value 1
\end{itemize}

These scalar statistics are generated about the state of the line:
\begin{itemize}
  \item \ttt{rx channel idle (\%)} reception channel idle time (full duplex) or channel
         idle time (half-duplex), as a percentage of the total simulation time
  \item \ttt{rx channel utilization (\%)} total successful reception time (full-duplex) or total
         successful reception/transmission time (half duplex), as a percentage
         of the total simulation time
  \item \ttt{rx channel collision (\%)} total unsuccessful reception time, as a percentage
         of the total simulation time
  \item \ttt{collisions} total number collisions (same as count of \fsignal{collisionSignal})
  \item \ttt{backoffs} total number of backoffs (same as count of \fsignal{backoffSignal})
\end{itemize}

% document error conditions (causing error() calls in the code)

% FIXME handleRestransmission() comment is not true: // no beginSendFrames(), because end of jam signal sending will trigger it automatically
%       in case of inner queue, the queued msg is not transmitted
% FIXME should not enter PAUSE state when !duplexMode


\section{Switches}

Ethernet switches play an important role in modern Ethernet LANs. Unlike
passive hubs and repeaters, that work in the physical layer, the switches
operate in the data link layer and routes data frames between the connected
subnets.

While a hub repeats the data frames on each connected line, possibly causing
collisions, switches help to segment the network to small collision domains.
In modern Gigabit LANs each node is connected to the switch direclty
by full-duplex lines, so no collisions are possible. In this case the
CSMA/CD is not needed and the channel utilization can be high.

\subsection{MAC relay units}

INET framework ethernet switches are built from \nedtype{IMACRelayUnit}
components. Each relay unit has N input and output gates for sending/receiving
Ethernet frames. They should be connected to \nedtype{IEtherMAC} modules.

Internally the relay unit holds a table for the destination address -> output
port mapping. When it receives a data frame it updates the table with the
source address->input port. The table can also be pre-loaded from a text file
while initializing the relay unit. The file name given as the \fpar{addressTableFile}
parameter. Each line of the file contains a hexadecimal MAC address and a decimal port
number separated by tabs. Comment lines beginning with '\#' are also allowed:

\begin{verbatim}
01 ff ff ff ff	0
00-ff-ff-ee-d1	1
0A:AA:BC:DE:FF	2
\end{verbatim}

% FIXME #352 addressTableSize is not checked in readAddressTable -> if overflown
%            then later check updateTableWithAddress has no effect
% FIXME format is wrong in the comment of readAddressTable()

The size of the lookup table is restricted by the \fpar{addressTableSize} parameter.
When the table is full, the oldest address is deleted. Entries are also deleted
if their age exceeds the duration given as the \fpar{agingTime} parameter.

If the destination address is not found in the table, the frame is broadcasted.
The frame is not sent to the same subnet it was received from, because the
target already received the original frame. The only exception if the frame
arrived through a radio channel, in this case the target can be out of range.
The port range 0..\fpar{numWirelessPorts}-1 are reserved for wireless connections.

The \nedtype{IMACRelayUnit} module is not a concrete implementation,
it just defines gates and parameters an \nedtype{IMACRelayUnit} should have.
Concrete inplementations add
capacity and performance aspects to the model (number of frames processed
per second, amount of memory available in the switch, etc.)
C++ implementations can subclass from the class \cppclass{MACRelayUnitBase}.

There are two versions of \nedtype{IMACRelayUnit}:

\begin{description}
  \item[\nedtype{MACRelayUnitNP}] models one or more CPUs with shared memory,
    working from a single shared queue.
  \item[\nedtype{MACRelayUnitPP}] models one CPU assigned to each incoming port,
    working with shared memory but separate queues.
\end{description}

In both models input messages are queued. CPUs poll messages from the queue
and process them in \fpar{processingTime}. If the memory usage exceeds
\fpar{bufferSize}, the frame will be dropped.

A simple scheme for sending PAUSE frames is built in (although
users will probably change it). When the buffer level goes
above a high watermark, PAUSE frames are sent on all ports.
The watermark and the pause time is configurable; use zero
values to disable the PAUSE feature.

% FIXME valid values for pauseTime: 0..0xFFFF
% FIXME ETHER_PAUSE_COMMAND_BYTES should be 4 in Ethernet.h (2bytes opcode + 2bytes pauseTime)
% FIXME PAUSE frame should not be sent on all ports probably
% TODO add lowWatermark, send PauseFrame(pauseUnits=0) to resume sending

The relay units collects the following statistics:

\begin{description}
\item[usedBufferBytes] memory usage as function of time
\item[processedBytes] count and length of processed frames
\item[droppedBytes] count and length of frames dropped caused by out of memory
\end{description}

% FIXME MACRelayUnitNP: no signals are generated, how does @statistic work in the ned file?

\subsection{EtherSwitch}

Model of an Ethernet switch containing a relay unit and multiple MAC units.

The duplexChannel attributes of the MACs must be set according to the
medium connected to the port; if collisions are possible (it's a bus or hub)
it must be set to false, otherwise it can be set to true.
By default it uses half duples MAC with CSMA/CD.

\begin{note}
Switches don't implement the Spanning Tree Protocol. You need to
avoid cycles in the LAN topology.
\end{note}

\section{Link Layer Control}
\label{sec:LLC}
% FIXME #353 there is no module for sending EtherFrameWithSNAP frames
% FIXME ETHER_SNAP_HEADER_LENGTH is wrong in Ethernet.h (+3 bytes, ssap+dsap+control)

\subsection{Frame types}

The raw 802.3 frame format header contains the MAC addresses of the destination and source
of the packet and the length of the data field. The frame footer contains the FCS
(Frame Check Sequence) field which is a 32-bit CRC.

\begin{center}
\begin{bytefield}[bitwidth=1.2em,bitheight=2\baselineskip]{20}
\bitbox{4}{\small MAC \\ destination} &
\bitbox{4}{\small MAC \\ source} &
\bitbox{4}{\small Length} &
\bitbox{4}{\small Payload} &
\bitbox{4}{\small FCS} \\
\bitbox{4}{\small 6 octets} &
\bitbox{4}{\small 6 octets} &
\bitbox{4}{\small 2 octets} &
\bitbox{4}{\small 46-1500 octets} &
\bitbox{4}{\small 4 octets}
\end{bytefield}
\end{center}

Each such frame is preceded by a 7 octet Preamble (with 10101010 octets) and
a 1 octet SFD (Start of Frame Delimiter) field (10101011) and followed by an
12 octet interframe gap. These fields are added and removed in the MAC layer,
so they are omitted here.

When multiple upper layer protocols use the same Ethernet line,
the kernel has to know which which component handles the incoming frames.
For this purpose a protocol identifier was added to the standard Ethernet
frames.

The first solution preceded the 802.3 standard and used a 2 byte protocol
identifier in place of the Length field. This is called Ethernet II
or DIX frame.
Each protocol id is above 1536, so the Ethernet II frames and the 802.3
frames can be distinguished.

\begin{center}
\begin{bytefield}[bitwidth=1.2em,bitheight=2\baselineskip]{20}
\bitbox{4}{\small MAC \\ destination} &
\bitbox{4}{\small MAC \\ source} &
\bitbox{4}{\small EtherType} &
\bitbox{4}{\small Payload} &
\bitbox{4}{\small FCS} \\
\bitbox{4}{\small 6 octets} &
\bitbox{4}{\small 6 octets} &
\bitbox{4}{\small 2 octets} &
\bitbox{4}{\small 46-1500 octets} &
\bitbox{4}{\small 4 octets}
\end{bytefield}
\end{center}

The LLC frame format uses a 1 byte source, a 1 byte destination, and a 1 byte
control information to identify the encapsulated protocol adopted from the
802.2 standard. These fields follow the standard 802.3 header, so the maximum
length of the payload is 1497 bytes:

\begin{center}
\begin{bytefield}[bitwidth=1.2em,bitheight=2\baselineskip]{20}
\bitbox{4}{\small MAC \\ destination} &
\bitbox{4}{\small MAC \\ source} &
\bitbox{4}{\small Length} &
\bitbox{4}{\small DSAP} &
\bitbox{4}{\small SSAP} &
\bitbox{4}{\small Control} &
\bitbox{4}{\small Payload} &
\bitbox{4}{\small FCS} \\
\bitbox{4}{\small 6 octets} &
\bitbox{4}{\small 6 octets} &
\bitbox{4}{\small 2 octets} &
\bitbox{4}{\small 1 octets} &
\bitbox{4}{\small 1 octets} &
\bitbox{4}{\small 1 octets} &
\bitbox{4}{\small 43-1497 octets} &
\bitbox{4}{\small 4 octets}
\end{bytefield}
\end{center}

The SNAP header uses the EtherType protocol identifiers inside an LLC header.
The SSAP and DSAP fields are filled with 0xAA (SAP\_SNAP), and the control
field is 0x03. They are followed by a 3 byte orgnaization and a 2 byte local
code the identify the protocol. If the organization code is 0, the local field
contains an EtherType protocol identifier.

\begin{center}
\begin{bytefield}[bitwidth=1.2em,bitheight=2\baselineskip]{20}
\bitbox{4}{\small MAC \\ destination} &
\bitbox{4}{\small MAC \\ source} &
\bitbox{4}{\small Length} &
\bitbox{4}{\small DSAP \\ 0xAA} &
\bitbox{4}{\small SSAP \\ 0xAA} &
\bitbox{4}{\small Control \\ 0x03} &
\bitbox{4}{\small OrgCode} &
\bitbox{4}{\small Local \\ Code} &
\bitbox{4}{\small Payload} &
\bitbox{4}{\small FCS} \\
\bitbox{4}{\small 6 octets} &
\bitbox{4}{\small 6 octets} &
\bitbox{4}{\small 2 octets} &
\bitbox{4}{\small 1 octets} &
\bitbox{4}{\small 1 octets} &
\bitbox{4}{\small 1 octets} &
\bitbox{4}{\small 3 octets} &
\bitbox{4}{\small 2 octets} &
\bitbox{4}{\small 38-1492 octets} &
\bitbox{4}{\small 4 octets}
\end{bytefield}
\end{center}

The INET defines these frames in the \ffilename{EtherFrame.msg} file.
The models supports Ethernet II, 803.2 with LLC header, and 803.3 with LLC and SNAP headers.
The corresponding classes are:
\msgtype{EthernetIIFrame}, \msgtype{EtherFrameWithLLC} and \msgtype{EtherFrameWithSNAP}. They all class
from \msgtype{EtherFrame} which only represents the basic MAC frame with source and
destination addresses. \nedtype{EtherMAC} only deals with \msgtype{EtherFrame}s, and does not
care about the specific subclass.

Ethernet frames carry data packets as encapsulated cMessage objects.
Data packets can be of any message type (cMessage or cMessage subclass).

The model encapsulates data packets in Ethernet frames using the \ttt{encapsulate()}
method of cMessage. Encapsulate() updates the length of the Ethernet frame too,
so the model doesn't have to take care of that.

The fields of the Ethernet header are passed in a \cppclass{Ieee802Ctrl}
control structure to the LLC by the network layer.


EtherJam, EtherPadding (interframe gap), EtherPauseFrame?


\subsection{EtherEncap}

The \nedtype{EtherEncap} module generates \msgtype{EthernetIIFrame} messages.

EtherFrameII

\subsection{EtherLLC}

EtherFrameWithLLC

SAP registration

% TODO delete EtherLLC, because LLC without SNAP is not used with IP (no ARP,IPv6 SAP)
% TODO modify EtherEncap to handle EtherFrameWithSNAP frames too (we can not send EtherFrameWithSNAP now)

\subsubsection{\nedtype{EtherLLC} and higher layers}

The \nedtype{EtherLLC} module can serve several applications (higher layer protocols),
and dispatch data to them. Higher layers are identified by DSAP.
See section "Application registration" for more info.

\nedtype{EtherEncap} doesn't have the functionality to dispatch to different
higher layers because in practice it'll always be used with IP.

\subsubsection{Communication between LLC and Higher Layers}

Higher layers (applications or protocols) talk to the \nedtype{EtherLLC} module.

When a higher layer wants to send a packet via Ethernet, it just
passes the data packet (a cMessage or any subclass) to \nedtype{EtherLLC}.
The message kind has to be set to IEEE802CTRL\_DATA.

In general, if \nedtype{EtherLLC} receives a packet from the higher layers,
it interprets the message kind as a command. The commands include
IEEE802CTRL\_DATA (send a frame), IEEE802CTRL\_REGISTER\_DSAP (register highher layer)
IEEE802CTRL\_DEREGISTER\_DSAP (deregister higher layer) and IEEE802CTRL\_SENDPAUSE
(send PAUSE frame) -- see EtherLLC for a more complete list.

The arguments to the command are NOT inside the data packet but
in a "control info" data structure of class \cppclass{Ieee802Ctrl}, attached to
the packet. See controlInfo() method of cMessage (OMNeT++ 3.0).

For example, to send a packet to a given MAC address and protocol
identifier, the application sets the data packet's message kind
to ETH\_DATA ("please send this data packet" command),
fills in the \nedtype{Ieee802Ctrl} structure with the destination MAC address and
the protocol identifier, adds the control info to the message, then sends
the packet to \nedtype{EtherLLC}.

When the command doesn't involve a data packet (e.g.
IEEE802CTRL\_(DE)REGISTER\_DSAP, IEEE802CTRL\_SENDPAUSE), a dummy packet
(empty cMessage) is used.

\subsubsection{Rationale}

The alternative of the above communications would be:

\begin{itemize}
  \item adding the parameters such as destination address into the data
    packet. This would be a poor solution since it would make the
    higher layers specific to the Ethernet model.
  \item encapsulating a data packet into an \textit{interface packet} which
    contains the destination address and other parameters. The
    disadvantages of this approach is the overhead associated with
    creating and destroying the interface packets.
\end{itemize}

Using a control structure is more efficient than the interface packet
approach, because the control structure can be created once inside
the higher layer and be reused for every packet.

It may also appear to be more intuitive in Tkenv because one can observe
data packets travelling between the higher layer and Ethernet
modules -- as opposed to "interface" packets.


\subsubsection{EtherLLC: SAP Registration}

The Ethernet model supports multiple applications or higher layer
protocols.

So that data arriving from the network can be dispatched to the
correct applications (higher layer protocols), applications
have to register themselves in \nedtype{EtherLLC}. The registration
is done with the IEEE802CTRL\_REGISTER\_DSAP command
(see section "Communication between LLC and higher layers")
which associates a SAP with the LLC port. Different applications
have to connect to different ports of \nedtype{EtherLLC}.

The ETHERCTRL\_REGISTER\_DSAP/IEEE802CTRL\_DEREGISTER\_DSAP commands use only the
dsap field in the \cppclass{Ieee802Ctrl} structure.

\subsection{EthernetInterface module}

The \nedtype{EthernetInterface} compound module implements the \nedtype{IWiredNic}
interface. Complements \nedtype{EtherMAC} and \nedtype{EtherEncap} with an output queue
for QoS and RED support. It also has configurable input/output filters as \nedtype{IHook}
components similarly to the \nedtype{PPPInterface} module.

% TODO there is no IWiredNic with EtherLLC

\section{Ethernet applications}

The \nedtype{inet.applications.ethernet} package contains modules
for a simple client-server application. The \nedtype{EtherAppCli} is a simple
traffic generator that peridically sends \msgtype{EtherAppReq} messages
whose length can be configured. destAddress, startTime,waitType, reqLength, respLength

The server component of the model (\nedtype{EtherAppSrv}) responds with a
\msgtype{EtherAppResp} message of the requested length. If the response does
not fit into one ethernet frame, the client receives the data in multiple
chunks.

% FIXME reqLength>1500 causes an error in the LLC module
% FIXME numFrames field of EtherAppRes is not used
% FIXME server always sends 1497 byte chunks, it should depend on the framing (1497 is for LLC)
% FIXME if registerSAP is false (default), the and EtherLLC used, then the client won't receive messages (auto config?)
% FIXME Ieee802Nic -> EthernetInterface in the NED comment

Both applications have a \fpar{registerSAP} boolean parameter.
This parameter should be set to \ttt{true} if the application is connected
to the \nedtype{EtherLLC} module which requires registration of the SAP
before sending frames.

Both applications collects the following statistics: sentPkBytes, rcvdPkBytes,
endToEndDelay.

The client and server application works with any model that accepts
Ieee802Ctrl control info on the packets (e.g. the 802.11 model).
The applications should be connected directly to the \nedtype{EtherLLC}
or an EthernetInterface NIC module.

The model also contains a host component that groups the applications
and the LLC and MAC components together (\nedtype{EtherHost}). This node does
not contain higher layer protocols, it generates Ethernet traffic directly.
By default it is configured to use half duplex MAC (CSMA/CD).

\section{Ethernet networks}

\subsection{\nedtype{LargeNet} model}

The \nedtype{LargeNet} model demonstrates how one can put together models of large
LANs with little effort, making use of MAC auto-configuration.

\nedtype{LargeNet} models a large Ethernet campus backbone. As configured in the
default omnetpp.ini, it contains altogether about 8000 computers
and 900 switches and hubs. This results in about 165MB process size
on my (32-bit) linux box when I run the simulation.
The model mixes all kinds of Ethernet technology: Gigabit Ethernet,
100Mb full duplex, 100Mb half duplex, 10Mb UTP, 10Mb bus ("thin Ethernet"),
switched hubs, repeating hubs.

The topology is in \nedtype{LargeNet}.ned, and it looks like this: there's chain
of n=15 large "backbone" switches (switchBB[]) as well as four more
large switches (switchA, switchB, switchC, switchD) connected to
somewhere the middle of the backbone (switchBB[4]). These 15+4 switches
make up the backbone; the n=15 number is configurable in omnetpp.ini.

Then there're several smaller LANs hanging off each backbone switch.
There're three types of LANs: small, medium and large (represented by
compound module types \nedtype{SmallLAN}, \nedtype{MediumLAN}, \nedtype{LargeLAN}). A small LAN
consists of a few computers on a hub (100Mb half duplex); a medium
LAN consists of a smaller switch with a hub on one of its port
(and computers on both); the large one also has a switch and a hub,
plus an Ethernet bus hanging of one port of the hub (there's still hubs
around with one BNC connector besides the UTP ones).
By default there're 5..15 LANs of each type hanging off each backbone
switch. (These numbers are also omnetpp.ini parameters like the length
of the backbone.)

The application model which generates load on the simulated LAN is
simple yet powerful. It can be used as a rough model for any
request-response based protocol such as SMB/CIFS (the Windows file
sharing protocol), HTTP, or a database client-server protocol.

Every computer runs a client application (\nedtype{EtherAppCli}) which connects
to one of the servers. There's one server attached to switches A, B,
C and D each: serverA, serverB, serverC and serverD -- server selection
is configured in omnetpp.ini). The servers run \nedtype{EtherAppSrv}.
Clients periodically send a request to the server, and the request
packet contains how many bytes the client wants the server to send back
(this can mean one or more Ethernet frames, depending on the byte count).
 Currently the request and reply lengths are configured in omnetpp.ini
as intuniform(50,1400) and truncnormal(5000,5000).

The volume of the traffic can most easily be controlled with the
time period between sending requests; this is currently
set in omnetpp.ini to exponential(0.50) (that is, average 2
requests per second). This already causes frames to be dropped
in some of the backbone switches, so the network is a bit
overloaded with the current settings.

The model generates extensive statistics. All MACs (and most other
modules too) write statistics into omnetpp.sca at the end
of the simulation: number of frames sent, received, dropped, etc.
These are only basic statistics, however it still makes the
scalar file to be several ten megabytes in size. You can use
the analysis tools provided with OMNeT++ to visualized the data
in this file. (If the file size is too big, writing statistics
can be disabled, by putting **.record-scalar=false in the ini file.)
The model can also record output vectors, but this is currently
disabled in omnetpp.ini because the generated file can easily reach
gigabyte sizes.

%%% Local Variables:
%%% mode: latex
%%% TeX-master: "usman"
%%% End:

\cleardoublepage

\chapter{The Radio Infrastructure}
\label{cha:radio}


\section{Overview}

Blah blah blah


%%% Local Variables:
%%% mode: latex
%%% TeX-master: "usman"
%%% End:


\cleardoublepage

\chapter{The 802.11 Model}
\label{cha:80211}


\section{Overview}

This chapter provides an overview of the IEEE 802.11 model for the INET Framework.

An IEEE 802.11 interface (NIC) comes in several flavours, differring
in their role (ad-hoc station, infrastructure mode station, or
access point) and their level of detail:

\begin{enumerate}
 \item \nedtype{Ieee80211Nic}: a generic (configurable) NIC
 \item \nedtype{Ieee80211NicAdhoc}: for ad-hoc mode
 \item \nedtype{Ieee80211NicAP}, \nedtype{Ieee80211NicAPSimplified}: for use in an access point
 \item \nedtype{Ieee80211NicSTA}, \nedtype{Ieee80211NicSTASimplified}: for use in an
   infrastructure-mode station
\end{enumerate}

NICs consist of four layers, which are the following (in top-down order):

\begin{enumerate}
  \item agent
  \item management
  \item MAC
  \item physical layer (radio)
\end{enumerate}

\textit{The physical layer} modules (\nedtype{Ieee80211Radio}; with some limitations,
\nedtype{SnrEval80211}, \nedtype{Decider80211} can also be used) deal with modelling
transmission and reception of frames. They model the characteristics of
the radio channel, and determine if a frame was received correctly
(that is, it did not suffer bit errors due to low signal power or
interference in the radio channel). Frames received correctly are passed
up to the MAC. The implementation of these modules is based on the
Mobility Framework.

\textit{The MAC layer} (\nedtype{Ieee80211Mac}) performs transmission of frames according
to the CSMA/CA protocol. It receives data and management frames from
the upper layers, and transmits them.

\textit{The management layer} performs encapsulation and decapsulation of data packets
for the MAC, and exchanges management frames via the MAC with its peer
management entities in other STAs and APs. Beacon, Probe Request/Response,
Authentication, Association Request/Response etc frames are generated
and interpreted by management entities, and transmitted/received via
the MAC layer. During scanning, it is the management entity that periodically
switches channels, and collects information from received beacons and
probe responses.

The management layer has several implementations which differ in their role
(STA/AP/ad-hoc) and level of detail: \nedtype{Ieee80211MgmtAdhoc},
\nedtype{Ieee80211MgmtAP}, \nedtype{Ieee80211MgmtAPSimplified}, \nedtype{Ieee80211MgmtSTA},
\nedtype{Ieee80211MgmtSTASimplified}. The ..Simplified ones differ from the others
in that they do not model the scan-authenticate-associate process,
so they cannot be used in experiments involving handover.

\textit{The agent} is what instructs the management layer to perform
scanning, authentication and association. The management layer itself
just carries out these commands by performing the scanning, authentication
and association procedures, and reports back the results to the agent.

The agent layer is currenly only present in the \nedtype{Ieee80211NicSTA} NIC module,
as an \nedtype{Ieee80211AgentSTA} module. The managament entities in other NIC
variants do not have as much freedom as to need an agent to control them.

By modifying or replacing the agent, one can alter the dynamic behaviour
of STAs in the network, for example implement different handover strategies.

\subsection{Limitations}

See the documentation of \nedtype{Ieee80211Mac} for features unsupported by this
model.

TODO further details about the implementation: what is modelled and what is
 not (beacons, auth, ...), communication between modules, frame formats,
 ...




%%% Local Variables:
%%% mode: latex
%%% TeX-master: "usman"
%%% End:


\cleardoublepage

% last updated to inet commit 'e10ce09097652b72f41fcbde5dc81a2cc2537d32'

\chapter{Node Mobility}
\label{cha:mobility}


\section{Overview}

In order to accurately evaluate a protocol for an ad-hoc network,
it is important to use a realistic model for the motion of mobile
hosts. Signal strengths, radio interference and channel occupancy
depends on the distances between nodes. The choice of the mobility
model can significantly influence the results of a simulation
(e.g. data packet delivery ratio, end-to-end delay, average hop count)
as shown in \cite{Camp02asurvey}.

There are two methods for incorporating mobility into simulations:
using traces and synthetic models. Traces contains recorded motion
of the mobile hosts, as observed in real life system. Synthetic models
use mathematical models for describing the behaviour of the mobile hosts.

There are mobility models that represent mobile nodes whose movements 
are independent of each other (entity models) and mobility models
that represent mobile nodes whose movements are dependent on each other
(group models). Some of the most frequently used entity models are the
Random Walk Mobility Model, Random Waypoint Mobility Model, Random
Direction Mobility Model, Gauss-Markov Mobility Model, City Section
Mobility Model. The group models include the Column Mobility Model,
Nomadic Community Mobility Model, Pursue Mobility Model,
Reference Point Group Mobility Model.

The INET framework has components for the following trace files:

\begin{itemize}
\item \tbf{Bonn Motion} native file format of the 
    \href{http://net.cs.uni-bonn.de/wg/cs/applications/bonnmotion/}{BonnMotion}
    scenario generation tool.
\item \tbf{Ns2} trace file generated by the CMU's scenario generator that used in Ns2.
\item \tbf{ANSim} XML trace file of the ANSim (Ad-Hoc Network Simulation) tool.
\end{itemize}

It is easy to integrate new entity mobility models into the INET framework,
but group mobility is not supported yet. Therefore all the models
shipped with INET are implementations of entitiy models:

\begin{itemize}
\item \tbf{Deterministic Motions} for fixed position nodes and nodes
      moving on a linear, circular, rectangular paths.
\item \tbf{Random Waypoint} model includes pause times between changes
      in destination and speed.
\item \tbf{Gauss-Markov} model uses one tuning parameter to vary the degree
      of randomness in mobility pattern.
\item \tbf{Mass Mobility} models a mass point with inertia and momentum.
\item \tbf{Chiang Mobility} uses a probabilistic transition matrix to change
      the state of motion of the node.
\end{itemize}

\section{Mobility in INET}

In INET mobile nodes have to contain a module implementing the
\nedtype{IMobility} marker interface. This module stores the current
coordinates of the node and is responsible for updating the position
periodically and emitting the \ttt{mobilityStateChanged} signal
when the position changed.

The p[0] and p[1] fields of the display string of the node is
also updated, so if the simulation run is animated, the node is
actually moving on the screen. The current position of the node
can be obtained from the display string.

The radio simulations has a \nedtype{ChannelAccess} module that takes case of
establishing communication channels between nodes that are within
communication distance and tearing down the connections when they
move out of range. The \nedtype{ChannelAccess} module uses to
\ttt{mobilityStateChanged} signal to determine when the connection
status needs to be updated.

There are two possibilities to implement a new mobility model. The simpler but
limited one is to use \nedtype{TurtleMobility} as the mobility component and to
write a script similar to the turtle graphics of LOGO. The second is to
implement a simple module in C++. In this case the C++ class of the mobility
module should be derived from \cppclass{IMobility} and its NED type should
implement the \nedtype{IMobility} interface.

%\begin{figure}
%\begin{center}
%\includegraphics{figures/mobility_classes}
%\end{center}
%\end{figure}

% FIXME The Z coordinate is often initialized randomly.
%       It would be better (backward compatibility)
%       to initialize them to 0. (contsraintAreaDepth=0 not handled everywhere)

\subsection{MobilityBase class}

The abstract \cppclass{MobilityBase} class is the base of the mobility
modules defined in the INET framework. This class allows to define
a cubic volume that the node can not leave. The volume is configured
by setting the \fpar{constraintAreaX}, \fpar{constraintAreaY},
\fpar{constraintAreaZ},
\fpar{constraintAreaWidth}, \fpar{constraintAreaHeight} and
\fpar{constraintAreaDepth} parameters.

When the module is initialized it sets the initial position of the node
by calling the \ffunc{initializePosition()} method.
The default implementation of this method sets the position from the
display string if the \fpar{initFromDisplayString} parameter is true.
Otherwise the position can be given as the \fpar{initialX},
\fpar{initialY} and \fpar{initialZ} parameters.
If neither of these parameters are given,
a random initial position is choosen within the contraint area.

The module is responsible for periodically updating the position.
For this purpose it should send timer messages to itself. These messages
are processed in the \ffunc{handleSelfMessage} method. In derived
classes, \ffunc{handleSelfMessage} should compute the new position
and update the display string and publish the new position by calling
the \ffunc{positionUpdated} method.

When the node reaches the boundary of the constraint area, the mobility
component has to prevent the node to exit. It can call the
\ffunc{handleIfOutside} method, that offers the following policies:

\begin{itemize}
  \item reflect of the wall
  \item reappear at the opposite edge (torus area)
  \item placed at a randomly chosen position of the area
  \item stop the simulation with an error
\end{itemize}

\subsection{MovingMobilityBase}

The abstract \cppclass{MovingMobilityBase} class can be used to model
mobilities when the node moves on a continous trajectory and
updates its position periodically. Subclasses only need to implement
the \ffunc{move} method that is responsible to update the current
position and speed of the node.

The abstract \ffunc{move} method is called autmotically in every
\fpar{updateInterval} steps. The method is also called when a client
requested the current position or speed or when the \ffunc{move} method
requested an update at a future moment by setting the \fvar{nextChange}
field. This can be used when the state of the motion changes at a
specific time that is not a multiple of \fpar{updateInterval}.
The method can set the \fpar{stationary} field to \ttt{true} to
indicate that the node reached its final position and no more position
update is needed.

% TODO draw a plot of t-position function marking the points when 
%      move() is called, stationary set to true, etc.

\subsection{LineSegmentsMobilityBase}

The path of a mobile node often consist of linear movements of constant
speed. The node moves with some speed for some time, then with another
speed for another duration and so on. If a mobility model fits this
description, it might be suitable to derive the implementing C++ class
from \cppclass{LineSegmentsMobilityBase}.

The module first choose a target position and a target time by calling
the \ffunc{setTargetPosition} method. If the target position differs
from the current position, it starts to move toward the target and
updates the position in the configured \fpar{updateInterval} intervals.
When the target position reached, it chooses a new target.

% TODO draw a plot like above, but containing linear segments, mark
%      the points when setTargetPosition called.

% FIXME LineSegmentsMobilityBase should not schedule the self message at updateInterval
%       when lastPosition==targetPosition (the node is waiting at the current position,
%       e.g. every second step in RandomWPMobility)
% TODO Consider an updateInterval computed from an updateDistance and speed, because position change
%      may be irrevelant during a preconfigured updateInterval.

\section{Implemented models}

\subsection{Deterministic movements}

\begin{description}

\item[StationaryMobility] This mobility module does nothing;
it can be used for stationary nodes.

\item[StaticGridMobility] Places all nodes in a rectangular grid.

% TODO it always creates an N x N grid, generalize

\item[LinearMobility] This is a linear mobility model with speed,
angle and acceleration parameters. Angle only changes when the mobile
node hits a wall: then it reflects off the wall at the same angle.

z coordinate is constant
movement is always parallel with X-Y plane

% TODO interpret 'angle' as asimuth and introduce inclination angle (default is 0)
%      to describe movement along arbitrary line segment
% FIXME why different speed and lastSpeed

\item[CircleMobility] Moves the node around a circle parallel to the X-Y plane
with constant speed.
The node bounces from the bounds of the constraint area.
The circle is given by the \fpar{cx}, \fpar{cy} and \fpar{r} parameters,
The initial position determined by the \fpar{startAngle} parameter.
The position of the node is refreshed in \fpar{updateInterval} steps.

\item[RectangleMobility] Moves the node around the constraint area.
configuration: speed, startPos, updateInterval
% should be derived from LineSegmentsMobilityBase?

\item[TractorMobility] Moves a tractor through a field with a certain
amount of rows. The following figure illustrates the movement of the
tractor when the \fpar{rowCount} parameter is 2. The trajectory follows
the segments in $1,2,3,4,5,6,7,8,1,2,3...$ order. The area is configured
by the \fpar{x1}, \fpar{y1}, \fpar{x2}, \fpar{y2} parameters.

% TODO use constraint area instead of new x1,y1,x2,y2 parameters as in RectangleMobility

\begin{center}
\setlength{\unitlength}{0.5mm}
\begin{picture}(80,80)
\put(40,72){$1$} \put(10,70){\vector(1,0){30}} \put(10,70){\line(1,0){60}}
\put(72,55){$2$} \put(70,70){\vector(0,-1){15}} \put(70,70){\line(0,-1){30}}
\put(40,42){$3$} \put(70,40){\vector(-1,0){30}} \put(70,40){\line(-1,0){60}}
\put(5,25){$4$} \put(10,40){\vector(0,-1){15}} \put(10,40){\line(0,-1){30}}
\put(40,12){$5$} \put(10,10){\vector(1,0){30}} \put(10,10){\line(1,0){60}}
\put(72,25){$6$} \put(70,10){\vector(0,1){15}} \put(70,10){\line(0,1){30}}
\put(40, 33){$7$}
\put(5,55){$8$} \put(10,40){\vector(0,1){15}} \put(10,40){\line(0,1){30}}
\put(0,72){$(x_1,y_1)$} \put(65,2){$(x_2,y_2)$}
\end{picture}
\end{center}

\end{description}

\subsection{Random movements}

\begin{description}

\item[RandomWPMobility]

In the Random Waypoint mobility model the nodes move in line segments. For each
line segment, a random destination position (distributed uniformly over the
playground) and a random speed is chosen. You can define a speed as a variate
from which a new value will be drawn for each line segment; it is customary to
specify it as \ttt{uniform(minSpeed, maxSpeed)}. When the node reaches the
target position, it waits for the time \fpar{waitTime} which can also be defined as a
variate. After this time the the algorithm calculates a new random position, etc.

\item[GaussMarkovMobility] The Gauss-Markov model contains a tuning
parameter, that control the randomness in the movement of the node.
Let the magnitude and direction of speed of the node at the $n$th time step be
$s_n$ and $d_n$. The next speed and direction is computed as

$$ s_{n+1} = \alpha s_n + (1 - \alpha) \bar{s} +
             \sqrt{(1-\alpha^2)} s_{x_n} $$

$$ d_{n+1} = \alpha s_n + (1 - \alpha) \bar{d} +
             \sqrt{(1-\alpha^2)} d_{x_n} $$

where $\bar{s}$ and $\bar{d}$ are constants representing the mean value
of speed and direction as $n \to \infty$; and $s_{x_n}$ and $d_{x_n}$
are random variables with Gaussian distribution.

Totally random walk (Brownian motion) is obtained by setting $\alpha=0$,
while $\alpha=1$ results a linear motion.

To ensure that the node does not remain at the boundary of the constraint
area for a long time, the mean value of the direction ($\bar{d}$) modified
as the node enters the margin area. For example at the right edge of the
area it is set to 180 degrees, so the new direction is away from the edge.

% FIXME the GaussMarkovMobility module has only one variance parameter.
%       it should have separate speed and direction parameters

\item[MassMobility] 

This is a random mobility model for a mobile host with
a mass. It is the one used in \cite{Perkins99optimizedsmooth}.

\begin{quote}
"An MH moves within the room according to the following pattern. It moves
along a straight line for a certain period of time before it makes a turn.
This moving period is a random number, normally distributed with average of
5 seconds and standard deviation of 0.1 second. When it makes a turn, the
new direction (angle) in which it will move is a normally distributed
random number with average equal to the previous direction and standard
deviation of 30 degrees. Its speed is also a normally distributed random
number, with a controlled average, ranging from 0.1 to 0.45 (unit/sec), and
standard deviation of 0.01 (unit/sec). A new such random number is picked
as its speed when it makes a turn. This pattern of mobility is intended to
model node movement during which the nodes have momentum, and thus do not
start, stop, or turn abruptly. When it hits a wall, it reflects off the
wall at the same angle; in our simulated world, there is little other
choice."
\end{quote}

This implementation can be parameterized a bit more, via the
\fpar{changeInterval}, \fpar{changeAngleBy} and \fpar{changeSpeedBy} parameters.
The parameters described above correspond to the following settings:

\begin{inifile}
changeInterval = normal(5, 0.1)
changeAngleBy = normal(0, 30)
speed = normal(avgSpeed, 0.01)
\end{inifile}

\item[ChiangMobility] Chiang's random walk movement model
(\cite{Chiang98wirelessnetwork}).

In this model, the state of the mobile node in each direction (x and y) can be:

\begin{itemize}
  \item 0: the node stays in its current position
  \item 1: the node moves forward
  \item 2: the node moves backward
\end{itemize}

The $(i,j)$ element of the state transition matrix determines the
probability that the state changes from $i$ to $j$:

$$ \left(
\begin{array}{ccc}
  0 & 0.5 & 0.5 \\
  0.3 & 0.7 & 0 \\
  0.3 & 0 & 0.7
\end{array}
\right) $$

The \nedtype{ChiangMobility} module supports the following parameters:
\begin{itemize}
  \item \fpar{updateInterval} position update interval
  \item \fpar{stateTransitionInterval} state update interval
  \item \fpar{speed}: the speed of the node
\end{itemize}


% FIXME last line of setTargetPosition() contains a sign error, should be
%          	targetPosition = lastPosition + lastSpeed * stateTransitionUpdateInterval;
% FIXME when reflecting at the boundary, state variables should be reflected too

\item[ConstSpeedMobility]

\nedtype{ConstSpeedMobility} does not use one of the standard mobility
approaches. The user can define a velocity for each Host and an update interval. If
the velocity is greater than zero (i.e. the Host is not stationary) the
\nedtype{ConstSpeedMobility} module calculates a random target position for the
Host. Depending to the update interval and the velocity it calculates the number of
steps to reach the destination and the step-size. Every update interval
\nedtype{ConstSpeedMobility} calculates the new position on its way to the
target position and updates the display. Once the target position is reached
\nedtype{ConstSpeedMobility} calculates a new target position.

This component has been taken over from Mobility Framework 1.0a5.

% FIXME this is a special case of RandomWPMobility, remove

\end{description}

\subsection{Replaying trace files}

\begin{description}

\item[BonnMotionMobility] Uses the native file format of
\href{http://www.cs.uni-bonn.de/IV/BonnMotion/}{BonnMotion}.

The file is a plain text file, where every line describes the motion
of one host. A line consists of one or more (t, x, y) triplets of real
numbers, like:

\begin{verbatim}
t1 x1 y1 t2 x2 y2 t3 x3 y3 t4 x4 y4 ...
\end{verbatim}

The meaning is that the given node gets to $(xk,yk)$ at $tk$. There's no
separate notation for wait, so x and y coordinates will be repeated there.

\item[Ns2Mobility] Nodes are moving according to the trace files used
in NS2.
The trace file has this format:

\begin{verbatim}
# '#' starts a comment, ends at the end of line
$node_(<id>) set X_ <x> # sets x coordinate of the node identified by <id>
$node_(<id>) set Y_ <y> # sets y coordinate of the node identified by <id>
$node_(<id>) set Z_ <z> # sets z coordinate (ignored)
$ns at $time "$node_(<id>) setdest <x> <y> <speed>" # at $time start moving
towards <x>,<y> with <speed>
\end{verbatim}

The \nedtype{Ns2MotionMobility} module has the following parameters:

\begin{itemize}
  \item \fpar{traceFile} the Ns2 trace file
  \item \fpar{nodeId} node identifier in the trace file; -1 gets substituted by
  parent module's index
  \item \fpar{scrollX},\fpar{scrollY} user specified translation of the
  coordinates
\end{itemize}

% TODO cleaning the code (e.g. duplicated bounds check in setTargetPosition())
% TODO implement cached file access as in BonnMotionMobility

\item[ANSimMobility] reads trace files of the \href{http://www.ansim.info}{ANSim} Tool. 

The nodes are moving along linear segments described by an XML trace file
conforming to this DTD:

\begin{verbatim}
<!ELEMENT mobility (position_change*)>
<!ELEMENT position_change (node_id, start_time, end_time, destination)>
<!ELEMENT node_id (#PCDATA)>
<!ELEMENT start_time (#PCDATA)>
<!ELEMENT end_time (#PCDATA)>
<!ELEMENT destination (xpos, ypos)>
<!ELEMENT xpos (#PCDATA)>
<!ELEMENT ypos (#PCDATA)>
\end{verbatim}

Parameters of the module:

\begin{itemize}
  \item \fpar{ansimTrace} the trace file
  \item \fpar{nodeId} the \verb!node_id! of this node, -1 gets substituted to
  parent module's index
\end{itemize}

\begin{note}
The \nedtype{ANSimMobility} module process only the \ttt{position\_{}change}
elements and it ignores the \ttt{start\_{}time} attribute. It starts the move
on the next segment immediately.
\end{note}

 
\end{description}




\section{Mobility scripts}

The \nedtype{TurtleMobility} module can be parametrized by a script file
containing LOGO-style movement commands in XML format.

The module has these parameters:

\begin{itemize}
\item \ttt{updateInterval} time interval to update the hosts position
\item \ttt{constraintAreaX}, \ttt{constraintAreaY}, \ttt{constraintAreaWidth},
      \ttt{constraintAreaHeight}: constraint area that the node can not leave
\item \ttt{turtleScript} XML file describing the movements
\end{itemize}

The content of the XML file should conform to the following DTD (can be
found as \ffilename{TurtleMobility.dtd} in the source tree):

\begin{verbatim}
<!ELEMENT movements (movement)*>

<!ELEMENT movement (repeat|set|forward|turn|wait|moveto|moveby)*>
<!ATTLIST movement id NMTOKEN #IMPLIED>

<!ELEMENT repeat (repeat|set|forward|turn|wait|moveto|moveby)*>
<!ATTLIST repeat n CDATA #IMPLIED>

<!ELEMENT set EMPTY>
<!ATTLIST set x     CDATA #IMPLIED
              y     CDATA #IMPLIED
              speed CDATA #IMPLIED
              angle CDATA #IMPLIED
              borderPolicy (reflect|wrap|placerandomly|error) #IMPLIED>

<!ELEMENT forward EMPTY>
<!ATTLIST forward d CDATA #IMPLIED
                  t CDATA #IMPLIED>

<!ELEMENT turn EMPTY>
<!ATTLIST turn angle CDATA #REQUIRED>

<!ELEMENT wait EMPTY>
<!ATTLIST wait t CDATA #REQUIRED>

<!ELEMENT moveto EMPTY>
<!ATTLIST moveto x CDATA #IMPLIED
                 y CDATA #IMPLIED
                 t CDATA #IMPLIED>

<!ELEMENT moveby EMPTY>
<!ATTLIST moveby x CDATA #IMPLIED
                 y CDATA #IMPLIED
                 t CDATA #IMPLIED>
\end{verbatim}

The file contains \ttt{movement} elements, each describing a trajectory.
The \ttt{id} attribute of the \ttt{movement} element can be used to
refer the movement from the ini file using the syntax:

\begin{inifile}
**.mobility.turtleScript = xmldoc("turtle.xml", "movements//movement[@id='1']")
\end{inifile}

The motion of the node is composed of uniform linear segments.
The state of motion is described by the following variables:
\begin{itemize} 
\item \ttt{position}: $(x,y)$ coordinate of the current location of the node
\item \ttt{speed}, \ttt{angle}: magnitude and direction of the node's velocity
\item \ttt{targetPos}: target position of the current line segment. If given
                       the \ttt{speed} and \ttt{angle} is not used
\item \ttt{targetTime} the end time of the current linear motion
\item \ttt{borderPolicy}: one of
    \begin{itemize}
      \item \ttt{reflect} the node reflects at the boundary,
      \item \ttt{wrap} the node appears at the other side of the area,
      \item \ttt{placerandomly} the node placed at a random position of the area,
      \item \ttt{error} signals an error when the node reaches the boundary
    \end{itemize}
\end{itemize}

The \ttt{movement} elements may contain the the following commands:

\begin{itemize}
\item \ttt{repeat(n)} repeats its content n times, or indefinetly if the n attribute
              is omitted.
\item \ttt{set(x,y,speed,angle,borderPolicy)} modifies the state of the node.
\item \ttt{forward(d,t)} moves the node for $t$ time or to the $d$ distance
with the current speed. If both $d$ and $t$ is given, then the current
speed is ignored.
\item \ttt{turn(angle)} increase the angle of the node by $angle$ degrees.
\item \ttt{moveto(x,y,t)} moves to point $(x,y)$ in the given time. If
$t$ is not specified, it is computed from the current speed.
\item \ttt{moveby(x,y,t)} moves by offset $(x,y)$ in the given time. If
$t$ is not specified, it is computed from the current speed.
\item \ttt{wait(t)} waits for the specified amount of time.
\end{itemize}

Attribute values must be given without physical units, distances are assumed
to be given as meters, time intervals in seconds and speeds in meter per seconds.
Attibutes can contain expressions that are evaluated each time the 
command is executed. The limits of the constraint area can be
referenced as \verb!$MINX!, \verb!$MAXX!, \verb!$MINY!, and \verb!$MAXY!.
Random number distibutions generate a new random number when evaluated,
so the script can describe random as well as deterministic scenarios.

To illustrate the usage of the module, we show how some mobility
models can be implemented as scripts:

\begin{itemize}
\item RectangleMobility:

\begin{verbatim}
    <movement>
        <set x="$MINX" y="$MINY" angle="0" speed="10"/>
        <repeat>
            <repeat n="2">
                <forward d="$MAXX-$MINX"/>
                <turn angle="90"/>
                <forward d="$MAXY-$MINY"/>
                <turn angle="90"/>
            </repeat>
        </repeat>
    </movement>
\end{verbatim}

\item Random Waypoint:

\begin{verbatim}
    <movement>
        <repeat>
            <set speed="uniform(20,60)"/>
            <moveto x="uniform($MINX,$MAXX)" y="uniform($MINY,$MAXY)"/>
            <wait t="uniform(5,10)">
        </repeat>
    </movement>
\end{verbatim}

\item MassMobility:

\begin{verbatim}
    <movement>
        <repeat>
            <set speed="uniform(10,20)"/>
            <turn angle="uniform(-30,30)"/>
            <forward t="uniform(0.1,1)"/>
        </repeat>
    </movement>
\end{verbatim}

\end{itemize}


%%% Local Variables:
%%% mode: latex
%%% TeX-master: "usman"
%%% End:



\cleardoublepage

% based on 'integration' branch: a4af10cd976d27c3af546230f4e27e3688cf29ad

\chapter{IPv4}
\label{cha:ipv4}


\section{Overview}

The IP protocol is the workhorse protocol of the TCP/IP protocol suite.
All UDP, TCP, ICMP packets are encapsulated into IP datagrams and
transported by the IP layer.
While higher layer protocols transfer data among two communication end-point,
the IP layer provides an hop-by-hop, unreliable and connectionless delivery
service. IP does not maintain any state information about the individual
datagrams, each datagram handled independently.

The nodes that are connected to the Internet can be either a host or a router.
The hosts can send and recieve IP datagrams, and their operating system
implements the full TCP/IP stack including the transport layer. On the
other hand, routers have more than one interface cards and perform packet
routing between the connected networks. Routers does not need the
transport layer, they work on the IP level only. The division
between routers and hosts is not strict, because if a host
have several interfaces, they can usually be configured to operate
as a router too.

Each node on the Internet has a unique IP address. IP datagrams contain
the IP address of the destination. The task of the routers is to find
out the IP address of the next hop on the local network, and forward
the packet to it. Sometimes the datagram is larger, than the maximum
datagram that can be sent on the link (e.g. Ethernet has an 1500 bytes limit.).
In this case the datagram is split into fragments and each fragment is
transmitted independently. The destination host must collect all fragments,
and assemble the datagram, before sending up the data to the transport
layer.

\subsection{INET modules}

The INET framework contains several modules to build the
IPv4 network layer of hosts and routers:
\begin{itemize}
  \item \nedtype{IPv4} is the main module that implements RFC791. This
        module performs IP encapsulation/decapsulation, fragmentation
        and assembly, and routing of IP datagrams.
  \item The \nedtype{RoutingTable} is a helper module that manages the routing
        table of the node. It is queried by the \nedtype{IPv4} module
        for best routes, and updated by the routing daemons implementing
        RIP, OSPF, Manet, etc. protocols.
  \item The \nedtype{ICMP} module can be used to generate ICMP error packets. It also
        supports ICMP echo applications.
  \item The \nedtype{ARP} module performs the dynamic translation of IP addresses
        to MAC addresses. 
\end{itemize}

These modules are assembled into a complete network layer module
called \nedtype{NetworkLayer}. This module has
dedicated gates for TCP, UDP, SCTP, RSVP, OSPF, Manet, and Ping
higher layer protocols. It can be connected to several network
interface cards: Ethernet, PPP, Wlan, or external interfaces.
The \nedtype{NetworkLayer} module is used to build IPv4 hosts
(\nedtype{StandardHost}) and routers (\nedtype{Router}).

The implementation of these modules are based on the following RFCs:
\begin{itemize}
  \item RFC791: Internet Protocol
  \item RFC792: Internet Control Message Protocol
  \item RFC826: Address Resolution Protocol
  \item RFC1122: Requirements for Internet Hosts - Communication Layers
\end{itemize}

The subsequent sections describe the IPv4 modules in detail.

\section{The IPv4 Module}

The \nedtype{IPv4} module implements the IPv4 protocol.

For connecting the upper layer protocols the \nedtype{IPv4} module
has \emph{transportIn[]} and \emph{transportOut[]} gate vectors.

The IP packets are sent to the \nedtype{ARP} module through the
\emph{queueOut} gate. The incoming IP packets are received
directly from the network interface cards through the
\emph{queueIn[]} gates. Each interface card knows its own
network layer gate index.

The C++ class of the \nedtype{IPv4} module is derived from \cppclass{QueueBase}.
There is a processing time associated with each incoming packet.
This processing time is specified by the \fpar{procDelay} module parameter.
If a packet arrives, when the processing of a previous has not been
finished, it is placed in a FIFO queue.

The current performance model assumes that each datagram is processed
within the same time, and there is no priority between the datagrams.
If you need a more sophisticated performance model, you may change
the module implementation (the IP class), and:
\begin{enumerate}
  \item override the \ffunc{startService()} method which determines processing
        time for a packet, or
  \item use a different base class.
\end{enumerate}

\subsection{IP packets}

IP datagrams start with a variable length IP header.
The minimum length of the header is 20 bytes, and
it can contain at most 40 bytes for options, so
the maximum length of the IP header is 60 bytes.

\begin{center}
\begin{bytefield}{32}
\bitheader{0,3,4,7,8,15,16,18,19,23,24,31} \\
\bitbox{4}{Version} &
\bitbox{4}{IHL} &
\bitbox{8}{\small Type of Service} &
\bitbox{16}{Total Length} \\
\bitbox{16}{Identification} &
\bitbox{3}{Flags} &
\bitbox{13}{Fragment Offset} \\
\bitbox{8}{Time to Live} &
\bitbox{8}{Protocol} &
\bitbox{16}{Header Checksum} \\
\bitbox{32}{Source Address} \\
\bitbox{32}{Destination Address} \\
\bitbox{24}{Options} &
\bitbox{8}{Padding} \\
\end{bytefield}
\end{center}

The \ttt{Version} field is 4 for IPv4. The 4-bit \ttt{IHL} field is the
number of 32-bit words in the header. It is needed because the header
may contain optional fields, so its length may vary. The minimum IP header
length is 20, the maximum length is 60. The header is always padded to
multiple of 4 bytes. The \ttt{Type of Service} field designed to store
priority and preference values of the IP packet, so applications can
request low delay, high throughput, and maximium reliability from the
routing algorithms. In reality these fields are rarely set by applications,
and the routers mostly ignore them. The \ttt{Total Length} field is the
length of the whole datagram in bytes. The \ttt{Identification} field
is used for identifying the datagram sent by a host. It is usually generated
by incrementing a counter for each outgoing datagram. When the datagram
gets fragmented by a router, its \ttt{Identification} field is kept unchanged
to the other end can collect them. In datagram fragments the \ttt{Fragment Offset}
is the address of the fragment in the payload of the original datagram. It is
measured in 8-byte units, so fragment lengths must be a multiple of 8.
Each fragment except the last one, has its \ttt{MF} (more fragments) bit set
in the \ttt{Flags} field. The other used flag in \ttt{Flags} is the \ttt{DF}
(don't fragment) bit which forbids the fragmentation of the datagram.
The \ttt{Time to Live} field is decremented by each router in the path,
and the datagram is dropped if it reached 0. Its purpose is to prevent
endless cycles if the routing tables are not properly configured, but
can be used for limiting hop count range of the datagram (e.g. for local
broadcasts, but the \fprog{traceroute} program uses this field too).
The \ttt{Protocol} field is for demultiplexing the payload of the IP
datagram to higher level protocols. Each transport protocol has a registered
protocol identifier. The \ttt{Header Checksum} field is the 16-bit one's
complement sum of the header fields considered as a sequence of 16-bit numbers.
The \ttt{Source Address} and \ttt{Destination Address} are the IPv4 addresses
of the source and destination respectively.

The \ttt{Options} field contains 0 or more IP options. It is always padded
with zeros to a 32-bit boundary. An option is either a single-byte option
code or an option code + option length followed by the actual values for
the option. Thus IP implementations can skip unknown options.

An IP datagram is represented by the \msgtype{IPv4Datagram} message class.
It contains variables corresponding the fields of the IP header, except:
\begin{itemize}
  \item \fvar{Header Checksum} omitted, modeled by error bit of packets
  \item \fvar{Options} only the following options are permitted and the
                       datagram can contain at most one option:
        \begin{itemize}
          \item Loose Source Routing
          \item Strict Source Routing
          \item Timestamp
          \item Record Route
        \end{itemize}
\end{itemize}

The \fvar{Type of Service} field is called \ttt{diffServCodePoint} in
\nedtype{IPv4Datagram}.

Before sending the \msgtype{IPv4Datagram} through the network, the \nedtype{IPv4}
module attaches a \cppclass{IPv4RoutingDecision} control info.
The control info contains the IP address of the next hop, and the
identifier of the interface it should be sent. The ARP module translate
the IP address to the hardware address on the local net of the specified
interface and forwards the datagram to the interface card.

\subsection{Interface with higher layer}

Higher layer protocols should be connected to the \ttt{transportIn}/\ttt{transportOut}
gates of the \nedtype{IPv4} module.

\subsubsection*{Sending packets}

Higher layer protocols can send a packet by attaching a \cppclass{IPv4ControlInfo}
object to their packet and sending it to the \nedtype{IPv4} module.

% receiving IP datagrams from higher layer?

The following fields must be set in the control info:
\begin{itemize}
  \item \fvar{procotol}: the \ttt{Protocol} field of the IP datagram. Valid values
        are defined in the \ttt{IPProtocolId} enumeration.
  \item \fvar{destAddr}: the \ttt{Destination Address} of the IP datagram.
\end{itemize}

Optionally the following fields can be set too:
\begin{itemize}
\item \fvar{scrAddr}: \ttt{Source Address} of the IP datagram. If given it must match with the
      address of one of the interfaces of the node, but the datagram is not necessarily
      routed through that interface. If left unspecified, then the address of the
      outgoing interface will be used.
\item \fvar{timeToLive}: TTL of the IP datagram or -1 (unspecified). If unspecified then the TTL
      of the datagram will be 1 for destination addresses in the
      224.0.0.0 -- 224.0.0.255 range. (Datagrams with these special multicast addresses
      do not need to go further that one hop, routers does not forward these datagrams.)
      Otherwise the TTL field is determined by the \fpar{defaultTimeToLive} or
      \fpar{defaultMCTimeToLive} module parameters depending whether the destination
      address is a multicast address or not.
\item \fvar{dontFragment}: the \ttt{Don't Fragment} flag of the outgoing datagram (default is \fkeyword{false})
\item \fvar{diffServCodePoint}: the \ttt{Type of Service} field of the outgoing datagram.
      (ToS is called \ttt{diffServCodePoint} in \msgtype{IPv4Datagram} too.)
\item \fvar{interfaceId}: id of outgoing interface (can be used to limit broadcast or restrict routing).
\item \fvar{nextHopAddr}: explicit routing info, used by Manet DSR routing. If specified, then
      \ttt{interfaceId} must also be specified. Ignored in Manet routing is disabled.
\end{itemize}

The IP module encapsulates the transport layer datagram into an \msgtype{IPv4Datagram}
and fills in the header fields according to the control info. The \ttt{Identification}
field is generated by incrementing a counter.

The generated IP datagram is passed to the routing algorithm. The routing decides if the
datagram should be delivered locally, or passed to one of the network interfaces
with a specified next hop address, or broadcasted on one or all of the network interfaces.
The details of the routing is described in the next subsection (\ref{subsec:ip_routing})
in detail.

Before sending the datagram on a specific interface, the \nedtype{IPv4} module
checks if the packet length is smaller than the \ttt{MTU} of the interface.
If not, then the datagram is fragmented. When the \ttt{Don't Fragment} flag
forbids fragmentation, an \ttt{Destination Unreachable} ICMP error is generated
with the \ttt{Fragmentation Error (5)} error code.
\begin{note}
Each fragment will encapsulate the whole higher layer datagram, although the
length of the IP datagram corresponds to the fragment length.
\end{note}

The fragments are sent to the \nedtype{ARP} module through the \ttt{queueOut} gate.
The \nedtype{ARP} module forwards the datagram immediately to point-to-point interface
cards. If the outgoing interface is a 802.x card, then before forwarding the datagram
it performs address resolution to obtain the MAC address of the destination.

% FIXME there is no fragmentation if the packet is delivered through a local interface (e.g. loopback).
%       check if this is correct. (Loopback IF has an MTU too.)

% fragmentAndSend()
% FIXME: when a fragment is fragmented, then 'Fragment Offset' is not set correctly (starts from 0, instead of the fragment's Fragment Offset)
% FIXME: 'Fragment Offset' should be measured in 8-byte offsets! Now it is handled in byte units (and serialized so) 
% FIXME: TTL<=0 should be checked in fragmentAndSend() before fragmenting. (sendDatagramToOutput only called from fragmentAndSend())

\subsubsection*{Receiving packets}

The \nedtype{IPv4} module of hosts processes the datagrams received from the network
in three steps:
\begin{enumerate}
  \item Reassemble fragments
  \item Decapsulate the transport layer datagram
  \item Dispatch the datagram to the appropriate transport protocol
\end{enumerate}

When a fragment received, it is added to the fragment buffer of the IP.
If the fragment was the last fragment of a datagram, the processing of 
the datagram continues with step 2. The fragment buffer stores the reception
time of each fragment. Fragments older than \fpar{fragmentTimeout} are
purged from the buffer. The default value of the timeout is 60s. The
timeout is only checked when a fragment is received, and at least 10s
elapsed since the last check.

An \msgtype{IPv4ControlInfo} attached to the decapsulated transport layer packet.
The control info contains fields copied from the IP header (source and destination
address, protocol, TTL, ToS) as well as the interface id through it was received.
The control info also stores the original IP datagram, because the transport
layer might signal an ICMP error, and the ICMP packet must encapsulate the
erronous IP datagram.
\begin{note}
IP datagrams containing a DSR packet are not decapsulated, the unchanged IP
datagram is passed to the DSR module instead.
\end{note}

After decapsulation, the transport layer packet will be passed to the appropriate
transport protocol. It must be connected to one of the \ttt{transportOut[]} gate.
The \nedtype{IPv4} module finds the gate using the \ttt{protocol id}$\rightarrow$
\ttt{gate index} mapping given in the \fpar{protocolMapping} string parameter.
The value must be a comma separated list of ''<protocol\_id>:<gate\_index>'' items.
For example the following line in the ini file maps TCP (6) to gate 0, UDP (17)
to gate 1, ICMP (1) to gate 2, IGMP (2) to gate 3, and RVSP (46) to gate 4.
\begin{inifile}
**.ip.protocolMapping="6:0,17:1,1:2,2:3,46:4"
\end{inifile}
If the protocol of the received IP datagram is not mapped, or the gate
is not connected, the datagram will be silently dropped.
% FIXME should send DESTINATION_UNREACHABLE/PROTOCOL_UNREACHABLE
% FIXME reassembleAndDeliver() checks that transport gate is connected, but handleReceivedICMP() does not check it

Some protocols are handled differently:
\begin{itemize}
  \item \ttt{ICMP}: ICMP errors are delivered to the protocol
        whose packet triggered the error. Only ICMP query
        requests and responses are sent to the \nedtype{ICMP} module.
  \item \ttt{IP}: sent through \ttt{preRoutingOut} gate. (bug!)
  \item \ttt{DSR}: ??? (subsection about Manet routing?)
\end{itemize}

% FIXME reassembleAndDeliver(): packets with IP protocol are sent through 'preRoutingOut' gate,
%       but there is no such gate in the IPv4 module. 


\subsection{Routing, and interfacing with lower layers}
\label{subsec:ip_routing}

The output of the network interfaces are connected to the
\ttt{queueIn} gates of the \nedtype{IPv4} module. The incoming
packets are either IP datagrams or ARP responses. The IP datagrams
are processed by the \nedtype{IPv4} module, the ARP
responses are forwarded to the \nedtype{ARP}. 

The \nedtype{IPv4} module first checks the error bit of the
incoming IP datagrams. There is a $header length/packet length$
probability that the IP header contains the error (assuming
1~bit error). With this probability an ICMP \ttt{Parameter Problem}
generated, and the datagram is dropped.

% FIXME if IP datagram hasBitError(), but it is decided not be in the IP header,
%       then the decapsulated packet should have the bit error.

When the datagram does not contain error in the IP header,
its \ttt{Time to Live} field is decremented and a routing decision
is made. As a result of the routing the datagram is either
delivered locally, or sent out one or more output interface.
When it is sent out, the routing algorithm must compute the
next hop of its route. The details are differ, depending on
that the destination address is multicast address or not.

When the datagram is decided to be sent up, it is processed
as described in the previous subsection (Receiving packets).
If it is decided to be sent out through some interface, it
is actually sent to the \nedtype{ARP} module through the
\ttt{queueOut} gate. An \msgtype{IPv4RoutingDecision} control
info is attached to the outgoing packet, containing the
outgoing interface id, and the IP address of the next hop.
The \nedtype{ARP} module resolve the IP address to a hardware
address if needed, and forwards the datagram to next hop.

\subsubsection*{Unicast/broadcast routing}

Datagrams having unicast or broadcast destination addresses are
routed in the following steps:

\begin{enumerate}
  \item Process source routing options.
  \item Deliver datagram locally. If the destination address is a local
  address, the limited broadcast address (255.255.255.255), or a local
  broadcast address, then it will be sent to the transport layer.
  \item Drop packets received from the network when IP forwarding is disabled.
  \item Route datagrams received from higher layer or network. There are
  three cases:
  \begin{enumerate}
    \item The datagram comes from the transport layer and its destination is
    the limited or a local broadcast address.
    In this case the datagram is sent through the specified
    broadcast interface. If the transport layer did not specify
    the interface, then a copy of the datagram is sent through each interface
    (except loopbacks).
    (The \fpar{forceBroadcast} parameter must be \fkeyword{true} to enable this.) 
    The next hop address will be 255.255.255.255, which is mapped to a link layer
    broadcast address if the interface supports it. 
    \item The datagram comes from the transport layer and the
    transport protocol provided explicit routing (Manet routing).
    If Manet routing is enabled, then the datagram will be sent
    through the specified interface to the specified next hop.
    If Manet routing is disabled or the next hop is not specified
    (only the outgoing interface), then the datagram is sent through
    that interface. If the specified interface is a broadcast interface,
    then the next hop is computed by looking up the best route
    from the routing table.
    \item The datagram received from the network, or it comes from
    the transport layer, but the outgoing interface is not specified.
    In this case, the best route to the destination is looked up from the
    routing table. The datagram forwarded to gateway of the
    route or directly to the destination. If no route is found, then
    a \ttt{Destination Unreachable} ICMP error is sent to the source of the
    datagram.
  \end{enumerate}
\end{enumerate}

% FIXME forceBroadcast branch is never reached (bug #366)
% FIXME in forceBroadcast branch dataAux sould be passed to fragmentAndSend
% FIXME local broadcasts from HL does not looped back, it is necessary because network cards do not receive the messages they send
% FIXME when a broadcast message is locally delivered and its source address is unspecified,
%       the destination address is copied to the source, before sending it up.
%       But broadcast addresses are not allowed as source. (copy-paste bug?)

\subsubsection*{Multicast routing}

If the destination address of the datagram is a multicast address (category D address),
then it is routed specially. The differences from normal routing rules are:
\begin{itemize}
  \item It can be delivered locally if the destination address
  is in the multicast group of some interface.
  \item The datagram is forwarded to each \emph{multicast route}
  found in the routing table.
\end{itemize}

More specifically, the routing routine for multicast datagrams performs these steps:
\begin{enumerate}
  \item Discard packets that arrived at an interface that does not belong
  to the best route to the source of the packet (i.e. did not arrived on the 
  shortest path).
  \item Discard incoming packets that can not be delivered locally and
  can not be forwarded.
  A non-local packet can not be forwarded if IP forwarding is disabled or the
  destination is a link local multicast address (224.0.0.x).
  \item Deliver the datagram locally. If the destination address of the
  datagram belongs to a multicast group of any local interface, it is sent
  up to the transport layer.
  \item Forward the multicast datagram. If the packet comes from the
  higher layer and its outgoing interface specified, then it is sent out
  on the specified interface only (with next hop address = destitation address).
  Otherwise a copy of the datagram is sent on each interface described by
  multicast routes to the destination. In this case the next hop address will be
  the gateway address of the route. If the original datagram is received
  from the network, then it is not sent on the interface it arrived at.
\end{enumerate}
        

\subsection{Parameters}

The \nedtype{IPv4} module has the following parameters:
\begin{itemize}
  \item \fpar{procDelay} processing time of each incoming datagram.
  \item \fpar{timeToLive} default TTL of unicast datagrams.
  \item \fpar{multicastTimeToLive} default TTL of multicast datagrams.
  \item \fpar{protocolMapping} string value containing the \ttt{protocol id}
        $\rightarrow$ \ttt{gate index} mapping, e.g. \ttt{``6:0,17:1,1:2,2:3,46:4''}.
  \item \fpar{fragmentTimeout} the maximum duration until fragments are kept
          in the fragment buffer.
  \item \fpar{forceBroadcast} if \fkeyword{true}, then link-local broadcast
          datagrams are sent out through each interface, if the higher
          layer did not specify the outgoing interface.
\end{itemize}

% compile time options: WITH\_MANET, NEWFRAGMENT

\subsection{Statistics}

The \nedtype{IPv4} module does not write any statistics into files,
but it has some statistical information that can be watched during
the simulation in the gui environment.
\begin{itemize}
  \item \ttt{numForwarded}: number of forwarded datagrams, i.e. sent to one of the
        interfaces (not broadcast), counted before fragmentation.
  \item \ttt{numLocalDeliver}: number of datagrams locally delivered.
        (Each fragment counted separately.)
  \item \ttt{numMulticast}: number of routed multicast datagrams.
  \item \ttt{numDropped} number of dropped packets.
        Either because there is no any interface, the interface is not specified and
        no \fpar{forceBroadcast}, or received from the network but IP forwarding disabled.
  \item \ttt{numUnroutable}: number of unroutable datagrams, i.e. there is no
        route to the destination. (But if outgoing interface is specified it is routed!)
\end{itemize}

In the graphical interface the bubble of the \nedtype{IPv4} module
also displays these counters.


\section{The RoutingTable module}

The \nedtype{RoutingTable} module represents the routing table.
IP hosts and routers contain one instance of this class. It has
methods to manage the routing table and the interface table,
so one can achieve functionality similar to the \fprog{route} and
\fprog{ifconfig} commands.

This is a simple module without gates, it requires function calls to it
(message handling does nothing). Methods are provided for reading and
updating the interface table and the route table, as well as for unicast
and multicast routing.

Interfaces are dynamically registered: at the start of the simulation,
every L2 module adds its own interface entry to the table.

The route table is read from a file; the file can
also fill in or overwrite interface settings. The route table can also
be read and modified during simulation, typically by routing protocol
implementations (e.g. OSPF).

Entries in the route table are represented by \cppclass{IPv4Route} objects.
\cppclass{IPv4Route} objects can be polymorphic: if a routing protocol needs
to store additional data, it can simply subclass from \cppclass{IPv4Route},
and add the derived object to the table. The \cppclass{IPv4Route} object
has the following fields:
\begin{itemize}
  \item \ttt{host} is the IP address of the target of the route (can be a host or network).
                   When an entry searched for a given destination address, the destination
                   address is compared with this \ttt{host} address using the \ttt{netmask}
                   below, and the longest match wins.
  \item \ttt{netmask} used when comparing \ttt{host} with the detination address.
                     It is 0.0.0.0 for the default route, 255.255.255.255 for
                     host routes (exact match), or the network or subnet mask
                     for network routes.
  \item \ttt{gateway} is the IP address of the gateway for indirect routes, or
                      0.0.0.0 for direct routes. Note that 0.0.0.0 can be used
                      even if the destination is not directly connected to this
                      node, but can be found using proxy ARP. 
  \item \ttt{interface} the outgoing interface to be used with this route.
  \item \ttt{type} \ttt{DIRECT} or \ttt{REMOTE}. For direct routes, the next hop
                   address is the destination address, for remote routes it is
                   the gateway address.
  \item \ttt{source} \ttt{MANUAL}, \ttt{IFACENETMASK}, \ttt{RIP}, \ttt{OSPF},
        \ttt{BGP}, \ttt{ZEBRA}, \ttt{MANET}, or \ttt{MANET2}. \ttt{MANUAL} means
        that the route was added by a routing file, or a network configurator.
        \ttt{IFACENETMASK} routes are added for each interface of the node.
        Other values means that the route is managed by the specific routing
        daemon.
  \item \ttt{metric} the ``cost'' of the route. Currently not used when choosing
                     the best route.
\end{itemize}

The \nedtype{RoutingTable} module has the following parameters:

\begin{itemize}
  \item \fpar{routerId}: for routers, the router id using IPv4 address dotted notation;
        specify ``auto'' to select the highest interface address; should be left empty ``''
        for hosts
  \item \fpar{IPForward}: turns IP forwarding on/off (It is always \fkeyword{true}
                          in a \nedtype{Router} and is \fkeyword{false} by default
                          in a \nedtype{StandardHost}.)
  \item \fpar{routingFile}: routing table file name
\end{itemize}

% FIXME RoutingTable::invalidateCache() should clear localBroadcastAddresses.
% RoutingTable::findBestMatchingRoute() should search in this order:
%          1. host routes (exact match)
%          2. network routes (longest match)
%          3. default routes (round robin)
%   It is ok, if host routes has 255.255.255.255 netmask, and default has 0.0.0.0 netmask.
% FIXME RoutingTable::findBestMatchingRoute() if(...MANET...) branch always set bestRoute to NULL,
%       because if there were exact match, it would have been choosen in the previous loop.

\subsection{The IP routing files}

Routing files are files with \ttt{.irt} or \ttt{.mrt} extension,
and their names are passed in the routingFileName parameter
to RoutingTable modules. RoutingTables are present in all
IP nodes (hosts and routers).

Routing files may contain network interface configuration and static
routes. Both are optional. Network interface entries in the file
configure existing interfaces; static routes are added to the route table.

Interfaces themselves are represented in the simulation by modules
(such as the PPP module). Modules automatically register themselves
with appropriate defaults in the RoutingTable, and entries in the
routing file refine (overwrite) these settings.
Interfaces are identified by names (e.g. ppp0, ppp1, eth0) which
are normally derived from the module's name: a module called
\ttt{"ppp[2]"} in the NED file registers itself as interface ppp2.

An example routing file (copied here from one of the example simulations):

\begin{verbatim}
ifconfig:

# ethernet card 0 to router
name: eth0   inet_addr: 172.0.0.3   MTU: 1500   Metric: 1  BROADCAST MULTICAST
Groups: 225.0.0.1:225.0.1.2:225.0.2.1

# Point to Point link 1 to Host 1
name: ppp0   inet_addr: 172.0.0.4   MTU: 576   Metric: 1

ifconfigend.

route:
172.0.0.2   *           255.255.255.255  H  0   ppp0
172.0.0.4   *           255.255.255.255  H  0   ppp0
default:    10.0.0.13   0.0.0.0          G  0   eth0

225.0.0.1   *           255.255.255.255  H  0   ppp0
225.0.1.2   *           255.255.255.255  H  0   ppp0
225.0.2.1   *           255.255.255.255  H  0   ppp0

225.0.0.0   10.0.0.13   255.0.0.0        G  0   eth0

routeend.
\end{verbatim}

The \ttt{ifconfig...ifconfigend.} part configures interfaces,
and \ttt{route..routeend.} part contains static routes.
The format of these sections roughly corresponds to the output
of the \ttt{ifconfig} and \ttt{netstat -rn} Unix commands.

An interface entry begins with a \ttt{name:} field, and lasts until
the next \ttt{name:} (or until \ttt{ifconfigend.}). It may
be broken into several lines.

Accepted interface fields are:

\begin{itemize}
  \item \ttt{name:} - arbitrary interface name (e.g. eth0, ppp0)
  \item \ttt{inet\_addr:} - IP address
  \item \ttt{Mask:} - netmask
  \item \ttt{Groups:} Multicast groups. 224.0.0.1 is added automatically,
     and 224.0.0.2 also if the node is a router (IPForward==true).
  \item \ttt{MTU:} - MTU on the link (e.g. Ethernet: 1500)
  \item \ttt{Metric:} - integer route metric
  \item flags: \ttt{BROADCAST}, \ttt{MULTICAST}, \ttt{POINTTOPOINT}
\end{itemize}

The following fields are parsed but ignored: \ttt{Bcast},\ttt{encap},
\ttt{HWaddr}.

Interface modules set a good default for MTU, Metric (as $2*10^9$/bitrate) and
flags, but leave \fvar{inet\_addr} and \fvar{Mask} empty. \fvar{inet\_addr} and
\fvar{mask} should be set either from the routing file or by a dynamic network
configuration module.

The route fields are:

\begin{verbatim}
Destination  Gateway  Netmask  Flags  Metric Interface
\end{verbatim}

\fvar{Destination}, \fvar{Gateway} and \fvar{Netmask} have the usual meaning.
The \fvar{Destination} field should either be an IP address or ``default''
(to designate the default route). For \fvar{Gateway}, \ttt{*} is also
accepted with the meaning \ttt{0.0.0.0}.

\fvar{Flags} denotes route type:

\begin{itemize}
  \item \textit{H} ``host'': direct route (directly attached to the router), and
  \item \textit{G} ``gateway'': remote route (reached through another router)
\end{itemize}

\fvar{Interface} is the interface name, e.g. \ttt{eth0}.

% FIXME 'H' and 'G' flags should be independent. Now they excludes each other, the parser sets route.type to the last one.
%       H = host/network
%       G = indirect/direct

\subsection{Network configurators}

Configuring a large network with routing files can be a tedious task.
INET contains a module (called \nedtype{FlatNetworkConfigurator})
for automatically assigning IP addresses and filling the routing tables
of the IP nodes of a network. Add this module to the top level
of the network, and it will do the work when initialized. Do not
specify any routing file, or leave them empty, because they can
interfere with the configurator.

The \nedtype{FlatNetworkConfigurator} searches each IP nodes of the network.
(IP nodes are those modules that have the @node NED property and
has a \nedtype{RoutingTable} submodule named ``routingTable'').
The configurator then assigns IP addresses to the IP nodes, controlled
by the following module parameters:
\begin{itemize}
  \item \fpar{netmask} common netmask of the addresses (default is 255.255.0.0)
  \item \fpar{networkAddress} higher bits are the network part of the addresses,
        lower bits should be 0. (default is 192.168.0.0)
\end{itemize}

With the default parameters the assigned addresses are in the range
192.168.0.1 - 192.168.255.254, so there can be maximum 65534 nodes in the
network. The netmask of the assigned IP addresses will be 255.255.255.255,
therefore each node is a subnet in itself; there are no subnet directed
broadcast addresses. The same IP address will be assigned to each interface
of the node, except the loopback interface which always has address 127.0.0.1
(with 255.0.0.0 mask).

After assigning the IP addresses, the configurator fills in the routing tables.
There are two kind of routes:
\begin{itemize}
  \item default routes: for nodes that has only one non-loopback interface
        a route is added that matches with any destination address
        (the entry has 0.0.0.0 \ttt{host} and \ttt{netmask} fields).
        These are remote routes, but the gateway address is left unspecified.
        The delivery of the datagrams rely on the proxy ARP feature of the
        routers. 
  \item direct routes following the shortest paths: for nodes that has more
        than one non-loopback interface a separate route is added to each
        IP node of the network. The outgoing interface is chosen by the
        shortest path to the target node. These routes are
        added as direct routes, even if there is no direct link with the
        destination. In this case proxy ARP is needed to deliver the datagrams.
\end{itemize}

% FIXME shortest path calculation should exclude the nodes where IPForward is disabled.
%       If the shortest path leads to a multihomed node in which IPForward is false, each datagram will be dropped.

% FIXME weird FlatNetworkConfigurator behaviour.
%       Assigned IP addresses does not mirror the hierachy of networks (e.g. each node in an Ethernet LAN handled as a one-element subnet).
%       No gateway address is set in the routes, delivery relies on proxy ARPing.
%       Direct routes created to each node, even if there is no direct link to it.
%       Different interfaces of a node should have different IP address.
%       Broadcast capable interfaces should have a real netmast (not 255.255.255.255) to support subnet directed IP broadcasts.

\section{The ICMP module}

The Internet Control Message Protocol (ICMP) is the error reporting and
diagnostic mechanism of the Internet.
It uses the services of IP, so it is a transport layer protocol, but unlike
TCP or UDP it is not used to transfer user data. It can not be separated
from the IP, because the routing errors are reported by ICMP.

The \nedtype{ICMP} module can be used to send error messages and ping
request. It can also respond to incoming ICMP messages.

Each ICMP message is encapsulated within an IP datagram, so its delivery
is unreliable.

\begin{center}
\begin{bytefield}{32}
\bitheader{0,7,8,15,31} \\
\bitbox{8}{Type} &
\bitbox{8}{Code} &
\bitbox{16}{Checksum} \\
\bitbox{32}{Rest of header} \\
\wordbox{2}{Internet Header + 8 bytes of Original Datagram}
\end{bytefield}
\end{center}

The corresponding message class (\msgtype{ICMPMessage}) contains only
the Type and Code fields. The message encapsulates the IP packet that 
triggered the error, or the data of the ping request/reply.

% FIXME type=PARAMETER_PROBLEM, code=0: missing Pointer field from ICMPMessage
%            REDIRECT: Gateway Internet Address
%            ECHO_REQUEST, ECHO_REPLY: Identifier, Sequence Number
%            TIMESTAMP_REQUEST, TIMESTAMP_REPLY: Identifier, Sequence Number, Originate Timestamp, Receive Timestamp, Transmit Timestamp

% FIXME wrong type codes for ICMP_DESTINATION_UNREACHABLE (3), ICMP_ECHO_REQUEST (8), ICMP_ECHO_REPLY (0), ICMP_TIMESTAMP_REQUEST (13), ICMP_TIMESTAMP_REPLY (14)

% FIXME ICMP error should not be send if the original datagram
%         1. is an ICMP error
%         2. was sent to a broadcast or multicast address
%         3. datagram was sent with a link-layer broadcast
%         4. a fragment other than the first
%         5. a datagram whose source address is 0.0.0.0, 127.*.*.*, broadcast or multicast address
%      currently only the 1. and half of 2. checked

The \nedtype{ICMP} module has two methods which can be used by other modules
to send ICMP error messages:
\begin{itemize}
  \item \ffunc[sendErrorMessage]{sendErrorMessage(IPv4Datagram*, ICMPType, ICMPCode)}
        used by the network layer to report erronous IPv4 datagrams. The ICMP header
        fields are set to the given type and code, and the ICMP message will encapsulate
        the given datagram.
  \item \ffunc[sendErrorMessage]{sendErrorMessage(cPacket*, IPv4ControlInfo*, ICMPType, ICMPCode)}
        used by the transport layer components to report erronous packets. The transport
        packet will be encapsulated into an IP datagram before wrapping it into the ICMP message.
\end{itemize}

The \nedtype{ICMP} module can be accessed from other modules of the node by calling
\ffunc{ICMPAccess::get()}.

When an incoming ICMP error message is received, the \nedtype{ICMP} module
sends it out on the \ttt{errorOut} gate unchanged. It is assumed that an
external module is connected to \ttt{errOut} that can process the error
packet. There is a simple module (\nedtype{ErrorHandling}) that simply
logs the error and drops the message. Note that the \nedtype{IPv4} module
does not send REDIRECT, DESTINATION\_UNREACHABLE,
TIME\_EXCEEDED and PARAMETER\_PROBLEM messages to the \nedtype{ICMP} module,
it will send them to the transport layer module that sent the bogus
packet encapsulated in the ICMP message.
\begin{note}
ICMP protocol encapsulates only the IP header + 8 byte following the IP header
from the bogus IP packet. The ICMP packet length computed from this truncated
packet, despite it encapsulates the whole IP message object.
As a consequence, calling \ffunc{decapsulate()} on the ICMP message
will cause an ``packet length became negative'' error. To avoid this,
use \ffunc{getEncapsulatedMsg()} to access the IP packet that caused the ICMP
error. 
\end{note}

The \nedtype{ICMP} module receives ping commands on the \ttt{pingIn}
gate from the application. The ping command can be any packet
having an \cppclass{IPv4ControlInfo} control info. The packet
will be encapsulated with an \msgtype{ICMPMessage} and
handed over to the IP.

If \nedtype{ICMP} receives an echo request from IP, the original
message object will be returned as the echo reply. Of course,
before sending back the object to IP, the source and destination
addresses are swapped and the message type changed to ICMP\_ECHO\_REPLY.

When an ICMP echo reply received, the application message decapsulated
from it and passed to the application through the \ttt{pingOut} gate.
The \cppclass{IPv4ControlInfo} also copied from the \msgtype{ICMPMessage}
to the application message. 

% FIXME ICMP TIMESTAMP requests are processed as ECHO requests

% \section{The IGMP module}

\section{The ARP module}

The \nedtype{ARP} module implements the Address Resolution Protocol (RFC826).
The ARP protocol is designed to translate a local protocol address
to a hardware address. Altough the ARP protocol can be used with
several network protocol and hardware addressing schemes, in practice
they are almost always IPv4 and 802.3 addresses. The INET implementation
of the ARP protocol (the \nedtype{ARP} module) supports only
IP address $\rightarrow$ MAC address translation. 

If a node wants to send an IP packet to a node whose MAC address is unknown,
it broadcasts an ARP frame on the Ethernet network.
In the request its publish its own IP and
MAC addresses, so each node in the local subnet can update their mapping.
The node whose MAC address was requested will respond with an ARP frame
containing its own MAC address directly to the node that sent the
request. When the original node receives the ARP response, it updates 
its ARP cache and sends the delayed IP packet using the learned MAC address.

The frame format of the ARP request and reponse is shown in Figure \ref{fig:ARP_frame}.
In our case the HTYPE (hardware type), PTYPE (protocol type), HLEN (hardware address length)
and PLEN (protocol address length) are constants: HTYPE=Ethernet (1), PTYPE=IPv4 (2048), HLEN=6,
PLEN=4. The OPER (operation) field is 1 for an ARP request and 2 for an ARP response.
The SHA field contains the 48-bit hardware address of the sender, SPA field is
the 32-bit IP address of the sender; THA and TPA are the addresses of the target.
The message class corresponding to the ARP frame is \msgtype{ARPPacket}.
In this class only the OPER, SHA, SPA, THA and TPA fields are stored.
The length of an \msgtype{ARPPacket} is 28 bytes.

\begin{figure}[h]
\begin{center}
\label{fig:ARP_frame}
\begin{bytefield}{16}
\bitheader{0,7,8,15} \\
\bitbox{16}{HTYPE} \\
\bitbox{16}{PTYPE} \\
\bitbox{8}{HLEN} &
\bitbox{8}{PLEN} \\
\bitbox{16}{OPER} \\
\wordbox{3}{SHA} \\
\wordbox{2}{SPA} \\
\wordbox{3}{THA} \\
\wordbox{2}{TPA} \\
\end{bytefield}
\caption{ARP frame}
\end{center}
\end{figure}

The \nedtype{ARP} module receives IP datagrams and ARP responses from \nedtype{IPv4}
on the \ttt{ipIn} gate and transmits IP datagrams and ARP requests on the \ttt{nicOut[]} gates
towards the network interface cards. ARP broadcasts the requests on the local network,
so the NIC's entry in the \nedtype{InterfaceTable} should have \ffunc{isBroadcast()} flag
set in order to participate in the address resolution.

The incoming IP packet should have an attached \cppclass{IPv4RoutingDecision} control
info containing the IP address of the next hop. If the hardware address is found
in the ARP cache, then the packet is transmitted to the addressed interface immediately.
Otherwise the packet is queued and an address resolution takes place.
The \nedtype{ARP} module creates an \msgtype{ARPPacket} object, sets the sender
MAC and IP address to its own address, sets the destination IP address
to the address of the target of the IP datagram, leave the destination MAC address
blank and broadcasts the packet on each network interface with broadcast capability.
Before sending the ARP packet, it retransmission a timer. If the timer expires,
it will retransmit the ARP request, until the maximum retry count is reached.
If there is no response to the ARP request, then the address resolution fails,
and the IP packet is dropped from the queue. Otherwise the MAC address of the
destination is learned and the IP packet can be transmitted on the corresponding
interface.

When an ARP packet is received on the \ttt{ipIn} gate, and the sender's IP
is already in the ARP cache, it is updated with the information in the ARP frame.
Then it is checked that the destination IP of the packet matches with our
address. In this case a new entry is created with the sender addresses in the
ARP cache, and if the packet is a request a response is created and sent directly
to the originator. If proxy ARP is enabled, the request can be responded
with our MAC address if we can route IP packets to the destination.

Usually each \nedtype{ARP} module maintains a local ARP cache.
However it is possible to use a global cache. The global cache is filled
in with entries of the IP and MAC addresses of the known interfaces
when the ARP modules are initiated (at simulation time 0).
\nedtype{ARP} modules that are using the global ARP cache
never initiate an address resolution; if an IP address not
found in the global cache, the simulation stops with an error.
However they will respond to ARP request, so the simulation can
be configured so that some \nedtype{ARP}s use local, while others
the global cache.

When an entry is inserted or updated in the local ARP cache,
the simulation time saved in the entry. The mapping in the
entry is not used after the configured \fpar{cacheTimeout}
elapsed. This parameter does not affect the entries of
the global cache however.

% FIXME why the global cache is cleared when an ARP module is deleted?

The module parameters of \nedtype{ARP} are:

\begin{itemize}
  \item \fpar{retryTimeout}: number of seconds ARP waits between retries to resolve an IPv4 address (default is 1s)
  \item \fpar{retryCount}: number of times ARP will attempt to resolve an IPv4 address (default is 3)
  \item \fpar{cacheTimeout}: number of seconds unused entries in the cache will time out (default is 120s)
  \item \fpar{proxyARP}: enables proxy ARP mode (default is \fkeyword{true})
  \item \fpar{globalARP}: use global ARP cache (default is \fkeyword{false})
\end{itemize}

The \nedtype{ARP} module emits four signals:

\begin{itemize}
  \item \ttt{sentReq}: emits 1 each time an ARP request is sent
  \item \ttt{sentReplies}: emits 1 each time an ARP response is sent
  \item \ttt{initiatedResolution}: emits 1 each time an ARP resolution is initiated 
  \item \ttt{failedResolution}: emits 1 each time an ARP resolution is failed
\end{itemize}

These signals are recorded as vectors and their counts as scalars.

% TODO watches, animation effects

\section{The IGMP module}


\section{The NetworkLayer module}

The \nedtype{NetworkLayer} module packs the \nedtype{IP}, \nedtype{ICMP},
\nedtype{ARP}, and \nedtype{IGMP} modules into one compound module.
The compound module defines gates for connecting UDP, TCP, SCTP, RSVP and
OSPF transport protocols. The \ttt{pingIn} and \ttt{pingOut} gates of the
\nedtype{ICMP} module are also available, while its \ttt{errorOut} gate
is connected to an inner \nedtype{ErrorHandling} component that writes
the ICMP errors to the log.

The component can be used in hosts and routers to support IPv4.

\section{The NetworkInfo module}

The \nedtype{NetworkInfo} module can be used to dump detailed information
about the network layer. This module does not send or received messages,
it is invoked by the \nedtype{ScenarioManager} instead. For example
the following \nedtype{ScenarioManager} script dump the routing table
of the \ttt{LSR2} module at simulation time $t=2$ into \ffilename{LSR2\_002.txt}:
\begin{filelisting}
<scenario>
  <at t="2">
    <routing module="NetworkInfo" target="LSR2" file="LSR2_002.txt"/>
  </at>
</scenario>
\end{filelisting}

The module currently support only the \ttt{routing} command which dumps
the routing table. The command has four parameters given as XML attributes:
\begin{itemize}
  \item \ttt{target} the name of the node that owns the routing table to be dumped
  \item \ttt{filename} the name of the file the output is directed to
  \item \ttt{mode} if set to ``a'', the output is appended to the file,
                   otherwise the target is truncated if the file existed
  \item \ttt{compat} if set to ``linux'', then the output is generated
                     in the format of the \ttt{route -n} command of Linux.
                     The output is sorted only if \ttt{compat} is
                     \fkeyword{true}.
\end{itemize}

\section{Applications}

The applications described in this section uses the services of the network
layer only, they do not need transport layer protocols.
They can be used with both IPv4 and IPv6.

\subsection{IP traffic generators}

Traffic generators that connect directly to IP (without using TCP or UDP):
\nedtype{IIPvXTraffixGenerator} (prototype).
 \nedtype{IPvXTrafGen},

Sends IP or IPv6 datagrams to the given address at the given \fpar{sendInterval}.
The \fpar{sendInterval} parameter can be a constant or a random value (e.g. exponential(1)).
If the \fpar{destAddresses} parameter contains more than one address, one
of them is randomly for each packet. An address may be given in the
dotted decimal notation (or, for IPv6, in the usual notation with colons),
or with the module name. (The \cppclass{IPvXAddressResolver} class is used to resolve
the address.) To disable the model, set destAddresses to "".

The \nedtype{IPvXTrafGen} sends messages with length \fpar{packetLength}.
The sent packet is emitted in the \fsignal{sentPk} signal.
The length of the sent packets can be recorded as scalars and vectors.

% FIXME packetLength declared as volatile, but in fact it is not.

The \nedtype{IPvXTrafSink} can be used as a receiver of the packets
generated by the traffic generator. This module emits the packet
in the \fsignal{rcvdPacket} signal and drops it. The \ttt{rcvdPkBytes}
and \ttt{endToEndDelay} statistics are generated from this signal.

The \nedtype{IPvXTrafGen} can also be the peer of the traffic generators;
it handles the received packets exactly like \nedtype{IPvXTrafSink}.

You can see an example usage of these applications in \ffilename{examples/inet/routerperf/omnetpp.ini}
simulaton.

\subsection{The PingApp application}

The \nedtype{PingApp} application
generates ping requests and calculates the packet loss and round trip
parameters of the replies.

Start/stop time, sendInterval etc. can be specified via parameters. An address
may be given in the dotted decimal notation (or, for IPv6, in the usual
notation with colons), or with the module name.
(The \cppclass{IPvXAddressResolver} class is used to resolve the address.)
To disable send, specify empty destAddr.

Every ping request is sent out with a sequence number, and replies are
expected to arrive in the same order. Whenever there's a jump in the
in the received ping responses' sequence number (e.g. 1, 2, 3, 5), then
the missing pings (number 4 in this example) is counted as lost.
Then if it still arrives later (that is, a reply with a sequence number
smaller than the largest one received so far) it will be counted as
out-of-sequence arrival. So the number of really lost pings will be
"lost" minus "out-of-order" (assuming there's no duplicate or bogus reply).

Uses \msgtype{PingPayload} as payload for the ICMP(v6) Echo Request/Reply packets.

\subsubsection*{Parameters}

\begin{itemize}
  \item \fpar{destAddr}: destination address
  \item \fpar{srcAddr}: source address (useful with multi-homing)
  \item \fpar{packetSize}: of ping payload, in bytes (default is 56)
  \item \fpar{sendInterval}: time to wait between pings (can be random, default is 1s)
  \item \fpar{hopLimit}: TTL or hopLimit for IP packets (default is 32)
  \item \fpar{count}: stop after \fpar{count} ping request, 0 means continuously
  \item \fpar{startTime}: send first ping request at \fpar{startTime}
  \item \fpar{stopTime}: time of finish sending, 0 means forever
  \item \fpar{printPing}: dump on stdout (default is \fkeyword{true})
\end{itemize}

\subsubsection*{Signals and Statistics}

\begin{itemize}
  \item \fsignal{endToEndDelay} value of the round trip time
  \item \fsignal{drop} number of dropped packets
  \item \fsignal{outOfOrderArrival} number of packets arrived out-of-order
  \item \fsignal{pingTx} sequence number of the sent ping request
  \item \fsignal{pingRx} sequence number of the received ping response
\end{itemize}

\ttt{pingRTT} stat

% FIXME seqNo should be part of ICMPMessage

%%% Local Variables:
%%% mode: latex
%%% TeX-master: "usman"
%%% End:


\cleardoublepage

\chapter{IPv6 and Mobile IPv6}
\label{cha:ipv6}


\section{Overview}

IPv6 support is implemented by several cooperating modules. The IPv6 module
implements IPv6 datagram handling (sending, forwarding etc). It relies on
\nedtype{RoutingTable6} to get access to the routes. \nedtype{RoutingTable6} also contains the
neighbour discovery data structures (destination cache, neighbour cache,
prefix list -- the latter effectively merged into the route table). Interface
configuration (address, state, timeouts etc) is held in the \nedtype{InterfaceTable},
in \cppclass{IPv6InterfaceData} objects attached to \cppclass{InterfaceEntry}
as its \ttt{ipv6()} member.

The module \nedtype{IPv6NeighbourDiscovery} implements all tasks associated with
neighbour discovery and stateless address autoconfiguration. The data
structures themselves (destination cache, neighbour cache, prefix list)
are kept in \nedtype{RoutingTable6}, and are accessed via public C++ methods.
Neighbour discovery packets are only sent and processed by this module --
when IPv6 receives one, it forwards the packet to \nedtype{IPv6NeighbourDiscovery}.

The rest of ICMPv6 (ICMP errors, echo request/reply etc) is implemented in
the module \nedtype{ICMPv6}, just like with IPv4. ICMP errors are sent into
\nedtype{IPv6ErrorHandling}, which the user can extend or replace to get errors
handled in any way they like.


%%% Local Variables:
%%% mode: latex
%%% TeX-master: "usman"
%%% End:


\cleardoublepage

\chapter{The UDP Model}
\label{cha:udp}


\section{Overview}

The UDP protocol is a very simple datagram transport protocol, which
basically makes the services of the network layer available to the applications.
It performs packet multiplexing and demultiplexing to ports and some basic
error detection only. 

The frame format as described in RFC768:

\begin{center}
\begin{bytefield}{32}
\bitheader{0,7,8,15,16,23,24,31} \\
\bitbox{16}{Source Port} &
\bitbox{16}{Destination Port} \\
\bitbox{16}{Length} &
\bitbox{16}{Checksum} \\
\wordbox{3}{Data}
\end{bytefield}
\end{center}

The ports represents the communication end points that are allocated by the
applications that want to send or receive the datagrams. The ``Data'' field
is the encapsulated application data, the ``Length'' and ``Checksum'' fields
are computed from the data.

The INET framework contains an \nedtype{UDP} module that performs the encapsulation/decapsulation
of user packets, an \nedtype{UDPSocket} class that provides the application the usual
socket interface, and several sample applications. 

These components implement the following statndards:
\begin{itemize}
\item RFC768: User Datagram Protocol
\item RFC1122: Requirements for Internet Hosts -- Communication Layers
\end{itemize}

\section{The UDP module}

The UDP protocol is implemented by the \nedtype{UDP} simple module. 
There is a module interface (\nedtype{IUDP}) that defines the gates of the
\nedtype{UDP} component. In the \nedtype{StandardHost} node, the UDP component
can be any module implementing that interface.

Each UDP module has gates to connect to the IPv4 and IPv6 network layer
(ipIn/ipOut and ipv6In/ipv6Out), and a gate array to connect to the applications
(appIn/appOut).

The UDP module can be connected to several applications, and each application
can use several sockets to send and receive UDP datagrams.
The state of the sockets are stored within the UDP module and the application
can configure the socket by sending command messages to the UDP module.
These command messages are distinguished by their kind and the type of their
control info. The control info identifies the socket and holds the parameters
of the command.

Applications don't have to send messages directly to the UDP module,
as they can use the \cppclass{UDPSocket} utility class, which encapsulates the messaging and
provides a socket like interface to applications.

\subsection{Sending UDP datagrams}

If the application want to send datagrams, it optionally can connect to the destination.
It does this be sending a message with UDP\_C\_CONNECT kind and \cppclass{UDPConnectCommand}
control info containing the remote address and port of the connection.
The UDP protocol is in fact connectionless, so it does not send any packets as a result
of the connect call. When the UDP module receives the connect request,
it simply remembers the destination address and port and use it as default destination
for later sends. The application can send several connect commands to the same socket.

% FIXME currently connect() or bind() is mandatory as the first command,
%       the application cannot send packets or set options otherwise

% FIXME connect() should allow unspecified dest address and -1 port (interpreted as disconnect())

For sending an UDP packet, the application should attach an \cppclass{UDPSendCommand}
control info to the packet, and send it to \nedtype{UDP}. The control info may contain
the destination address and port. If the destination address or port
is unspecified in the control info then the packet is sent to the connected target.

The \nedtype{UDP} module encapsulates the application's packet into an \msgtype{UDPPacket},
creates an appropriate IP control info and send it over ipOut or ipv6Out depending on
the destination address.

The destination address can be the IPv4 local broadcast address (255.255.255.255)
or a multicast address. Before sending broadcast messages, the socket must be configured
for broadcasting. This is done by sending an message to the UDP module. The message
kind is UDP\_C\_SETOPTION and its control info (an \cppclass{UDPSetBroadcastCommand})
tells if the broadcast is enabled. You can limit the multicast to the local network
by setting the TTL of the IP packets to 1. The TTL can be configured per socket,
by sending a message to the UDP with an \cppclass{UDPSetTimeToLive} control info
containing the value. If the node has multiple interfaces, the application can
choose which is used for multicast messages. This is also a socket option, the
id of the interface (as registered in the interface table) can be given in an
\cppclass{UDPSetMulticastInterfaceCommand} control info.

% FIXME currently sending broadcast messages is enabled without setting SO_BROADCAST to true,
%       this is not so in UNIX

% FIXME there should be a separate TTL for multicast (not used for unicast), default value is 1
%       see IP_MULTICAST_TTL in `man 7 ip`

\begin{note}
The \nedtype{UDP} module supports only local broadcasts (using the special 255.255.255.255 address).
Packages that are broadcasted to a remote subnet are handled as undeliverable messages.
\end{note}

If the UDP packet cannot be delivered because nobody listens on the destination port,
the application will receive a notification about the failure. The notification is 
a message with UDP\_I\_ERROR kind having attached an \cppclass{UDPErrorIndication}
control info. The control info contains the local and destination address/port,
but not the original packet.

After the application finished using a socket, it should close it by sending a message
UDP\_C\_CLOSE kind and \cppclass{UDPCloseCommand} control info. The control info
contains only the socket identifier. This command frees the resources associated
with the given socket, for example its socket identifier or bound address/port.

\subsection{Receiving UDP datagrams}

Before receiving UDP datagrams applications should first ``bind'' to the given UDP port.
This can be done by sending a message with message kind UDP\_C\_BIND attached with an
\cppclass{UDPBindCommand} control info. The control info contains the socket identifier
and the local address and port the application want to receive UDP packets.
Both the address and port is optional. If the address is unspecified, than the UDP
packets with any destination address is passed to the application. If the port is
-1, then an unused port is selected automatically by the UDP module.
The localAddress/localPort combination must be unique.

When a packet arrives from the network, first its error bit is checked. Erronous messages
are dropped by the UDP component. Otherwise the application bound to the destination port
is looked up, and the decapsulated packet passed to it. If no application is bound to
the destination port, an ICMP error is sent to the source of the packet. If the socket is
connected, then only those packets are delivered to the application, that received from
the connected remote address and port.

The control info of the decapsulated packet is an \cppclass{UDPDataIndication}
and contains information about the source and destination address/port, the TTL,
and the identifier of the interface card on which the packet was received.

The applications are bound to the unspecified local address, then they receive any packets
targeted to their port. UDP also supports multicast and broadcast addresses; if they
are used as destination address, all nodes in the multicast group or subnet receives the packet.
The socket receives the broadcast packets only if it is configured for broadcast.
To receive multicast messages, the socket must join to the group of the multicast address.
This is done be sending the UDP module an UDP\_C\_SETOPTION message with
\cppclass{UDPJoinMulticastGroupCommand} control info. The control info specifies the 
multicast address and the interface identifier. If the interface identifier is given
only those multicast packets are received that arrived at that interface.
The socket can stop receiving multicast messages if it leaves the multicast group.
For this purpose the application should send the UDP another UDP\_C\_SETOPTION
message in their control info (\cppclass{UDPLeaveMulticastGroupCommand}) specifying
the multicast address of that group.

% TODO clarify: multicast packets should not be delivered to connected sockets?

\subsection{Signals}

The \nedtype{UDP} module emits the following signals:
\begin{itemize}
  \item \fsignal{sentPk} when an UDP packet sent to the IP, the packet
  \item \fsignal{rcvdPk} when an UDP packet received from the IP, the packet
  \item \fsignal{passedUpPk} when a packet passed up to the application, the packet
  \item \fsignal{droppedPkWrongPort} when an undeliverable UDP packet received, the packet
  \item \fsignal{droppedPkBadChecksum} when an erronous UDP packet received, the packet
\end{itemize}

\section{UDP sockets}

UDPSocket is a convenience class, to make it easier to send and receive
UDP packets from your application models. You'd have one (or more)
UDPSocket object(s) in your application simple module class, and call
its member functions (bind(), connect(), sendTo(), etc.) to create and
configure a socket, and to send datagrams.

UDPSocket chooses and remembers the sockId for you, assembles and sends command
packets such as UDP\_C\_BIND to UDP, and can also help you deal with packets and
notification messages arriving from UDP.

Here is a code fragment that creates an UDP socket and sends a 1K packet
over it (the code can be placed in your handleMessage() or activity()):

\begin{cpp}
UDPSocket socket;
socket.setOutputGate(gate("udpOut"));
socket.connect(IPvXAddress("10.0.0.2"), 2000);

cPacket *pk = new cPacket("dgram");
pk->setByteLength(1024);
socket.send(pk);

socket.close();
\end{cpp}

% when the localAddr is unspecified by the socket (~ INADDR_ANY), then the kernel sets the
% source address field of the outgoing packet according to the outgoing interface

Processing messages sent up by the UDP module is relatively straightforward.
You only need to distinguish between data packets and error notifications,
by checking the message kind (should be either UDP\_I\_DATA or UDP\_I\_ERROR),
and casting the control info to UDPDataIndication or UDPErrorIndication.
USPSocket provides some help for this with the \ffunc{belongsToSocket()} and
\ffunc{belongsToAnyUDPSocket()} methods.

\begin{cpp}
void MyApp::handleMessage(cMessage *msg)
{
    if (msg->getKind() == UDP_I_DATA)
    {
        if (socket.belongsToSocket())
            processUDPPacket(PK(msg));
    }
    else if (msg->getKind() == UDP_I_ERROR)
    {
        processUDPError(msg);
    }
    else
    {
        error("Unrecognized message (%s)", msg->getClassName());
    }
}
\end{cpp}

\section{UDP applications}

All UDP applications should be derived from the \nedtype{IUDPApp} module interface,
so that the application of \nedtype{StandardHost} could be configured without changing its NED file.

The following applications are implemented in INET:
\begin{itemize}
\item \nedtype{UDPBasicApp} sends UDP packets to a given IP address at a given interval
\item \nedtype{UDPBasicBurst} sends UDP packets to the given IP address(es) in bursts, or acts as a packet sink.
\item \nedtype{UDPEchoApp} similar to \nedtype{UDPBasicApp}, but it sends back the packet after reception
\item \nedtype{UDPSink} consumes and prints packets received from the \nedtype{UDP} module
\item \nedtype{UDPVideoStreamCli},\nedtype{UDPVideoStreamSvr} simulates UDP streaming
\end{itemize}

The next sections describe these applications in details.

\subsection{UDPBasicApp}

The \nedtype{UDPBasicApp} sends UDP packets to a the IP addresses given in the
\fpar{destAddresses} parameter. The application sends a message to one of the
targets in each \fpar{sendInterval} interval. The interval between message and
the message length can be given as a random variable. Before the packet is
sent, it is emitted in the \fsignal{sentPk} signal.

The application simply prints the received UDP datagrams. The \fsignal{rcvdPk}
signal can be used to detect the received packets.

The number of sent and received messages are saved as scalars at the end of the
simulation.

% could be a simple packet generator without the ability to receive packets?

\subsection{UDPSink}

This module binds an UDP socket to a given local port, and prints the
source and destination and the length of each received packet.

% TODO does not accept broadcast messages

\subsection{UDPEchoApp}

Similar to \nedtype{UDPBasicApp}, but it sends back the packet after reception.
It accepts only packets with \msgtype{UDPEchoAppMsg} type, i.e. packets that
are generated by another \nedtype{UDPEchoApp}.

When an echo response received, it emits an \fsignal{roundTripTime} signal.

\subsection{UDPVideoStreamCli}

This module is a video streaming client. It send one ``video streaming request'' to
the server at time \fpar{startTime} and receives stream from \nedtype{UDPVideoStreamSvr}.

The received packets are emitted by the \fsignal{rcvdPk} signal.

\subsection{UDPVideoStreamSvr}

This is the video stream server to be used with \nedtype{UDPVideoStreamCli}.

The server will wait for incoming "video streaming requests".
When a request arrives, it draws a random video stream size
using the \fpar{videoSize} parameter, and starts streaming to the client.
During streaming, it will send UDP packets of size \fpar{packetLen} at every
\fpar{sendInterval}, until \fpar{videoSize} is reached. The parameters \fpar{packetLen}
and \fpar{sendInterval} can be set to constant values to create CBR traffic,
or to random values (e.g. sendInterval=uniform(1e-6, 1.01e-6)) to
accomodate jitter.

The server can serve several clients, and several streams per client.

% FIXME why streamVector? VideoStreamData could be deleted immediately after last byte sent
% TODO this is video-on-demand, support multicast/broadcast video streaming too

\subsection{UDPBasicBurst}

Sends UDP packets to the given IP address(es) in bursts, or acts as a
packet sink. Compatible with both IPv4 and IPv6.

\subsubsection*{Addressing}

The \fpar{destAddresses} parameter can contain zero, one or more destination
addresses, separated by spaces. If there is no destination address given,
the module will act as packet sink. If there are more than one addresses,
one of them is randomly chosen, either for the whole simulation run,
or for each burst, or for each packet, depending on the value of the
\fpar{chooseDestAddrMode} parameter. The \fpar{destAddrRNG} parameter controls which
(local) RNG is used for randomized address selection.
The own addresses will be ignored.
 
An address may be given in the dotted decimal notation, or with the module 
name. (The \cppclass{IPvXAddressResolver} class is used to resolve the address.)
You can use the "Broadcast" string as address for sending broadcast messages.

INET also defines several NED functions that can be useful: 
\begin{itemize}
\item[-] moduleListByPath("pattern",...): \\
         Returns a space-separated list of the modulenames.
         All modules whole getFullPath() matches one of the pattern parameters will get included.
         The patterns may contain wilcards in the same syntax as in ini files.
         See cTopology::extractByModulePath() function
         example: destaddresses = moduleListByPath("**.host[*]", "**.fixhost[*]")
\item[-] moduleListByNedType("fully.qualified.ned.type",...): \\ 
         Returns a space-separated list of the modulenames with the given NED type(s).
         All modules whose getNedTypeName() is listed in the given parameters will get included.
         The NED type name is fully qualified.
         See cTopology::extractByNedTypeName() function
         example: destaddresses = moduleListByNedType("inet.nodes.inet.StandardHost")
\end{itemize}

The peer can be UDPSink or another UDPBasicBurst.

\subsubsection*{Bursts}

The first burst starts at \fpar{startTime}. Bursts start by immediately sending 
a packet; subsequent packets are sent at \fpar{sendInterval} intervals. The 
sendInterval parameter can be a random value, e.g. exponential(10ms).
A constant interval with jitter can be specified as 1s+uniform(-0.01s,0.01s)
or uniform(0.99s,1.01s). The length of the burst is controlled by the
\fpar{burstDuration} parameter. (Note that if \fpar{sendInterval} is greater than 
\fpar{burstDuration}, the burst will consist of one packet only.) The time between
burst is the \fpar{sleepDuration} parameter; this can be zero (zero is not
allowed for \fpar{sendInterval}.) The zero \fpar{burstDuration} is interpreted as infinity.

\subsubsection*{Packets}

Packet length is controlled by the \fpar{messageLength} parameter.

The module adds two parameters to packets before sending:
\begin{itemize}
\item[-] sourceID: source module ID
\item[-] msgId: incremented by 1 after send any packet.
\end{itemize}
When received packet has this parameters, the module checks the order of received packets.

\subsubsection*{Operation as sink}

When \fpar{destAddresses} parameter is empty, the module receives packets and makes statistics only.

\subsubsection*{Statistics}

Statistics are collected on outgoing packets:
\begin{itemize}
\item[-] sentPk: packet object
\end{itemize}

Statistics are collected on incoming packets:
\begin{itemize}
\item[-] outOfOrderPk: statistics of out of order packets.
       The packet is out of order, when has msgId and sourceId parameters and module
       received bigger msgId from same sourceID.
\item[-] dropPk: statistics of dropped packets. 
       The packet is dropped when not out-of-order packet and delay time is larger than
       delayLimit parameter. The delayLimit=0 is infinity.
\item[-] rcvdPk: statistics of not dropped, not out-of-order packets.
\item[-] endToEndDelay: end to end delay statistics of not dropped, not out-of-order packets.
\end{itemize}

%%% Local Variables:
%%% mode: latex
%%% TeX-master: "usman"
%%% End:


\cleardoublepage

\chapter{The TCP Models}
\label{cha:tcp}


\section{Overview}

Blah blah blah

\subsection{TCP segments}

The TCP segment as described in RFC793:

\begin{center}
\begin{bytefield}{32}
\bitheader{0,3,4,7,8,15,16,31} \\
\bitbox{16}{Source Port} &
\bitbox{16}{Destination Port} \\
\bitbox{32}{Sequence Number} \\
\bitbox{32}{Acknowledgment Number} \\
\bitbox{4}{\small Data Offset} &
\bitbox{6}{Reserved} &
\bitbox{6}{Flags} &
\bitbox{16}{Window} \\
\bitbox{16}{Checksum} &
\bitbox{16}{Urgent Pointer} \\
\bitbox{24}{Options} &
\bitbox{8}{Padding} \\
\wordbox{3}{Data}
\end{bytefield}
\end{center}

Here
\begin{itemize}
  \item the Source and Destination Ports, together with the Source and Destination
  addresses of the IP header identifies the communication endpoints.
  \item the Sequence Number identifier of the first data byte transmitted in the sequence,
  Sequence Number + 1 identifies the second byte, so on.
  \item the Acknowlegment Number refers to the next byte (if the ACK flag is set) expected
  by the receiver using its sequence number
  \item the Data Offset is the length of the TCP header in 32-bit words (needed because the
  Options field has variable length)
  \item the Reserved bits are unused
  \item the Flags field composed of 6 bits:
  \begin{itemize}
    \item URG: Urgent Pointer field is significant
    \item ACK: Acknowledgment field is significant
    \item PSH: Push Function
    \item RST: Reset the connection
    \item SYN: Synchronize sequence number
    \item FIN: No more data from sender
  \end{itemize}
  \item the Window is the number of bytes the receiver TCP can accept (because of its
  limited buffer)
  \item the Checksum is the 1-complement sum of the 16-bit words of the IP/TCP header and
  data bytes
  \item the Urgent Pointer is the offset of the urgent data (if URG flag is set)
  \item the Options field is variable length, it can occupy 0-40 bytes in the header and is
  always padded to a multiple of 4 bytes.
\end{itemize}

\subsection{TCP connections}

establishment/release

state diagram

\subsection{Flow control}

window update

window scale option (RFC1323)

persistence timer: recover after window update segment lost (and current window is 0)

keepalive timer

\subsection{Transmission policies}

\subsubsection*{Retransmissions}

retransmission timer

computing RTT mean and mean deviation (D)

Jacobson's formula: timeout = RTT + 4 * D

Karn's modification: do not update RTT on any segments that have been retransmitted
                     the timeout is doubled on each failure until the segment gets through

\subsubsection*{Delayed ACK algorithm}

RFC1122 4.2.3.2

A host that is receiving a stream of TCP data segments can
increase efficiency in both the Internet and the hosts by
sending fewer than one ACK (acknowledgment) segment per data
segment received; this is known as a "delayed ACK" [TCP:5].

Delay is max. 500ms.

A delayed ACK gives the application an opportunity to
update the window and perhaps to send an immediate
response.  In particular, in the case of character-mode
remote login, a delayed ACK can reduce the number of
segments sent by the server by a factor of 3 (ACK,
window update, and echo character all combined in one
segment).

In addition, on some large multi-user hosts, a delayed
ACK can substantially reduce protocol processing
overhead by reducing the total number of packets to be
processed [TCP:5].  However, excessive delays on ACK's
can disturb the round-trip timing and packet "clocking"
algorithms [TCP:7].

RFC2581 3.2

a TCP receiver SHOULD send an immediate ACK
when the incoming segment fills in all or part of a gap in the
sequence space.

\subsubsection*{Nagle's algorithm}

RFC896 describes the ``small packet problem": when the application
sends single-byte messages to the TCP, and it transmitted immediatly
in a 41 byte TCP/IP packet (20 bytes IP header, 20 bytes TCP header,
1 byte payload), the result is a 4000\% overhead that can cause
congestion in the network.

The solution to this problem is to delay the transmission until
enough data received from the application and send all collected
data in one packet. Nagle proposed that
when a TCP connection has outstanding data that has not
yet been acknowledged, small segments should not be sent
until the outstanding data is acknowledged.

In the \nedtype{TCP} module this behaviour can be enabled
by setting the \fpar{nagleEnabled} parameter to true.

\subsubsection*{Selective Acknowledgments}

RFC2018

RFC2883

\subsubsection*{Silly window avoidance}

The Silly Window Syndrome (SWS) is described in RFC813. It occurs when
a TCP receiver advertises a small window and the TCP sender immediately
sends data to fill the window. Let's take the example when the sender
process writes a file into the TCP stream in big chunks, while the
receiver process reads the bytes one by one. The first few bytes
are transmitted as whole segments until the receiver buffer
becomes full. Then the application reads one
byte, and a window size 1 is offered to the sender. The sender sends
a segment with 1 byte payload immediately, the receiver buffer becomes
full, and after reading 1 byte, the offered window is 1 byte again.
Thus almost the whole file is transmitted in very small segments.

In order to avoid SWS, both sender and receiver must try to avoid this
situation. The receiver must not advertise small windows and the sender
must not send small segments when only a small window is advertised.

In RFC813 it is offered that
\begin{enumerate}
  \item the receiver should not advertise windows that is smaller than the maximum
        segment size of the connection
  \item the sender should wait until the window is large enough for a maximum sized
        segment. 
\end{enumerate}

\subsection{Congestion control}

describe congestion

congestion window: sender-side limit on the amount of data the sender can transmit
                   into the network before receiving an ACK

slow start: initial congestion window is MSS, double on each successful transmission.
When treshold reached increment only by 1 MSS.


RFC2581

Definitions:
SMSS: sender maximum segment size
RMSS: receiver maximum segment size (default 536)
rwnd: most recently advertised receiver window
IW: initial sender's congestion window
LW: loss window, size of congestion window after a TCP sender detects loss
RW: restart window, size of congestion window after a TCP restarts transmission after an idle period
fligth size: amount of data has been sent but not yet acknowledged
cwnd: congestion window, sender-size limit on the amount of data the sender
      can transmit into the network before receiving an ACK
rwnd: receiver advertised window, receiver-side limit on the amount of outstanding data
sstresh: whether slow start or congestion avoidance used

IW <= 2*MSS


\subsubsection*{Slow Start and Congestion Avoidance}

slow start: cwnd < ssthresh, congestion avoidance: cwnd > ssthresh

Initial window (IW) is at most 2 MSS bytes, no more than 2 segments. 

During slow start the cwnd is increased by at most 1 SMSS bytes for each ACK received.

During congestion avoidance, cwnd is incremented by 1 full-sized segment
per round-trip-time (RTT). One formula commonly used to update cwnd during
congestion is $ cwnd += SMSS*SMSS/cwnd $. This adjustment is executed on every
incoming non-duplicate ACK. (0 rounded up to 1)

When a TCP sender detects segment loss using the retransmission timer,
the value of sthresh is set to no more than $ max(FlightSize/2, 2*SMSS) $.
(Flight size is the amount of data has been sent but not yet acknowledged.)
At the same time, cwnd is set to LW (the loss window, no more than 1MSS).

\subsubsection*{Fast Retransmit and Fast Recovery}

send a duplicate ACK immedatly when out-of-order segment received

send immediate ACK when the segment fills in a gap in the segment space

the sender retransmit the segments immediately when 3 duplicated ACK received

Fast recovery: no slow start after loss detected as fast retransmit

   1.  When the third duplicate ACK is received, set ssthresh to no more
       than the value given in equation 3.

   2.  Retransmit the lost segment and set cwnd to ssthresh plus 3*SMSS.
       This artificially "inflates" the congestion window by the number
       of segments (three) that have left the network and which the
       receiver has buffered.

   3.  For each additional duplicate ACK received, increment cwnd by
       SMSS.  This artificially inflates the congestion window in order
       to reflect the additional segment that has left the network.

   4.  Transmit a segment, if allowed by the new value of cwnd and the
       receiver's advertised window.

   5.  When the next ACK arrives that acknowledges new data, set cwnd to
       ssthresh (the value set in step 1).  This is termed "deflating"
       the window.

       This ACK should be the acknowledgment elicited by the
       retransmission from step 1, one RTT after the retransmission
       (though it may arrive sooner in the presence of significant out-
       of-order delivery of data segments at the receiver).
       Additionally, this ACK should acknowledge all the intermediate
       segments sent between the lost segment and the receipt of the
       third duplicate ACK, if none of these were lost.

\subsection{Implemented standards}

This implementation supports:
\begin{itemize}
\item RFC 793 - Transmission Control Protocol
\item RFC 896 - Congestion Control in IP/TCP Internetworks
\item RFC 1122 - Requirements for Internet Hosts -- Communication Layers
\item RFC 1323 - TCP Extensions for High Performance
\item RFC 2018 - TCP Selective Acknowledgment Options
\item RFC 2581 - TCP Congestion Control
\item RFC 2883 - An Extension to the Selective Acknowledgement (SACK) Option for TCP
\item RFC 3042 - Enhancing TCP's Loss Recovery Using Limited Transmit
\item RFC 3390 - Increasing TCP's Initial Window
\item RFC 3517 - A Conservative Selective Acknowledgment (SACK)-based Loss Recovery
                 Algorithm for TCP
\item RFC 3782 - The NewReno Modification to TCP's Fast Recovery Algorithm
\end{itemize}


\section{TCP module}

The \nedtype{TCP} simple module is the main implementation of the TCP protocol in the INET framework.
Other implementation are described in section \ref{sec:other_tcp}.
The \nedtype{TCP} module as other transport protocols work above the network layer and below the application
layer, therefore it has gates to be connected with the IPv4 or IPv6 network (ipIn/ipOut or ipv6In/ipv6Out),
and with the applications (appIn[k], appOut[k]).
One \nedtype{TCP} module can serve several application modules, and several
connections per application. The $k$th application connects to \nedtype{TCP}'s
\ttt{appIn[k]} and \ttt{appOut[k]} ports. When talking to applications, a
connection is identified by the \textit{(application port index, connId)} pair,
where \textit{connId} is assigned by the application in the OPEN call.

The TCP module usually specified by its module interface
(\nedtype{ITCP}) in the NED definition of hosts, so it can be replaced with any implementation
that communicates through the same gates. The \nedtype{TCP} model relies on
sending and receiving \cppclass{IPControlInfo} objects
attached to TCP segment objects as control info (see \ffunc{cMessage::setControlInfo()}).

\cppclass{TCP} subclassed from \cppclass{cSimpleModule}. It manages socketpair-to-connection
mapping, and dispatches segments and user commands to the appropriate cppclass{TCPConnection} object.

\subsection{TCP packets}

The INET framework models the TCP header with the \msgtype{TCPSegment} message class.
This contains the fields of a TCP frame, except:
\begin{itemize}
  \item Data Offset (number of 32 bit words in the header): represented by cMessage::length()
  \item Reserved (reserved for future use)
  \item Checksum: modelled by cMessage::hasBitError()
  \item Header Options: currently only EOL, NOP, MSS, WS, SACK\_PERMITTED, SACK and TS are implemented
  \item Padding
\end{itemize}

The only options accepted in TCP segments are:
\begin{itemize}
  \item EOL: End of Option List
  \item NOP: No Operation
  \item MSS: Maximum Segment Size
  \item WS: Window Size
  \item SACK\_PERMITTED: Selective Acknowledgment Permitted
  \item SACK: Selective Acknowledgment
  \item TS: Timestamp
\end{itemize}

The Data field can either be represented by (see \cppclass{TCPDataTransferMode}):
\begin{itemize}
  \item its byte count only
  \item the transferred C++ packet objects
  \item raw bytes represented a \cppclass{ByteArray} instance
\end{itemize}

corresponding to transfer modes BYTECOUNT, OBJECT, BYTESTREAM resp.

Limitations

%  - URG and PSH bits not handled. Receiver always acts as if PSH was set
%    on all segments: always forwards data to the app as soon as possible.
%  - no RECEIVE command. Received data are always forwarded to the app as
%    soon as possible, as if the app issued a very large RECEIVE request
%    at the beginning. This means there's currently no flow control
%    between TCP and the app.
%  - all timeouts are precisely calculated: timer granularity (which is caused
%    by "slow" and "fast" i.e. 500ms and 200ms timers found in many *nix TCP
%    implementations) is not simulated
%  - new ECN flags (CWR and ECE). Need to be added to header by [RFC 3168].

\subsection{TCP commands}

The application and the TCP module communicates with each other
by sending \cppclass{cMessage} objects. These messages are specified
in the \ffilename{TCPCommand.msg} file.

The \cppclass{TCPCommandCode} enumeration defines the message kinds
that are sent by the application to the TCP:
\begin{itemize}
  \item TCP\_C\_OPEN\_ACTIVE: active open
  \item TCP\_C\_OPEN\_PASSIVE: passive open
  \item TCP\_C\_SEND: send data
  \item TCP\_C\_CLOSE: no more data to send
  \item TCP\_C\_ABORT: abort connection
  \item TCP\_C\_STATUS: request status info from TCP
\end{itemize}

Each command message should have an attached control info of type \cppclass{TCPCommand}.
Some commands (TCP\_C\_OPEN\_xxx, TCP\_C\_SEND) use subclasses.
The \cppclass{TCPCommand} object has a \fvar{connId} field that identifies the
connection locally within the application. \fvar{connId} is to be chosen by the
application in the open command.

The \cppclass{TCPOpenCommand} control info used for active and passive opens.
The object holds the local and remote address and port of the connection.
In case of passive opens (i.e. \ffunc{listen()} calls) only the local
address/port must be filled in. In case of active open, the remote
address and port are mandatory, while the local address and port can be
unspecified (IP will choose the source address, TCP will choose an ephemeral port).
If the application executes a passive open, it can specify that want to handle
only one connection at a time, or multiple simultanous connections. If the
\fvar{fork} field is true, it emulates the Unix accept(2) semantics: a new
connection structure is created for the connection (with a new connId),
and the connection with the old connection id remains listening.
With \fvar{fork} is false, the first connection is accepted (with the original connId),
and further incoming connections will be refused by the TCP by sending an RST segment.
The \fvar{dataTransferMode} field in \cppclass{TCPOpenCommand} specifies
whether the application data is transmitted as C++ objects, real bytes or byte
counts only. The congestion control algorithm can also be specified
on a per connection basis by setting \fvar{tcpAlgorithmClass} field to the
name of the algorithm.

% TODO describe open in detail: active/passive, fork, transferMode, etc.

When the application receives a message from the TCP, the message kind is
set to one of the \cppclass{TCPStatusInd} values:
\begin{itemize}
  \item TCP\_I\_ESTABLISHED: connection established
  \item TCP\_I\_CONNECTION\_REFUSED: connection refused
  \item TCP\_I\_CONNECTION\_RESET: connection reset
  \item TCP\_I\_TIME\_OUT: connection establish timer went off, or max retransmission count reached
  \item TCP\_I\_DATA: data packet
  \item TCP\_I\_URGENT\_DATA: urgent data packet
  \item TCP\_I\_PEER\_CLOSED: FIN received from remote TCP
  \item TCP\_I\_CLOSED: connection closed normally
  \item TCP\_I\_STATUS: status info
\end{itemize}

These messages also have an attached control info with \cppclass{TCPCommand}
or derived type (TCPConnectInfo, TCPStatusInfo, TCPErrorInfo).
\cppclass{TCPConnectInfo} used with TCP\_I\_ESTABLISHED and 
contains the local and remove IP address and port. In the responses
to status requests \cppclass{TCPStatusInfo} object holds the detailed
status info, such as state, local and remote addresses and ports, mss, etc.

To close, the client sends a \cppclass{cMessage} to \nedtype{TCP}
with the \ttt{TCP\_C\_CLOSE} message kind and \cppclass{TCPCommand}
control info.

\begin{note}
If you do active OPEN, then send data and close before the connection
has reached ESTABLISHED, the connection will go from SYN\_SENT to CLOSED
without actually sending the buffered data. This is consistent with
RFC 793 but may not be what you would expect.
\end{note}

\begin{note}
Handling segments with SYN+FIN bits set (esp. with data too) is
inconsistent across TCPs, so check this one if it is of importance.
\end{note}

% receive() calls are not modeled, incoming data passed to the application right away
% how accurate the modeling of the receiver window?

\subsection{TCP parameters}

\begin{itemize}
  \item \fpar{advertisedWindow} in bytes, corresponds with the maximal receiver buffer capacity (Note: normally, NIC queues should be at least this size, default is  14*mss)
  \item \fpar{delayedAcksEnabled} delayed ACK algorithm (RFC 1122) enabled/disabled
  \item \fpar{nagleEnabled} Nagle's algorithm (RFC 896) enabled/disabled
  \item \fpar{limitedTransmitEnabled} Limited Transmit algorithm (RFC 3042) enabled/disabled (can be used for TCPReno/TCPTahoe/TCPNewReno/TCPNoCongestionControl)
  \item \fpar{increasedIWEnabled} Increased Initial Window (RFC 3390) enabled/disabled
  \item \fpar{sackSupport} Selective Acknowledgment (RFC 2018, 2883, 3517) support (header option) (SACK will be enabled for a connection if both endpoints support it)
  \item \fpar{windowScalingSupport} Window Scale (RFC 1323) support (header option) (WS will be enabled for a connection if both endpoints support it)
  \item \fpar{timestampSupport} Timestamps (RFC 1323) support (header option) (TS will be enabled for a connection if both endpoints support it)
  \item \fpar{mss} Maximum Segment Size (RFC 793) (header option, default is 536)
  \item \fpar{tcpAlgorithmClass} the name of TCP flavour
  
             Possible values are ``TCPReno'' (default), ``TCPNewReno'', ``TCPTahoe'', ``TCPNoCongestionControl'' and ``DumpTCP''.
             In the future, other classes can be written which implement Vegas, LinuxTCP  or other variants.
             See section \ref{sec:tcp_flavours} for detailed description of implemented flavours.

             Note that TCPOpenCommand allows tcpAlgorithmClass to be chosen per-connection.

  \item \fpar{recordStats} if set to false it disables writing excessive amount of output vectors
\end{itemize}

\subsection{Statistics}

The TCP module collects the following vectors:
\begin{compactitem}
  \item \ttt{send window}
  \item \ttt{receive window}
  \item \ttt{advertised window}
  \item \ttt{sent seq}
  \item \ttt{sent ack}
  \item \ttt{rcvd seq}
  \item \ttt{rcvd ack}
  \item \ttt{unacked bytes}
  \item \ttt{rcvd dupAcks}
  \item \ttt{pipe}
  \item \ttt{sent sacks}
  \item \ttt{rcvd sacks}
  \item \ttt{rcvd oooseg}
  \item \ttt{rcvd naseg}
  \item \ttt{rcvd sackedBytes}
  \item \ttt{tcpRcvQueueBytes}
  \item \ttt{tcpRcvQueueDrops}
\end{compactitem}

% TODO definitions

\subsection{Animation effects}

\section{TCP connections}

TCPConnection manages the connection, with the help of other objects.
TCPConnection itself implements the basic TCP "machinery": takes care
of the state machine, stores the state variables (TCB), sends/receives
SYN, FIN, RST, ACKs, etc.

TCPConnection internally relies on 3 objects. The first two are subclassed
from TCPSendQueue and TCPReceiveQueue. They manage the actual data stream,
so TCPConnection itself only works with sequence number variables.
This makes it possible to easily accomodate need for various types of
simulated data transfer: real byte stream, "virtual" bytes (byte counts
only), and sequence of cMessage objects (where every message object is
mapped to a TCP sequence number range).

entry points from TCP:
 processTimer(msg)
 processTCPSegment(packet, srcAddr, dstAddr)
 processAppCommand(msg)

\subsection{Opening connections}

initial sequence number generation

MSL: maximum segment lifetime = 2 minutes

sequence space $2^32$ octet = 4.5hours at 2MB/s, 5.4min at 100MB/s

% FIXME TCP should have a memory of last sequence number for each connection (local/remote address/port pair)
%       to avoid old segments accepted in the new connection
%       if the memory is not avaiable (e.g. after a crash) then it should wait MSL first (optionally, if the app requested)
%       otherwise it should assign an ISS that is greater than last used sequence number if within MSL.
%       The clock based implementation of selectInitialSeqNum() is not enough.
%       See RFC793 3.3.

\subsection{Sending and Receiving Data}

retransmission timout

urgent data

% FIXME urgBit is never set

% FIXME model TCP_NODELAY, there is no PUSH flag in socket.send() (TCP_PUSH option ?)

\subsection{RESET handling}

reset generation

 1. when segment arrives in CLOSED state
 2. If the connection is in any non-synchronized state (LISTEN,
    SYN-SENT, SYN-RECEIVED), and the incoming segment acknowledges
    something not yet sent (the segment carries an unacceptable ACK),
    a reset is sent.
    
reset processing

  If the
  receiver was in the LISTEN state, it ignores it.  If the receiver was
  in SYN-RECEIVED state and had previously been in the LISTEN state,
  then the receiver returns to the LISTEN state, otherwise the receiver
  aborts the connection and goes to the CLOSED state.  If the receiver
  was in any other state, it aborts the connection and advises the user
  and goes to the CLOSED state.

\subsection{Closing connections}

CLOSE is an operation meaning ``I have no more data to send''.


\section{TCP queues}

send queues

receive queues

retransmit queue

mapping segments into the sequence space


\section{TCP flavours}
\label{sec:tcp_flavours}

The TCPAlgorithm object controls
retransmissions, congestion control and ACK sending: delayed acks, slow start,
fast retransmit, etc. are all implemented in TCPAlgorithm subclasses.
This simplifies the design of TCPConnection and makes it a lot easier to
implement new TCP variations such as NewReno, Vegas or LinuxTCP as
TCPAlgorithm subclasses.

Currently implemented TCPAlgorithm classes are TCPReno, TCPTahoe, TCPNewReno,
TCPNoCongestionControl and DumbTCP.

% TODO class diagram

The concrete TCPAlgorithm class to use can be chosen per connection (in OPEN)
or in a module parameter.

Adaptive retransmissions: TCPBaseAlg::rttMeasurementComplete() -- RFC793 3.7

Delayed ACK: each algorithms except DumbTCP applies a 200ms delay
before sending ACK.

Flow control: finite receive buffer size (initiated by parameter
advertisedWindow). If receive buffer is exhausted (by out-of-order
segments) and the payload length of a new received segment
is higher than free receiver buffer, the new segment will be dropped.
Such drops are recorded in tcpRcvQueueDropsVector.

\subsection{DumbTCP}

A very-very basic TCPAlgorithm implementation, with hardcoded
retransmission timeout and no other sophistication. It can be
used to demonstrate what happened if there was no adaptive
timeout calculation, delayed acks, silly window avoidance,
congestion control, etc.

\subsection{TCPBaseAlg}

The abstract \cppclass{TCPBaseAlg} class implements basic TCP
algorithms for adaptive retransmissions, persistence timers,
delayed ACKs, Nagle's algorithm, Increased Initial Window
-- EXCLUDING congestion control. Congestion control
is implemented in subclasses such as TCPTahoe or TCPReno.

Congestion window is set to SMSS when the connection is established,
and not touched after that. Subclasses may redefine any of the virtual
functions here to add their congestion control code.

\begin{note}
Note: currently the timers and time calculations are done in double
and NOT in Unix (200ms or 500ms) ticks. It's possible to write another
TCPAlgorithm which uses ticks (or rather, factor out timer handling to
separate methods, and redefine only those).
\end{note}

\subsection{TCPNoCongestion}

TCP with no congestion control (i.e. congestion window kept very large).
Can be used to demonstrate effect of lack of congestion control.

\subsection{TCPTahoe}

The \cppclass{TCPTahoe} algorithm class extends \cppclass{TCPBaseAlg}
with Slow Start, Congestion Avoidance and Fast Retransmit congestion
control algorithms.

\subsection{TCPReno}

\subsection{TCPNewReno}

\subsection{Implementing new TCPAlgorithms}

\section{TCP socket}

%The \cppclass{TCPSocket} C++ class is provided to simplify managing TCP connections
%from applications. \cppclass{TCPSocket} handles the job of assembling and sending
%command messages (OPEN, CLOSE, etc) to \nedtype{TCP}, and it also simplifies
%the task of dealing with packets and notification messages coming from \nedtype{TCP}.

\cppclass{TCPSocket} is a convenience class, to make it easier to manage TCP connections
from your application models. You'd have one (or more) \cppclass{TCPSocket} object(s)
in your application simple module class, and call its member functions
(bind(), listen(), connect(), etc.) to open, close or abort a TCP connection.

TCPSocket chooses and remembers the connId for you, assembles and sends command
packets (such as OPEN\_ACTIVE, OPEN\_PASSIVE, CLOSE, ABORT, etc.) to TCP,
and can also help you deal with packets and notification messages arriving
from TCP.

A session which opens a connection from local port 1000 to 10.0.0.2:2000,
sends 16K of data and closes the connection may be as simple as this
(the code can be placed in your \ffunc{handleMessage()} or
\ffunc{activity()}):

\begin{cpp}
TCPSocket socket;
socket.connect(IPvXAddress("10.0.0.2"), 2000);

msg = new cMessage("data1");
msg->setByteLength(16*1024);  // 16K
socket.send(msg);

socket.close();
\end{cpp}

% FIXME missing setOutputGate() call

Dealing with packets and notification messages coming from TCP is somewhat
more cumbersome. Basically you have two choices: you either process those
messages yourself, or let TCPSocket do part of the job. For the latter,
you give TCPSocket a callback object on which it'll invoke the appropriate
member functions: \ffunc{socketEstablished()}, \ffunc{socketDataArrived()},
\ffunc{socketFailure()}, \ffunc{socketPeerClosed()},
etc (these are methods of \cppclass{TCPSocket::CallbackInterface}).,
The callback object can be your simple module class too.

This code skeleton example shows how to set up a TCPSocket to use the module
itself as callback object:

\begin{cpp}
class MyModule : public cSimpleModule, public TCPSocket::CallbackInterface
{
    TCPSocket socket;
    virtual void socketDataArrived(int connId, void *yourPtr,
                                   cPacket *msg, bool urgent);
    virtual void socketFailure(int connId, void *yourPtr, int code);
    ...
};

void MyModule::initialize() {
    socket.setCallbackObject(this,NULL);
}

void MyModule::handleMessage(cMessage *msg) {
    if (socket.belongsToSocket(msg))
        socket.processMessage(msg); // dispatch to socketXXXX() methods
    else
        ...
}

void MyModule::socketDataArrived(int, void *, cPacket *msg, bool) {
    ev << "Received TCP data, " << msg->getByteLength() << " bytes\\n";
    delete msg;
}

void MyModule::socketFailure(int, void *, int code) {
    if (code==TCP_I_CONNECTION_RESET)
        ev << "Connection reset!\\n";
    else if (code==TCP_I_CONNECTION_REFUSED)
        ev << "Connection refused!\\n";
    else if (code==TCP_I_TIMEOUT)
        ev << "Connection timed out!\\n";
}
\end{cpp}

If you need to manage a large number of sockets (e.g. in a server
application which handles multiple incoming connections), the
\cppclass{TCPSocketMap} class may be useful. The following code
fragment to handle incoming connections is from the LDP module:

\begin{cpp}
TCPSocket *socket = socketMap.findSocketFor(msg);
if (!socket)
{
    // not yet in socketMap, must be new incoming connection: add to socketMap
    socket = new TCPSocket(msg);
    socket->setOutputGate(gate("tcpOut"));
    socket->setCallbackObject(this, NULL);
    socketMap.addSocket(socket);
}
// dispatch to socketEstablished(), socketDataArrived(), socketPeerClosed()
// or socketFailure()
socket->processMessage(msg);
\end{cpp}


\section{Other TCP implementations}
\label{sec:other_tcp}

\subsection{TCP LWIP}

lwIP is a light-weight implementation of the TCP/IP protocol suite
that was originally written by Adam Dunkels of the Swedish Institute of
Computer Science. The current development homepage is
\url{http://savannah.nongnu.org/projects/lwip/}.

The implementation targets embedded devices: it has
very limited resource usage (it works with tens of kilobytes of RAM and
around 40 kilobytes of ROM) and does not require an underlying OS.

The \nedtype{TCP\_lwIP} model based on the 1.3.2 version of the LWIP sources.

Features:
- round trip time estimation, adaptive retransmission timeout
- fast retransmit and fast recovery
- slow start threshold
- silly window avoidance


% lwIP license file missing from INET source

\subsection{TCP NSC}

TCP model based on the Network Simulation Cradle by Sam Jansen.
The NSC is available on the http://research.wand.net.nz/software/nsc.php page.
You must read the inet/3dparty/README before use this TCP implementation.
This model is compatible with both IPv4 (~IPv4) and ~IPv6.
The TCP\_TRANSFER\_OBJECT data transfer mode isn't implemented yet.
See the \nedtype{ITCP} for the TCP layer general informations.

<b>Settings</b>

stackName: You can select a TCP implementation with the stackName parameter
(On the 64 bit systems, the liblinux2.6.26.so and liblinux2.6.16.so are available only).

stackBufferSize: The buffer size value for selected TCP implementation.
The NSC sets the wmem\_max, rmem\_max, tcp\_rmem, tcp\_wmem parameters to this value
on linux TCP implementations. For details, you can see the NSC documentation.



%%% Local Variables:
%%% mode: latex
%%% TeX-master: "usman"
%%% End:


\cleardoublepage

\chapter{The SCTP Model}
\label{cha:sctp}


\section{Overview}

Blah blah blah


%%% Local Variables:
%%% mode: latex
%%% TeX-master: "usman"
%%% End:


\cleardoublepage

\chapter{Internet Routing}
\label{cha:routing}


\section{Overview}

Blah blah blah


%%% Local Variables:
%%% mode: latex
%%% TeX-master: "usman"
%%% End:


\cleardoublepage

\chapter{The MPLS Models}
\label{cha:mpls}


\section{Overview}

Blah blah blah


%%% Local Variables:
%%% mode: latex
%%% TeX-master: "usman"
%%% End:


\cleardoublepage

\chapter{Applications}
\label{cha:apps}


\section{Overview}

This chapter describes application models and traffic generators.

Blah blah blah


%%% Local Variables:
%%% mode: latex
%%% TeX-master: "usman"
%%% End:


\cleardoublepage

\chapter{History}
\label{cha:History}

\section{IPSuite to INET Framework (2000-2006)}

The predecessor of the INET framework was written by Klaus
Wehrle, Jochen Reber, Dirk Holzhausen, Volker Boehm, Verena Kahmann,
Ulrich Kaage and others at the University of Karlsruhe during 2000-2001,
under the name IPSuite.

The MPLS, LDP and RSVP-TE models were built as an add-on to IPSuite
during 2003 by Xuan Thang Nguyen (Xuan.T.Nguyen@uts.edu.au) and other
students at the University of Technology, Sydney under supervision of
Dr Robin Brown. The package consisted of around 10,000 LOCs, and was
published at http://charlie.it.uts.edu.au/~tkaphan/xtn/capstone (now
unavailable).

After a period of IPSuite being unmaintained, Andras Varga took over
the development in July 2003. Through a series of snapshot releases in
2003-2004, modules got completely reorganized, documented, and many of them
rewritten from scratch. The MPLS models (including RSVP-TE, LDP, etc)
also got refactored and merged into the codebase.

During 2004, Andras added a new, modular and extensible TCP implementation,
application models, Ethernet implementation and an all-in-one IP model
to replace the earlier, modularized one.

The package was renamed INET Framework in October 2004.

Support for wireless and mobile networks got added during summer 2005
by using code from the Mobility Framework.

The MPLS models (including LDP and RSVP-TE) got revised and mostly
rewritten from scratch by Vojta Janota in the first half of 2005
for his diploma thesis. After further refinements by Vojta, the new code
got merged into the INET CVS in fall 2005, and got eventually released
in the March 2006 INET snapshot.

The OSPFv2 model was created by Andras Babos during 2004 for his diploma
thesis which was submitted early 2005. This work was sponsored by Andras Varga,
using revenues from commercial OMNEST licenses. After several refinements and fixes,
the code got merged into the INET Framework in 2005, and became part of the
March 2006 INET snapshot.

The Quagga routing daemon was ported into the INET Framework also by Vojta
Janota. This work was also sponsored by Andras Varga. During fall 2005 and 
the months after, ripd and ospfd were ported, and the methodology of porting 
was refined. Further Quagga daemons still remain to be ported.

Based on experience from the IPv6Suite (from Ahmet Sekercioglu's group at
CTIE, Monash University, Melbourne) and IPv6SuiteWithINET (Andras's effort
to refactor IPv6Suite and merge it with INET early 2005), Wei Yang Ng
(Monash Uni) implemented a new IPv6 model from scratch for the
INET Framework in 2005 for his diploma thesis, under guidance from Andras
who was visiting Monash between February and June 2005. This IPv6 model
got first included in the July 2005 INET snapshot, and gradually refined
afterwards.

The SCTP implementation was contributed by Michael Tuexen, Irene Ruengeler
and Thomas Dreibholz

Support for Sam Jensen's Network Simulation Cradle,
which makes real-world TCP stacks available in simulations was added
by Zoltan Bojthe in 2010.

TCP SACK and New Reno implementation was contributed by Thomas Reschka.

Several other people have contributed to the INET Framework by providing
feedback, reporting bugs, suggesting features and contributing patches;
I'd like to acknowledge their help here as well.



%%% Local Variables:
%%% mode: latex
%%% TeX-master: "usman"
%%% End:


\cleardoublepage

\bibliographystyle{alpha}
\bibliography{inet-manual}


%% no need for the following since 'tocbibind' package
%% \phantomsection
%% \addcontentsline{toc}{chapter}{\indexname}
\printindex

\end{document}

%%% Local Variables:
%%% mode: latex
%%% TeX-master: t
%%% End:
